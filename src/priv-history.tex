\chapter{历史
  %History
}

\section{ 加州大学伯克利分校的研究经费
  % Research Funding at UC Berkeley
}

RISC-V 架构和实现的开发部分由以下赞助商资助。
% Development of the RISC-V architecture and implementations has been
% partially funded by the following sponsors.

\begin{itemize}
\item {\bf Par Lab:} 微软支持的研究(资助\#024263)
  和英特尔(资助\#024894)资助并通过匹配资助U.C. Discovery(资助\#DIG07-10227)。
  Par Lab 的附属公司诺基亚、NVIDIA、甲骨文和三星提供了额外的支持。
% {\bf Par Lab:} Research supported by Microsoft (Award \#024263)
%   and Intel (Award \#024894) funding and by matching funding by
%   U.C. Discovery (Award \#DIG07-10227). Additional support came from
%   Par Lab affiliates Nokia, NVIDIA, Oracle, and Samsung.

\item {\bf Project Isis:} 美国能源部奖 DE-SC0003624。
% DoE Award DE-SC0003624.

\item {\bf ASPIRE Lab}: DARPA PERFECT计划,资助HR0011-12-2-0016。
DARPA POEM计划赞助HR0011-11-C-0100。 
未来架构研究中心 (C-FAR),一个由半导体研究公司资助的STARnet中心。 
来自 ASPIRE 工业赞助商、英特尔和 ASPIRE 附属公司、谷歌、华为、诺基亚、NVIDIA、甲骨文和三星的额外支持。
% DARPA PERFECT program, Award HR0011-12-2-0016.
%   DARPA POEM program Award HR0011-11-C-0100.  The Center for Future
%   Architectures Research (C-FAR), a STARnet center funded by the
%   Semiconductor Research Corporation.  Additional support from ASPIRE
%   industrial sponsor, Intel, and ASPIRE affiliates, Google, Huawei,
%   Nokia, NVIDIA, Oracle, and Samsung.

\end{itemize}

本文内容并不一定反映美国政府的立场或政策,不应推断为官方宣传。
% The content of this paper does not necessarily reflect the position or the
% policy of the US government and no official endorsement should be
% inferred. 


