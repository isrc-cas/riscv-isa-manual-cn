%%%%Nice fonts in diagrams
%%Images
\makeatletter
% for the fig2dev version on P desktop
\gdef\SetFigFont#1#2#3#4#5{%
  \reset@font\fontsize{12}{#2pt}%
%  \fontfamily{#3}\fontseries{#4}\fontshape{#5}%
  \fontfamily{\sfdefault}\fontseries{#4}\fontshape{#5}%
  \selectfont}
\makeatother

\chapter{RVWMO 说明材料(0.1版本)}
\label{sec:memorymodelexplanation}
这节使用了更加非正式的语言和具体的例子,提供了更多关于RVWMO(第~\ref{ch:memorymodel}章)的解释。
这些解释都是为了澄清该公理和保留的程序次序规则的含义和目的。
这个附录应当被视为评注;而所有的规范性材料都在第~\ref{ch:memorymodel}章和ISA规范的主体的其余部分中提供。
当前的所有已知的差异性都被列在了第~\ref{sec:memory:discrepancies}节。任何其它的差异性都是无意的。
% This section provides more explanation for RVWMO (Chapter~\ref{ch:memorymodel}), using more informal language and concrete examples.
% These are intended to clarify the meaning and intent of the axioms and preserved program order rules.
% This appendix should be treated as commentary; all normative material is provided in Chapter~\ref{ch:memorymodel} and in the rest of the main body of the ISA specification.
% All currently known discrepancies are listed in Section~\ref{sec:memory:discrepancies}.
% Any other discrepancies are unintentional.

\section{为什么用 RVWMO?}
\label{sec:whyrvwmo}

内存一致性模型遵循着从弱到强的松散谱系。
弱内存模型允许更多的硬件实现的灵活性,并提供理论上比强模型更好的性能、每瓦特的性能、能量、可扩展性,和硬件验证开销,但代价是更复杂的编程模型。
强模型提供了更简单的编程模型,但是对于可以在流水线和内存系统中执行的各种(非推测性的)硬件优化,要强加更多的约束开销,并且反过来在能量、区域开销和验证负担方面强加一些成本。
% Memory consistency models fall along a loose spectrum from weak to strong.
% Weak memory models allow more hardware implementation flexibility and deliver arguably better performance, performance per watt, power, scalability, and hardware verification overheads than strong models, at the expense of a more complex programming model.
% Strong models provide simpler programming models, but at the cost of imposing more restrictions on the kinds of (non-speculative) hardware optimizations that can be performed in the pipeline and in the memory system, and in turn imposing some cost in terms of power, area overhead, and verification burden.

RISC-V选择了RVWMO内存模型,它是释放一致性的一个变体。这将它置于了内存模型谱系的两个极端之间。
RVWMO内存模型使架构师能够构建简单的实现、激进的实现,将实现深深地嵌入到一个更大的系统之中,并服务于复杂的内存系统交互,或者任何其它的可能性,所有这些同时又能够以足够强大的高性能支持编程语言内存模型。
% RISC-V has chosen the RVWMO memory model, a variant of release consistency.
% This places it in between the two extremes of the memory model spectrum.
% The RVWMO memory model enables architects to build simple implementations, aggressive implementations, implementations embedded deeply inside a much larger system and 
% subject to complex memory system interactions, or any number of other possibilities, all while simultaneously being strong enough to support programming language memory models at high performance.

为了促进来自其它架构的代码的移植,一些硬件实现可以选择实现Ztso扩展,它默认提供了更严格的RVTSO次序的语义。
为RVWMO编写的代码是与RVTSO自动且固有地兼容的,但是假定RVTSO写的代码不保证在RVWMO实现上能够正确地运行。
事实上,大多数RVWMO实现都将(并且应当)简单地拒绝运行RVTSO专用的二进制文件。
每个实现必须因此进行选择,或者优先兼容RVTSO代码(例如,为了便于来自x86的移植),或者反之优先兼容其他实现了RVWMO的RISC-V核。
% To facilitate the porting of code from other architectures, some hardware implementations may choose to implement the Ztso extension, which provides stricter RVTSO ordering semantics by default.
% Code written for RVWMO is automatically and inherently compatible with RVTSO, but code written assuming RVTSO is not guaranteed to run correctly on RVWMO implementations.
% In fact, most RVWMO implementations will (and should) simply refuse to run RVTSO-only binaries.
% Each implementation must therefore choose whether to prioritize compatibility with RVTSO code (e.g., to facilitate porting from x86) or whether to instead prioritize compatibility with other RISC-V cores implementing RVWMO.

在RVTSO下,代码中为RVWMO所写的一些屏障和/或内存次序注释可能变得冗余;
在Ztso实现上默认采用RVWMO的代价是获取那些在实现上已经变成no-op的屏障(例如:FENCE~R,RW 和 FENCE~RW,W)的增量开销。
然而,如果希望兼容非Ztso的实现,这些屏障在代码中仍然必须存在。
% Some fences and/or memory ordering annotations in code written for RVWMO may become redundant under RVTSO; 
% the cost that the default of RVWMO imposes on Ztso implementations is the incremental overhead of fetching those fences (e.g., FENCE~R,RW and FENCE~RW,W) which become no-ops on that implementation.
% However, these fences must remain present in the code if compatibility with non-Ztso implementations is desired.

\section{Litmus 测试}\label{sec:litmustests}
这章的解释使用了{\em litmus测试},或者说,为测试或突出显示内存模型的一个特定部分而设计的小型程序。
图~\ref{fig:litmus:sample}显示了带有两个硬件线程的litmus测试的一个例子。
作为对这个图和对本章中之后所有图的约定,我们假定{\tt s0}-{\tt s2}在所有硬件线程中都被预先设置为相同的值,并且{\tt s0}持有由{\tt x}标签的地址,{\tt s1}持有{\tt y}的,而{\tt s2}持有{\tt z}的,这里{\tt x}、{\tt y}和{\tt z}是对齐到8字节边界的不相交的内存位置。
每张图在左侧显示了litmus测试的代码,在右侧则是一个特定的有效或无效执行的可视化。
% The explanations in this chapter make use of {\em litmus tests}, or small programs designed to test or highlight one particular aspect of a memory model.
% Figure~\ref{fig:litmus:sample} shows an example of a litmus test with two harts.
% As a convention for this figure and for all figures that follow in this chapter, we assume that {\tt s0}--{\tt s2} are pre-set to the same value in all harts and that {\tt s0} holds the address labeled {\tt x}, {\tt s1} holds {\tt y}, and {\tt s2} holds {\tt z}, where {\tt x}, {\tt y}, and {\tt z} are disjoint memory locations aligned to 8 byte boundaries.
% Each figure shows the litmus test code on the left, and a visualization of one particular valid or invalid execution on the right.

\begin{figure}[h!]
  \centering
    \begin{tabular}{m{.4\linewidth}m{.05\linewidth}m{.4\linewidth}}
    \tt\small
    \begin{tabular}{cl||cl}
    \multicolumn{2}{c}{Hart 0} & \multicolumn{2}{c}{Hart 1} \\
    \hline
          & $\vdots$    &     & $\vdots$    \\
          & li t1,1     &     & li t4,4     \\
      (a) & sw t1,0(s0) & (e) & sw t4,0(s0) \\
          & $\vdots$    &     & $\vdots$    \\
          & li t2,2     &     &             \\
      (b) & sw t2,0(s0) &     &             \\
          & $\vdots$    &     & $\vdots$    \\
      (c) & lw a0,0(s0) &     &             \\
          & $\vdots$    &     & $\vdots$    \\
          & li t3,3     &     & li t5,5     \\
      (d) & sw t3,0(s0) & (f) & sw t5,0(s0) \\
          & $\vdots$    &     & $\vdots$    \\
    \end{tabular}
    & &
    \input{figs/litmus_sample.pdf_t}
\end{tabular}
    \caption{一个litmus测试的示例和一个被禁止的执行({\tt a0=1})。}
  \label{fig:litmus:sample}
\end{figure}

Litmus测试被用于理解特定具体情境中的内存模型的含义。
例如,在图~\ref{fig:litmus:sample}的litmus测试中,根据运行时来自各个硬件线程的指令流的动态交错情况,第一个硬件线程中的{\tt a0}的最终的值可以是2、4或5。
然而,在这个例子中,硬件线程0中的{\tt a0}的最终的值将永远都不会是1或3;按直觉,值1在加载执行时将不再可见,而值3在加载执行时尚未成为可见的。
我们下面来分析这个测试和一些其它的测试。
% Litmus tests are used to understand the implications of the memory model in specific concrete situations.
% For example, in the litmus test of Figure~\ref{fig:litmus:sample}, the final value of {\tt a0} in the first hart can be either 2, 4, or 5, depending on the dynamic interleaving of the instruction stream from each hart at runtime.
% However, in this example, the final value of {\tt a0} in Hart 0 will never be 1 or 3; intuitively, the value 1 will no longer be visible at the time the load executes, and the value 3 will not yet be visible by the time the load executes.
% We analyze this test and many others below.

\begin{table}[h]
  \centering\small
  \begin{tabular}{|c|l|}
    \hline
    边       & 全名 (和解释) \\
    \hline
    \sf rf   & 读从(Reads From) (从各存储到返回该存储写入值的加载) \\
    \hline
    \sf co   & 一致性(Coherence) (关于存储到各地址的一个总的次序) \\
    \hline
    \sf fr   & 从读(From-Reads) (从各加载到读取加载所返回值的存储的共同后继) \\
    \hline
    \sf ppo  & 保留程序次序(Preserved Program Order) \\
    \hline
    \sf fence & 通过一个FENCE指令强行采取的排序 \\
    \hline
    \sf addr & 地址依赖(Address Dependency) \\
    \hline
    \sf ctrl & 控制依赖(Control Dependency) \\
    \hline
    \sf data & 数据依赖(Data Dependency) \\
    \hline
  \end{tabular}
  \caption{在这个附录中绘制的litmus测试图表的要点}
  \label{tab:litmus:key}
\end{table}

每个litmus测试的右侧显示的图展示了正在被考虑的特定执行候选的一个可视化的表示。
这些图标使用在内存模型文献中常见的符号,来限制可能的全局内存次序(可能在执行中产生问题)的集合。
它也是附录~\ref{sec:herd}中展示的\textsf{herd}模型的基础。
表~\ref{tab:litmus:key}中解释了该符号。
在列出的关系中,在硬件线程之间、{\sf co}边、{\sf fr}边和{\sf ppo}边之间的{\sf rf}边直接限制了全局内存次序(正如通过{\sf ppo}也可以限制{\sf fence}、{\sf addr}、{\sf data},和一些{\sf ctrl})。
其它边(例如infa-hart {\sf rf}边)是信息性的,但是不会限制全局内存次序。
% The diagram shown to the right of each litmus test shows a visual representation of the particular execution candidate being considered.
% These diagrams use a notation that is common in the memory model literature for constraining the set of possible global memory orders that could produce the execution in question.
% It is also the basis for the \textsf{herd} models presented in Appendix~\ref{sec:herd}.
% This notation is explained in Table~\ref{tab:litmus:key}.
% Of the listed relations, {\sf rf} edges between harts, {\sf co} edges, {\sf fr} edges, and {\sf ppo} edges directly constrain the global memory order (as do {\sf fence}, {\sf addr}, {\sf data}, and some {\sf ctrl} edges, via {\sf ppo}).
% Other edges (such as intra-hart {\sf rf} edges) are informative but do not constrain the global memory order.

例如,在表~\ref{fig:litmus:sample}中,{\tt a0=1}只可能发生在(c)读取由(a)写入的值、且下列情形之一为真的时候:
% For example, in Figure~\ref{fig:litmus:sample}, {\tt a0=1} could occur only if (c) reads the value written by (a) and one of the following were true:
\begin{itemize}
  \item 在全局内存次序中(以及在一致性次序{\sf co}中),(b)出现在(a)之前。然而这违反了RVWMO PPO规则~\ref{ppo:->st}。从(b)到(a)的{\sf co}边突出了这一矛盾。
  % (b) appears before (a) in global memory order (and in the coherence order {\sf co}).  However, this violates RVWMO PPO rule~\ref{ppo:->st}.  The {\sf co} edge from (b) to (a) highlights this contradiction.
  \item 在全局内存次序中(以及在一致性次序{\sf co}中),(a)出现在(b)之前。然而,在这种情况中,加载值公理将被违反,因为在程序次序中,(a)不是先于(c)的最近匹配的存储。从(c)到(b)的{\sf fr}边突出了这一矛盾。
  % (a) appears before (b) in global memory order (and in the coherence order {\sf co}).  However, in this case, the Load Value Axiom would be violated, because (a) is not the latest matching store prior to (c) in program order.  The {\sf fr} edge from (c) to (b) highlights this contradiction.
\end{itemize}
由于这些场景都不满足RVWMO公理,结果{\tt a0=1}就被禁止了。
% Since neither of these scenarios satisfies the RVWMO axioms, the outcome {\tt a0=1} is forbidden.

除了在这个附录中描述的内容,在\url{https://github.com/litmus-tests/litmus-tests-riscv}中还提供了一套超过七千个的石蕊测试。
% Beyond what is described in this appendix, a suite of more than seven thousand litmus tests is available at \url{https://github.com/litmus-tests/litmus-tests-riscv}.

\begin{commentary}
  litmus测试项目也提供了关于如何在RISC-V硬件上运行litmus测试,和如何将结果与操作和公理模型进行比较的指令。
  % The litmus tests repository also provides instructions on how to run
  % the litmus tests on RISC-V hardware and how to compare the results
  % with the operational and axiomatic models.
\end{commentary}

\begin{commentary}
  在未来,我们期望把这些关于内存模型的litmus测试也改编作为RISC-V一致性测试套件的一部分而使用。
  % In the future, we expect to adapt these memory model litmus tests for use as part of the RISC-V compliance test suite as well.
\end{commentary}

\section{RVWMO规则的解释}
在这节中,我们提供了对所有RVWMO规则和公理的解释和例子。
% In this section, we provide explanation and examples for all of the RVWMO rules and axioms.

\subsection{保留程序次序和全局内存次序}
保留程序次序代表了必须在全局内存次序中被遵循的程序次序的子集。
概念上,从其它硬件线程和/或观察者的角度,来自相同硬件线程的、按照保留程序次序被排序的事件,必须以该次序出现。
% Preserved program order represents the subset of program order that must be respected within the global memory order.
% Conceptually, events from the same hart that are ordered by preserved program order must appear in that order from the perspective of other harts and/or observers.
% Events from the same hart that are not ordered by preserved program order, on the other hand, may appear reordered from the perspective of other harts and/or observers.

% 换句话说,来自相同硬件线程的、没有按保留程序次序排序的事件,从其它硬件线程和/或观察者的角度,可以以新的次序出现。

非正式地讲,全局内存次序代表了加载和存储所执行的次序。
正式的内存模型文献已经从围绕执行概念构建的规范中移出,但是该思想对于建立非正式的直觉仍然是有用的。
对于加载,当它的返回值被确定时,它被称为已执行的。对于存储,不是当它在流水线内部被执行时、而是只有当它的值已经被传播到全局可见的存储时,它才被称为已执行的。
在这个意义上,全局内存次序也代表了一致性协议和/或余下的内存系统的贡献:把每个硬件线程发出的(可能被重新排序的)内存访问交错到所有硬件线程都赞成的单一的总次序之中。
% Informally, the global memory order represents the order in which loads and stores perform.
% The formal memory model literature has moved away from specifications built around the concept of performing, but the idea is still useful for building up informal intuition.
% A load is said to have performed when its return value is determined.
% A store is said to have performed not when it has executed inside the pipeline, but rather only when its value has been propagated to globally visible memory.
% In this sense, the global memory order also represents the contribution of the coherence protocol and/or the rest of the memory system to interleave the (possibly reordered) memory accesses being issued by each hart into a single total order agreed upon by all harts.

加载执行的次序并不总是直接对应于那两个加载所返回的值的相对生存时间。
特别地,对相同的地址,一个加载$b$可以在另一个加载$a$之前执行(例如,$b$可以在$a$之前执行,并且在全局内存次序中,$b$可以出现在$a$之前),但是尽管如此,$a$可以返回一个比$b$更早旧的值。
这种差异性(在其它事情之中)捕获了核心与内存之间安置的缓冲的重新排序效果。
例如,$b$可能已经返回了$a$存储在存储缓冲区中的一个值,同时$a$可能已经忽略了较新的存储,反而从内存中读取了一个较旧的值。
为了解释这个情况,在每次加载执行的时候,它返回的值由加载值公理决定,而不只是通过确定在全局内存次序中最近对相同地址的存储来严格地决定,正如下面描述的那样。
% The order in which loads perform does not always directly correspond to the relative age of the values those two loads return.
% In particular, a load $b$ may perform before another load $a$ to the same address (i.e., $b$ may execute before $a$, and $b$ may appear before $a$ in the global memory order), but $a$ may nevertheless return an older value than $b$.
% This discrepancy captures (among other things) the reordering effects of buffering placed between the core and memory.
% For example, $b$ may have returned a value from a store in the store buffer, while $a$ may have ignored that younger store and read an older value from memory instead.
% To account for this, at the time each load performs, the value it returns is determined by the load value axiom, not just strictly by determining the most recent store to the same address in the global memory order, as described below.

\subsection{\nameref*{rvwmo:ax:load}}
\label{sec:memory:loadvalueaxiom}
\begin{tabular}{p{1cm}|p{12cm}} &
\nameref{rvwmo:ax:load}: \loadvalueaxiom
\end{tabular}

保留程序次序不需要遵循“在重叠的地址上,一个存储跟随着一个加载”的次序。
这种复杂度的提升是因为,在几乎所有实现中存储缓冲区都是随处可见的。
非正式地说,当存储仍然在存储缓冲区中的时候,加载可以通过从存储转发来执行(返回一个值),并因此出现在了存储自身的执行(写回到全局可见内存)之前。
因此,任何其它的硬件线程将观察到,加载在存储之前执行。
% Preserved program order is {\em not} required to respect the ordering of a store followed by a load to an overlapping address.
% This complexity arises due to the ubiquity of store buffers in nearly all implementations.
% Informally, the load may perform (return a value) by forwarding from the store while the store is still in the store buffer, and hence before the store itself performs (writes back to globally visible memory).
% Any other hart will therefore observe the load as performing before the store.

\begin{figure}[h!]
  \centering
  \begin{tabular}{m{.4\linewidth}@{\qquad}m{.45\linewidth}}
  {
    \tt\small
    \begin{tabular}{cl||cl}
    \multicolumn{2}{c}{Hart 0} & \multicolumn{2}{c}{Hart 1} \\
    \hline
          & li t1, 1    &     & li t1, 1    \\
      (a) & sw t1,0(s0) & (e) & sw t1,0(s1) \\
      (b) & lw a0,0(s0) & (f) & lw a2,0(s1) \\
      (c) & fence r,r   & (g) & fence r,r   \\
      (d) & lw a1,0(s1) & (h) & lw a3,0(s0) \\
      \hline
      \multicolumn{4}{c}{输出结果 {\tt a0=1}, {\tt a1=0}, {\tt a2=1}, {\tt a3=0}}
    \end{tabular}
  }
  &
  \input{figs/litmus_sb_fwd.pdf_t}
  \end{tabular}
  \caption{一个存储缓冲区转发石蕊测试(允许的输出结果)
    % A store buffer forwarding litmus test (outcome permitted)
    }
  \label{fig:litmus:storebuffer}
\end{figure}

考虑表~\ref{fig:litmus:storebuffer}的litmus测试。
当在一个带有存储缓冲区(store buffers)的实现上运行这个程序时,它可能得到{\tt a0=1}, {\tt a1=0}, {\tt a2=1}, {\tt a3=0}的最终输出结果,如下:
% Consider the litmus test of Figure~\ref{fig:litmus:storebuffer}.
% When running this program on an implementation with store buffers, it is possible to arrive at the final outcome
% {\tt a0=1}, {\tt a1=0}, {\tt a2=1}, {\tt a3=0}
% as follows:
\begin{itemize}
  \item (a) 执行并进入第一个硬件线程的私有存储缓冲区(store buffer)  % executes and enters the first hart's private store buffer
  \item (b) 执行并从存储缓冲区中的(a)转发它的返回值1  % executes and forwards its return value 1 from (a) in the store buffer
  \item (c) 当所有之前的加载操作(例如,(b))都已经完成时执行  % executes since all previous loads (i.e., (b)) have completed
  \item (d) 执行并从内存读取值0  % executes and reads the value 0 from memory
  \item (e) 执行并进入第二个硬件线程的私有存储缓冲区  % executes and enters the second hart's private store buffer
  \item (f) 执行并从存储缓冲区中的(e)转发它的值1  % executes and forwards its return value 1 from (e) in the store buffer
  \item (g) 从所有之前的加载操作(例如,(f))都已经完成时执行  % executes since all previous loads (i.e., (f)) have completed
  \item (h) 执行并从内存读取值0  % executes and reads the value 0 from memory
  \item (a) 从第一个硬件线程的存储缓冲区排放到内存  % drains from the first hart's store buffer to memory
  \item (e) 从第二个硬件线程的存储缓冲区排放到内存  % drains from the second hart's store buffer to memory
\end{itemize}
因此,内存模型必须能够解释这种行为。
% Therefore, the memory model must be able to account for this behavior.

换句话说,假设保留程序次序确实包括了下列假定的规则:
在保留的程序次序中,内存访问$a$先于内存访问$b$(并因此也在全局内存次序中先于$b$),如果在程序次序中$a$先于$b$,并且$a$和$b$访问相同的内存位置,$a$是一个写,而$b$是一个读。把这个称作“规则X”。然后我们得到如下结果:
% To put it another way, suppose the definition of preserved program order did include the following hypothetical rule:
% memory access $a$ precedes memory access $b$ in preserved program order (and hence also in the global memory order) if $a$ precedes $b$ in program order and $a$ and $b$ are accesses to the same memory location, $a$ is a write, and $b$ is a read.  Call this ``Rule X''.  Then we get the following:

\begin{itemize}
  \item (a) 先于 (b): 根据规则 X
  \item (b) 先于 (d): 根据规则 \ref{ppo:fence}
  \item (d) 先于 (e): 根据加载值公理。  否则,如果(e)先于(d),那么将需要(d)返回值1。(这是一个完全合法的执行;它只是并非问题所在)% Otherwise, if (e) preceded (d), then (d) would be required to return the value 1.  (This is a perfectly legal execution; it's just not the one in question)
  \item (e) 先于 (f): 根据规则 X
  \item (f) 先于 (h): 根据规则 \ref{ppo:fence}
  \item (h) 先于 (a): 根据加载值公理,同上。
\end{itemize}
全局内存次序必须是一个总次序,而不能有循环,因为循环将暗示该循环内的每个事件都发生在它自己之前,这是不可能的。
因此,上面提出的执行将被禁止,并因此,规则X的添加将禁止带有存储缓冲区转发的实现,这显然是不可取的。
% The global memory order must be a total order and cannot be cyclic, because a cycle would imply that every event in the cycle happens before itself, which is impossible.
% Therefore, the execution proposed above would be forbidden, and hence the addition of rule X would forbid implementations with store buffer forwarding, which would clearly be undesirable.

尽管如此,即使在全局内存次序中,(b)先于(a)且/或(f)先于(e),这个例子中唯一合理的可能性也是,对于(b),返回由(a)所写的值,而(f)和(e)类似。
这种情况的组合导致了加载值公理的定义中的第二个选项。即使在全局内存次序中,(b)先于(a),由于在(b)执行的时候(a)还位于存储缓冲区中,(a)将仍然对(b)可见。
因此,即使在全局内存次序中(b)先于(a),(b)也应当返回由(a)所写的值,因为在程序次序中(a)先于(b)。对于(e)和(f)也类似。
% Nevertheless, even if (b) precedes (a) and/or (f) precedes (e) in the global memory order, the only sensible possibility in this example is for (b) to return the value written by (a), and likewise for (f) and (e).  
% This combination of circumstances is what leads to the second option in the definition of the load value axiom.
% Even though (b) precedes (a) in the global memory order, (a) will still be visible to (b) by virtue of sitting in the store buffer at the time (b) executes.
% Therefore, even if (b) precedes (a) in the global memory order, (b) should return the value written by (a) because (a) precedes (b) in program order.
% Likewise for (e) and (f).

\begin{figure}[h!]
  \centering
  \begin{tabular}{m{.4\linewidth}@{\qquad}m{.4\linewidth}}
  {
    \tt\small
    \begin{tabular}{cl||cl}
    \multicolumn{2}{c}{Hart 0} & \multicolumn{2}{c}{Hart 1} \\
    \hline
          & li t1, 1    &     & li t1, 1      \\
      (a) & sw t1,0(s0) &     & LOOP:         \\
      (b) & fence w,w   & (d) & lw a0,0(s1)   \\
      (c) & sw t1,0(s1) &     & beqz a0, LOOP \\
          &             & (e) & sw t1,0(s2)   \\
          &             & (f) & lw a1,0(s2)   \\
          &             &     & xor a2,a1,a1  \\
          &             &     & add s0,s0,a2  \\
          &             & (g) & lw a2,0(s0)   \\
      \hline
      \multicolumn{4}{c}{输出结果: {\tt a0=1}, {\tt a1=1}, {\tt a2=0}}
    \end{tabular}
  }
  &
  \input{figs/litmus_ppoca.pdf_t}
  \end{tabular}
  \caption{“PPOCA”存储缓冲区转发litmus测试(允许的输出结果)}
  \label{fig:litmus:ppoca}
\end{figure}

在图~\ref{fig:litmus:ppoca}中显示了另一个突出存储缓冲区行为的测试。
在这个例子中,由于控制依赖,(d)的次序排在(e)之前,而由于地址依赖,(f)的次序排在(g)之前。
然而,即使(f)返回了由(e)所写的值,(e)的次序也并不需要排在(f)之前。
这个可能对应到下列事件序列:
% Another test that highlights the behavior of store buffers is shown in Figure~\ref{fig:litmus:ppoca}.
% In this example, (d) is ordered before (e) because of the control dependency, and (f) is ordered before (g) because of the address dependency.
% However, (e) is {\em not} necessarily ordered before (f), even though (f) returns the value written by (e).
% This could correspond to the following sequence of events:
\begin{itemize}
  \item (e) 推测地执行,并进入第二个硬件线程的私有存储缓冲区(但是没有排放到内存) % executes speculatively and enters the second hart's private store buffer (but does not drain to memory)
  \item (f) 推测地执行,并从存储缓冲区中的(e)转发它的值1 % executes speculatively and forwards its return value 1 from (e) in the store buffer
  \item (g) 推测地执行,并从内存读取值0  % executes speculatively and reads the value 0 from memory
  \item (a) 执行,进入第一个硬件线程的私有存储缓冲区,并排放到内存  % executes, enters the first hart's private store buffer, and drains to memory
  \item (b) 执行,并退场  % executes and retires
  \item (c) 执行,进入第一个硬件线程的私有存储缓冲区,并排放到内存  % executes, enters the first hart's private store buffer, and drains to memory
  \item (d) 执行,并从内存读取值1  % executes and reads the value 1 from memory
  \item (e), (f), 和 (g) 提交,因为推测是正确的  % commit, since the speculation turned out to be correct
  \item (e) 从存储缓冲区排放到内存  % drains from the store buffer to memory
\end{itemize}

\subsection{\nameref*{rvwmo:ax:atom}}
\label{sec:memory:atomicityaxiom}
\begin{tabular}{p{1cm}|p{12cm}} &
\nameref{rvwmo:ax:atom} (对于对齐的原子): \atomicityaxiom
\end{tabular}

RISC-V架构把原子性的概念从排序的概念中解耦出来。
不像诸如TSO的架构,RISC-V在RVWMO下的原子性不会默认采用任何排序需求。
排序的语义仅仅由PPO规则保证,否则就是适用的。
% The RISC-V architecture decouples the notion of atomicity from the notion of ordering.  
% Unlike architectures such as TSO, RISC-V atomics under RVWMO do not impose any ordering requirements by default.  
% Ordering semantics are only guaranteed by the PPO rules that otherwise apply.

RISC-V包含两种类型的原子性:AMO和LR/SC对。通过下列方式,它们在概念上有不同的表现。
LR/SC的行为就像是,旧值被带到核心,修改,然后写回到内存,所有这些保留都维持在该内存位置。
另一方面,AMO在概念上表现得像是,它们直接在内存中执行。AMO因此有固有的原子性,而LR/SC对的原子性在某种意义上略有不同,
即在内存位置方面,在起初的硬件线程持有该保留的期间,不会被另一个硬件线程所修改。
% RISC-V contains two types of atomics: AMOs and LR/SC pairs.
% These conceptually behave differently, in the following way.
% LR/SC behave as if the old value is brought up to the core, modified, and written back to memory, all while a reservation is held on that memory location.
% AMOs on the other hand conceptually behave as if they are performed directly in memory.
% AMOs are therefore inherently atomic, while LR/SC pairs are atomic in the slightly different sense that the memory location in question will not be modified by another hart during the time the original hart holds the reservation.

\begin{figure}[h!]
  \centering\small
  \begin{verbbox}
  (a) lr.d a0, 0(s0)
  (b) sd   t1, 0(s0)
  (c) sc.d t2, 0(s0)
  \end{verbbox}
  \theverbbox
  ~~~~~~
  \begin{verbbox}
  (a) lr.d a0, 0(s0)
  (b) sw   t1, 4(s0)
  (c) sc.d t2, 0(s0)
  \end{verbbox}
  \theverbbox
  ~~~~~~
  \begin{verbbox}
  (a) lr.w a0, 0(s0)
  (b) sw   t1, 4(s0)
  (c) sc.w t2, 0(s0)
  \end{verbbox}
  \theverbbox
  ~~~~~~
  \begin{verbbox}
  (a) lr.w a0, 0(s0)
  (b) sw   t1, 4(s0)
  (c) sc.w t2, 8(s0)
  \end{verbbox}
  \theverbbox
  \caption{在所有的四个(独立的)代码片段中,存储条件(c)是被允许的,但是不保证成功
    % In all four (independent) code snippets, the store-conditional (c) is permitted but not guaranteed to succeed
    }
  \label{fig:litmus:lrsdsc}
\end{figure}

原子性公理禁止在全局内存次序中,来自其它硬件线程的存储,在一个LR核、与该LR配对的SC之间交错。
原子性公理没有禁止在程序次序或全局内存次序中,加载在成对的操作之间交错,也没有禁止在程序次序或全局内存次序中,
来自相同硬件线程的存储或者对非重叠位置的存储出现在成对的操作之间。例如,图~\ref{fig:litmus:lrsdsc}中的SC指令可以(但是不保证)成功。
那些成功没有一个将违背原子性公理,因为其间的非条件存储是与成对的加载-存储指令和存储-条件指令来自相同的硬件线程。
这样,一个在缓存行粒度追踪内存访问(并因此将看到图~\ref{fig:litmus:lrsdsc}中的四个片段是完全相同的)的内存系统将不会强制让碰巧(假)共享了相同缓存行另一部分作为保留所正在持有的内存位置的存储-条件指令失败。
% The atomicity axiom forbids stores from other harts from being interleaved in global memory order between an LR and the SC paired with that LR.
% The atomicity axiom does not forbid loads from being interleaved between the paired operations in program order or in the global memory order, nor does it forbid stores from the same hart or stores to non-overlapping locations from appearing between the paired operations in either program order or in the global memory order.
% For example, the SC instructions in Figure~\ref{fig:litmus:lrsdsc} may (but are not guaranteed to) succeed.
% None of those successes would violate the atomicity axiom, because the intervening non-conditional stores are from the same hart as the paired load-reserved and store-conditional instructions.
% This way, a memory system that tracks memory accesses at cache line granularity (and which therefore will see the four snippets of Figure~\ref{fig:litmus:lrsdsc} as identical) will not be forced to fail a store-conditional instruction that happens to (falsely) share another portion of the same cache line as the memory location being held by the reservation.

这个原子性公理也技术性地支持了LR和SC接触不同地址和/或使用不同访问尺寸的情形;然而,在实际中,预计这种行为的使用情形会很稀少。同样地,那种在一个LR/SC对之间,来自相同硬件线程的存储与该LR或SC引用的内存位置实际重叠的情景,与其间的存储可能简单地落在相同的缓存行上的情景相比,也是稀少的。
% The atomicity axiom also technically supports cases in which the LR and SC touch different addresses and/or use different access sizes; however, use cases for such behaviors are expected to be rare in practice.
% Likewise, scenarios in which stores from the same hart between an LR/SC pair actually overlap the memory location(s) referenced by the LR or SC are expected to be rare compared to scenarios where the intervening store may simply fall onto the same cache line.

\subsection{\nameref*{rvwmo:ax:prog}}
\label{sec:memory:progress}
\begin{tabular}{p{1cm}|p{12cm}} &
\nameref{rvwmo:ax:prog}: \progressaxiom
\end{tabular}

进程公理确保了一个最小的向前进程保证。它确保了来自一个硬件线程的存储将在有限数量的时间之内,最终变得对于系统中的其它硬件线程可见,
并且来自其它硬件线程的加载将最终能够读取那些值(或由该值而来的后继值)。没有这个规则的话,举个例子,
一个自旋锁无限地在一个值上旋转,将变得合法,甚至是在有来自另一个硬件线程的存储正在等待该自旋锁解锁的时候。
% The progress axiom ensures a minimal forward progress guarantee.
% It ensures that stores from one hart will eventually be made visible to other harts in the system in a finite amount of time, and that loads from other harts will eventually be able to read those values (or successors thereof).
% Without this rule, it would be legal, for example, for a spinlock to spin infinitely on a value, even with a store from another hart waiting to unlock the spinlock.

进程公理并不试图在一个RISC-V实现中的硬件线程上采用任何其它的公平、延迟或者服务质量的概念。任何更强的公平性概念都取决于剩余的ISA和/或平台和/或设备的定义和实现。
% The progress axiom is intended not to impose any other notion of fairness, latency, or quality of service onto the harts in a RISC-V implementation.
% Any stronger notions of fairness are up to the rest of the ISA and/or up to the platform and/or device to define and implement.

在几乎所有的情况中,向前进程公理都将被任何标准的缓存一致协议所满足。带有非一致性缓存的实现可能不得不提供一些其它的机制来确保所有的存储(或者由此而来的后继者)对于所有硬件线程的最终可见性。
% The forward progress axiom will in almost all cases be naturally satisfied by any standard cache coherence protocol.
% Implementations with non-coherent caches may have to provide some other mechanism to ensure the eventual visibility of all stores (or successors thereof) to all harts.

\subsection{重叠地址排序(规则~\ref{ppo:->st}-\ref{ppo:amoforward})
% Overlapping-Address Orderings (Rules~\ref{ppo:->st}--\ref{ppo:amoforward})
}
\label{sec:memory:overlap}
\begin{tabular}{p{1cm}|p{12cm}}
  & Rule \ref{ppo:->st}: \ppost \\
  & Rule \ref{ppo:rdw}: \ppordw \\
  & Rule \ref{ppo:amoforward}: \ppoamoforward \\
\end{tabular}

相同地址排序,如果后者是一个存储,那么是简单的:一个加载或存储永远不会被重新排序到一个与后面的存储重叠的内存位置。
从微架构的视角,总的来说,如果推测被证明是无效的,很难或者说不可能来撤销一个推测性重排的存储,因此这种行为被模型简单地禁止了。
换句话说,不需要从一个存储到后一个加载的相同地址排序。正如在~\ref{sec:memory:loadvalueaxiom}节中讨论的那样,这反映了将值从缓冲的存储转发到之后的加载的实现的可观察的行为。
% Same-address orderings where the latter is a store are straightforward: a load or store can never be reordered with a later store to an overlapping memory location.  From a microarchitecture perspective, generally speaking, it is difficult or impossible to undo a speculatively reordered store if the speculation turns out to be invalid, so such behavior is simply disallowed by the model.
% Same-address orderings from a store to a later load, on the other hand, do not need to be enforced.
% As discussed in Section~\ref{sec:memory:loadvalueaxiom}, this reflects the observable behavior of implementations that forward values from buffered stores to later loads.

相同地址的加载-加载排序的要求要微妙得多。基础要求是,较新的加载一定不能返回比同一个硬件线程中对相同地址进行的较旧的加载所返回的值更旧的值。
这通常被称为“CoRR”(读-读对的一致性),或者称为更宽泛的“一致性”或者“各位置的顺序连贯性”需求的一部分。
过去,一些架构已经放松了相同地址的加载-加载排序,但是事后看来,这通常会让编程模型变得过于复杂,并且因此RVWMO需要强制执行CoRR排序。
然而,因为全局内存次序对应于加载执行的次序,而不是值被返回的次序,所以,从全局内存次序的角度,捕获CoRR的需求需要一点间接性。
% Same-address load-load ordering requirements are far more subtle.
% The basic requirement is that a younger load must not return a value that is older than a value returned by an older load in the same hart to the same address.  This is often known as ``CoRR'' (Coherence for Read-Read pairs), or as part of a broader ``coherence'' or ``sequential consistency per location'' requirement.
% Some architectures in the past have relaxed same-address load-load ordering, but in hindsight this is generally considered to complicate the programming model too much, and so RVWMO requires CoRR ordering to be enforced.
% However, because the global memory order corresponds to the order in which loads perform rather than the ordering of the values being returned, capturing CoRR requirements in terms of the global memory order requires a bit of indirection.

\begin{figure}[h!]
  \center
  \begin{tabular}{m{.4\linewidth}@{\qquad}m{.4\linewidth}}
    {\tt\small
    \begin{tabular}{cl||cl}
    \multicolumn{2}{c}{Hart 0} & \multicolumn{2}{c}{Hart 1} \\
    \hline
          & li t1, 1    &     & li~ t2, 2    \\
      (a) & sw t1,0(s0) & (d) & lw~ a0,0(s1) \\
      (b) & fence w, w  & (e) & sw~ t2,0(s1) \\
      (c) & sw t1,0(s1) & (f) & lw~ a1,0(s1) \\
          &             & (g) & xor t3,a1,a1 \\
          &             & (h) & add s0,s0,t3 \\
          &             & (i) & lw~ a2,0(s0) \\
      \hline
      \multicolumn{4}{c}{输出结果: {\tt a0=1}, {\tt a1=2}, {\tt a2=0}}
    \end{tabular}
  }
  &
  \input{figs/litmus_mp_fenceww_fri_rfi_addr.pdf_t}
  \end{tabular}
  \caption{石蕊测试MP+fence.w.w+fir-rfi-addr(允许的输出结果)。 
  % Litmus test MP+fence.w.w+fri-rfi-addr (outcome permitted)
  }
  \label{fig:litmus:frirfi}
\end{figure}

考虑图~\ref{fig:litmus:frirfi}的石蕊测试,它是更一般的“fri-rfi”式样的一个特别的实例。术语“fri-rfi”表示序列(d)、(e)、(f):(d)“从读取”来自相同硬件线程的(e)(例如,从一个比(e)更早的写读取),而(f)从来自相同硬件线程的(e)读取。
% Consider the litmus test of Figure~\ref{fig:litmus:frirfi}, which is one particular instance of the more general ``fri-rfi'' pattern.
% The term ``fri-rfi'' refers to the sequence (d), (e), (f): (d) ``from-reads'' (i.e., reads from an earlier write than) (e) which is the same hart, and (f) reads from (e) which is in the same hart.

从微架构的视角,输出结果{\tt a0=1},{\tt a1=2},{\tt a2=0}是合法的(比起各种其它更加不怎么微妙的输出结果)。直观地讲,下列将产生上述提及的输出结果:
% From a microarchitectural perspective, outcome {\tt a0=1}, {\tt a1=2}, {\tt a2=0} is legal (as are various other less subtle outcomes).  Intuitively, the following would produce the outcome in question:
\begin{itemize}
  \item (d) 暂停(不论出于什么原因;或许它在等待一些其它先前的指令时就暂停了)
  \item (e) 执行,并进入存储缓冲区(但是还没有排放到内存)
  \item (f) 执行,并从存储缓冲区中的(e)转发
  \item (g)、(h)和(i)执行
  \item (a) 执行并排放到内存,(b)执行,且(c)执行并排放到内存
  \item (d) 解除暂停并执行
  \item (e) 从存储缓冲区排放到内存
\end{itemize}
这个对应于全局内存次序(f)、(i)、(a)、(c)、(d)、(e)。注意,即使(f)在(d)之前执行,由(f)返回的值也比由(d)返回的值更新。
因此,这个执行是合法的,并且不会违背CoRR需求。
% This corresponds to a global memory order of (f), (i), (a), (c), (d), (e).
% Note that even though (f) performs before (d), the value returned by (f) is newer than the value returned by (d).
% Therefore, this execution is legal and does not violate the CoRR requirements.

类似地,如果两个背靠背的加载返回了相同存储所写入的值,那么在全局内存次序中,它们也可以乱序出现而不违背CoRR。注意这与说两个加载返回相同的值是不相同的,因为两个不同的存储也可以写入相同的值。
% Likewise, if two back-to-back loads return the values written by the same store, then they may also appear out-of-order in the global memory order without violating CoRR.  Note that this is not the same as saying that the two loads return the same value, since two different stores may write the same value.

\begin{figure}[h!]
  \centering
  \begin{tabular}{m{.4\linewidth}@{\qquad\quad}m{.6\linewidth}}
  {
    \tt\small
    \begin{tabular}{cl||cl}
    \multicolumn{2}{c}{Hart 0} & \multicolumn{2}{c}{Hart 1} \\
    \hline
          & li t1, 1    & (d) & lw~ a0,0(s1) \\
      (a) & sw t1,0(s0) & (e) & xor t2,a0,a0 \\
      (b) & fence w, w  & (f) & add s4,s2,t2 \\
      (c) & sw t1,0(s1) & (g) & lw~ a1,0(s4) \\
          &             & (h) & lw~ a2,0(s2) \\
          &             & (i) & xor t3,a2,a2 \\
          &             & (j) & add s0,s0,t3 \\
          &             & (k) & lw~ a3,0(s0) \\
      \hline
      \multicolumn{4}{c}{输出结果: {\tt a0=1}, {\tt a1=$v$}, {\tt a2=$v$}, {\tt a3=0}}
    \end{tabular}
  }
  &
  \input{figs/litmus_rsw.pdf_t}
   \end{tabular}
  \caption{石蕊测试RSW(允许的输出结果) 
  % Litmus test RSW (outcome permitted)
  }
  \label{fig:litmus:rsw}
\end{figure}

考虑图~\ref{fig:litmus:rsw}的石蕊测试。
输出结果{\tt a0=1}、{\tt a1=$v$}、{\tt a2=$v$}、{\tt a3=0}(这里$v$是由另一个硬件线程所写入的某个值)
可以通过允许(g)和(h)重排而被观察到。这个做法可能是推测性的,并且该推测可以被微架构证明(例如,通过监视缓存失效情况而没有发现),
因为无论如何,在(g)之后重新执行(h)都将返回相同存储所写入的值。因此假设,无论如何a1和a2都将最终由相同的存储写入相同的值,(g)和(h)可以被合法地重新排序。
对应这个执行的全局内存次序将是(h)、(k)、(a)、(c)、(d)、(g)。
% Consider the litmus test of Figure~\ref{fig:litmus:rsw}.
% The outcome {\tt a0=1}, {\tt a1=$v$},  {\tt a2=$v$}, {\tt a3=0} (where $v$ is some value written by another hart) can be observed by allowing (g) and (h) to be reordered.  This might be done speculatively, and the speculation can be justified by the microarchitecture (e.g., by snooping for cache invalidations and finding none) because replaying (h) after (g) would return the value written by the same store anyway.
% Hence assuming {\tt a1} and {\tt a2} would end up with the same value written by the same store anyway, (g) and (h) can be legally reordered.
% The global memory order corresponding to this execution would be (h),(k),(a),(c),(d),(g).

图~\ref{fig:litmus:rsw}中{\tt a1}不等于{\tt a2}的测试的执行实际上需要(g)在全局内存次序中出现在(h)之前。允许(h)在全局内存次序中出现在(g)之前,那种情况中将导致违反CoRR,因为接下来(h)将返回一个比(g)所返回的更旧的值。
因此,PPO规则~\ref{ppo:rdw}禁止这种CoRR违背的发生。严格来说,PPO规则~\ref{ppo:rdw}在所有情况中执行CoRR都达成了一种小心的平衡,同时又足够弱,以允许在真实的微架构中经常出现的“RSW”和“fri-rfi”式样。
% Executions of the test in Figure~\ref{fig:litmus:rsw} in which {\tt a1} does not equal {\tt a2} do in fact require that (g) appears before (h) in the global memory order.
% Allowing (h) to appear before (g) in the global memory order would in that case result in a violation of CoRR, because then (h) would return an older value than that returned by (g).
% Therefore, PPO rule~\ref{ppo:rdw} forbids this CoRR violation from occurring.
% As such, PPO rule~\ref{ppo:rdw} strikes a careful balance between enforcing CoRR in all cases while simultaneously being weak enough to permit ``RSW'' and ``fri-rfi'' patterns that commonly appear in real microarchitectures.

还有一个重叠地址规则:PPO规则~\ref{ppo:amoforward}简单地陈述了,一个值不能从AMO或SC返回到后续的加载,直到AMO或SC已经全局执行(在SC的情况中,
还要是成功的全局执行)。从概念的观点,AMO和SC指令理应在内存中原子性地执行,这有点是顺理成章的。然而,特别地,PPO规则~\ref{ppo:amoforward}陈述了,
硬件可能甚至不会无意识地转发正在被AMOSWAP存储到后续加载的值,即使对于AMOSWAP,
该存储的值实际在语义上并不依赖于内存中的先前的值,就像对于其它AMO的情况一样。
即使当从SC转发存储的值在语义上不依赖于所配对的LR所返回的值时也同样如此。
% There is one more overlapping-address rule: PPO rule~\ref{ppo:amoforward} simply states that a value cannot be returned from an AMO or SC to a subsequent load until the AMO or SC has (in the case of the SC, successfully) performed globally.
% This follows somewhat naturally from the conceptual view that both AMOs and SC instructions are meant to be performed atomically in memory.
% However, notably, PPO rule~\ref{ppo:amoforward} states that hardware may not even non-speculatively forward the value being stored by an AMOSWAP to a subsequent load, even though for AMOSWAP that store value is not actually semantically dependent on the previous value in memory, as is the case for the other AMOs.
% The same holds true even when forwarding from SC store values that are not semantically dependent on the value returned by the paired LR.

上面这三个PPO规则也应用在当上述提到的内存访问只有部分重叠的情况中。
这是可能发生的,例如,当使用了不同尺寸的访问来访问相同的对象。也要注意,为了两个内存访问重叠,
两个重叠的内存操作的基地址不需要必定是相同的。当使用了未对齐的内存访问的时候,重叠地址PPO规则独立地应用到每个组件内存访问。
% The three PPO rules above also apply when the memory accesses in question only overlap partially.
% This can occur, for example, when accesses of different sizes are used to access the same object.
% Note also that the base addresses of two overlapping memory operations need not necessarily be the same for two memory accesses to overlap.
% When misaligned memory accesses are being used, the overlapping-address PPO rules apply to each of the component memory accesses independently.

% ----------  word文档初稿中没有这一段的翻译,可能是后期版本加上去的,需再三核查  ----------------
% \begin{comment}
% The formal model captures this as follows:
% \begin{itemize}
%   \item (a) precedes (b) in preserved program order because both are stores to the same address, and (b) is a store (Rule~\ref{ppo:->st}).  Therefore, (c) cannot return the value written by (a), because (b) is a later store to the same address in both program order and the global memory order, and so returning the value written by (a) would violate the load value axiom.
%   \item (c) precedes (d) in preserved program order because both are accesses to the same address, and (d) is a store.  (c) also precedes (d) in program order.  Therefore, (c) is not able to return the value written by (d), because neither option in the load value axiom applies.
% \end{itemize}
% \end{comment}

\subsection{屏障(规则~\ref{ppo:fence})}
\label{sec:mm:fence}
\begin{tabular}{p{1cm}|p{12cm}} &
Rule \ref{ppo:fence}: \ppofence
\end{tabular}

默认情况下,FENCE指令确保程序次序中所有的来自屏障之前的指令的内存访问(即,“前趋集”)在全局内存次序中,早于程序次序中来自屏障之后的指令的内存访问(即,“后继集”)出现。
然而,屏障可以选择性地把前趋集和/或后继集进一步限制到一个更小的内存访问集合,以提供某些加速。
特别地,屏障拥有限制前趋集和/或后继集的PR、PW、SR和SW位。当且仅当设置了PR位(对应于PW)时,前趋集包括加载(对应于存储)。
类似地,当且仅当设置了SR(对应于SW)时,后继集包括加载(对应于存储)。
% By default, the FENCE instruction ensures that all memory accesses from instructions preceding the fence in program order (the ``predecessor set'') appear earlier in the global memory order than memory accesses from instructions appearing after the fence in program order (the ``successor set'').
% However, fences can optionally further restrict the predecessor set and/or the successor set to  a smaller set of memory accesses in order to provide some speedup.
% Specifically, fences have PR, PW, SR, and SW bits which restrict the predecessor and/or successor sets.
% The predecessor set includes loads (resp.\@ stores) if and only if PR (resp.\@ PW) is set.
% Similarly, the successor set includes loads (resp.\@ stores) if and only if SR (resp.\@ SW) is set.

当前的FENCE编码有关于四个位PR、PW、SR和SW的九个非平凡的组合,加上一个额外的编码FENCE.TSO,它有助于“获取+释放”或RVTSO语义的映射。
剩余的七个组合没有前趋集和/或后继集,并因此都是no-op。对于十个非平凡的选项,只有六个是在实际中经常使用的:
% The FENCE encoding currently has nine non-trivial combinations of the four bits PR, PW, SR, and SW, plus one extra encoding FENCE.TSO which facilitates mapping of ``acquire+release'' or RVTSO semantics.
% The remaining seven combinations have empty predecessor and/or successor sets and hence are no-ops.
% Of the ten non-trivial options, only six are commonly used in practice:
{
\begin{itemize}
  \item FENCE RW,RW
  \item FENCE.TSO
  \item FENCE RW,W
  \item FENCE R,RW
  \item FENCE R,R
  \item FENCE W,W
\end{itemize}
}
使用PR、PW、SR和SW的任何其它组合的FENCE指令是被保留的。我们强烈建议编程人员坚持使用这六个。其它的组合可能与内存模型有未知的或者不期望的交互。
% FENCE instructions using any other combination of PR, PW, SR, and SW are reserved.  We strongly recommend that programmers stick to these six.
% Other combinations may have unknown or unexpected interactions with the memory model.

最后,我们注意到,由于RISC-V使用一种多重拷贝原子的内存模型,编程人员因此可以以一种线程本地的方式来推断屏障位。在非多重拷贝原子的内存模型中,没有“屏障累积性”的复杂性概念。
% Finally, we note that since RISC-V uses a multi-copy atomic memory model, programmers can reason about fences bits in a thread-local manner.  There is no complex notion of ``fence cumulativity'' as found in memory models that are not multi-copy atomic.

\subsection{显式同步(规则~\ref{ppo:acquire}-\ref{ppo:pair})}
% \subsection{Explicit Synchronization (Rules~\ref{ppo:acquire}--\ref{ppo:pair})}
\label{sec:memory:acqrel}
\begin{tabular}{p{1cm}|p{12cm}}
  & 规则 \ref{ppo:acquire}: \ppoacquire \\
  & 规则 \ref{ppo:release}: \pporelease \\
  & 规则 \ref{ppo:rcsc}: \pporcsc \\
  & 规则 \ref{ppo:pair}: \ppopair \\
\end{tabular}

一个{\em 获取}操作,正如应当被用在临界区开始处那样,需要在程序次序中的所有之后的内存操作也都在全局内存次序中在获取操作之后。这确保了,例如,临界区之内的所有的加载和存储,相对于正在保护它们的同步变量,都是最新的。
获取次序可以通过两种方式之一而采用:通过一个acquire注释,它采用只相对于同步变量自身的次序,或者通过一个FENCE R, RW,它采用相对于所有先前的加载的次序。
% An {\em acquire} operation, as would be used at the start of a critical section, requires all memory operations following the acquire in program order to also follow the acquire in the global memory order.
% This ensures, for example, that all loads and stores inside the critical section are up to date with respect to the synchronization variable being used to protect it.
% Acquire ordering can be enforced in one of two ways: with an acquire annotation, which enforces ordering with respect to just the synchronization variable itself, or with a FENCE~R,RW, which enforces ordering with respect to all previous loads.

\begin{figure}[h!]
  \centering\small
  \begin{verbatim}
          sd           x1, (a1)     # 任意不相关的存储 
          ld           x2, (a2)     # 任意不相关的加载  
          li           t0, 1        # 初始化交换值 
      again:
          amoswap.w.aq t0, t0, (a0) # 尝试获取锁 
          bnez         t0, again    # 如果被占用则重试 
          # ...
          # Critical section.
          # ...
          amoswap.w.rl x0, x0, (a0) # 通过存入0来释放锁 
          sd           x3, (a3)     # 任意不相关的存储 
          ld           x4, (a4)     # 任意不相关的加载 
  \end{verbatim}
  \caption{带原子性的自旋锁}
  \label{fig:litmus:spinlock_atomics}
\end{figure}

考虑图~\ref{fig:litmus:spinlock_atomics}。
因为这个例子使用{\em aq},临界区中的加载和存储被保证在全局内存次序中出现在用于获取锁的AMOSWAP之后。然而,假设{\tt a0}、{\tt a1}和{\tt a2}指向不同的内存位置,临界区中的加载和存储可能会、或可能不会在全局内存次序中,出现在例子开始的“任意不相干的加载”之后。
% Consider Figure~\ref{fig:litmus:spinlock_atomics}.
% Because this example uses {\em aq}, the loads and stores in the critical section are guaranteed to appear in the global memory order after the AMOSWAP used to acquire the lock.  However, assuming {\tt a0}, {\tt a1}, and {\tt a2} point to different memory locations, the loads and stores in the critical section may or may not appear after the ``Arbitrary unrelated load'' at the beginning of the example in the global memory order.

\begin{figure}[h!]
  \centering\small
  \begin{verbatim}
          sd           x1, (a1)     # 任意不相关的存储 
          ld           x2, (a2)     # 任意不相关的加载 
          li           t0, 1        # 初始化交换值 
      again:
          amoswap.w    t0, t0, (a0) # 尝试获取锁
          fence        r, rw        # 强制采用“acquire”内存次序
          bnez         t0, again    # 如果被占用则重试
          # ...
          # Critical section.
          # ...
          fence        rw, w        # 强制采用“release”内存次序
          amoswap.w    x0, x0, (a0) # 通过存入0来释放锁 
          sd           x3, (a3)     # 任意不相关的存储 
          ld           x4, (a4)     # 任意不相关的加载
  \end{verbatim}
  \caption{带屏障的自旋锁}
  \label{fig:litmus:spinlock_fences}
\end{figure}

现在,考虑图~\ref{fig:litmus:spinlock_fences}中的替代方案。
在这种情况中,即使AMOSWAP不会采用带有{\em aq}位的次序,尽管如此,屏障也会使获取AMOSWAP在全局内存次序中早于临界区中的所有加载和存储出现。
然而,注意,在这种情况中,屏障也会强制采用额外的次序:它也需要程序开始处的“任意不相干的加载”在全局内存次序中比临界区的加载和存储出现得更早。(然而,这个特别的屏障并不强制采用任何相对于片段开始处的“任意不相干的存储”的次序。)
通过这种方式,屏障强加的次序比通过{\em .aq}采用的次序会稍微地更粗糙些。
% Now, consider the alternative in Figure~\ref{fig:litmus:spinlock_fences}.
% In this case, even though the AMOSWAP does not enforce ordering with an {\em aq} bit, the fence nevertheless enforces that the acquire AMOSWAP appears earlier in the global memory order than all loads and stores in the critical section.
% Note, however, that in this case, the fence also enforces additional orderings: it also requires that the ``Arbitrary unrelated load'' at the start of the program appears earlier in the global memory order than the loads and stores of the critical section.  (This particular fence does not, however, enforce any ordering with respect to the ``Arbitrary unrelated store'' at the start of the snippet.)
% In this way, fence-enforced orderings are slightly coarser than orderings enforced by {\em .aq}.

释放次序和获取次序的效果完全相同,只是方向相反。释放的语义需要在程序次序中所有的先于释放操作的加载和存储也要在全局内存次序中先于释放操作。
这确保了,例如,在临界区中的内存访问在全局内存次序中出现在锁释放存储之前。
正如和获取的语义一样,释放的语义可以使用relase注释或者用FENCE RW, W操作来强制采用。使用相同的例子,临界区中的加载和存储和代码片段末尾处的“任意不相干存储”之间的次序只由图~\ref{fig:litmus:spinlock_fences}的FENCE RW,W采用,而不是图~\ref{fig:litmus:spinlock_atomics}中的{\em rl}。
% Release orderings work exactly the same as acquire orderings, just in the opposite direction.  Release semantics require all loads and stores preceding the release operation in program order to also precede the release operation in the global memory order.
% This ensures, for example, that memory accesses in a critical section appear before the lock-releasing store in the global memory order.  Just as for acquire semantics, release semantics can be enforced using release annotations or with a FENCE~RW,W operation.  Using the same examples, the ordering between the loads and stores in the critical section and the ``Arbitrary unrelated store'' at the end of the code snippet is enforced only by the FENCE~RW,W in Figure~\ref{fig:litmus:spinlock_fences}, not by the {\em rl} in Figure~\ref{fig:litmus:spinlock_atomics}.

单独使用RCpc注释,存储-释放到加载-获取的次序是不会被强制采用的。这有助于在TSO和/或RCpc内存模型下所写的代码的移植。
为了强制采用存储-释放到加载-获取的次序,代码必须使用store-release-RCsc和load-acquire-RCsc操作,以便应用PPO规则\ref{ppo:rcsc}。
对于许多C/C++中的使用情形,只有RCpc来举一些例子是足够的,但是对于许多C/C++、Java和Linux中的其它的使用情形是不够的;详情请见~\ref{sec:memory:porting}节。
% With RCpc annotations alone, store-release-to-load-acquire ordering is not enforced.  This facilitates the porting of code written under the TSO and/or RCpc memory models.  
% To enforce store-release-to-load-acquire ordering, the code must use store-release-RCsc and load-acquire-RCsc operations so that PPO rule \ref{ppo:rcsc} applies.
% RCpc alone is sufficient for many use cases in C/C++ but is insufficient for many other use cases in C/C++, Java, and Linux, to name just a few examples; see Section~\ref{sec:memory:porting} for details.

PPO规则~\ref{ppo:pair}说明了,一个SC必须在全局内存次序中出现在它所配对的LR之后。由于固有的数据依赖,这将自然地从LR/SC的常见使用开始去执行一个原子的读-修改-写操作。
然而,PPO规则~\ref{ppo:pair}也会应用,即使当正在存储的值在句法上并不依赖于所配对的LR所返回的值。
% PPO rule~\ref{ppo:pair} indicates that an SC must appear after its paired LR in the global memory order.
% This will follow naturally from the common use of LR/SC to perform an atomic read-modify-write operation due to the inherent data dependency.
% However, PPO rule~\ref{ppo:pair} also applies even when the value being stored does not syntactically depend on the value returned by the paired LR.

最后,我们注意到,只使用屏障,编程人员在分析排序注释的时候,不需要担心“累积性”。
% Lastly, we note that just as with fences, programmers need not worry about ``cumulativity'' when analyzing ordering annotations.

\subsection{句法依赖(规则~\ref{ppo:addr}-\ref{ppo:ctrl})}
\label{sec:memory:dependencies}
\begin{tabular}{p{1cm}|p{12cm}}
  & 规则 \ref{ppo:addr}: \ppoaddr \\
  & 规则 \ref{ppo:data}: \ppodata \\
  & 规则 \ref{ppo:ctrl}: \ppoctrl \\
\end{tabular}

从一个加载到相同硬件线程中的后续内存操作的依赖是RVWMO内存模型所考虑的。Alpha内存模型由于选择{\em 不}强制采用这些依赖而著名,但是大多数现代硬件和软件内存模型都考虑允许依赖指令被重新排序,是过于混乱和违反直觉的。此外,现代代码有时会故意使用这种依赖,作为一种特别轻量级的排序实施机制。
% Dependencies from a load to a later memory operation in the same hart are respected by the RVWMO memory model.
% The Alpha memory model was notable for choosing {\em not} to enforce the ordering of such dependencies, but most modern hardware and software memory models consider allowing dependent instructions to be reordered too confusing and counterintuitive.
% Furthermore, modern code sometimes intentionally uses such dependencies as a particularly lightweight ordering enforcement mechanism.

第~\ref{sec:memorymodel:dependencies}节中的术语工作如下。
无论何时,当写入每个目的寄存器的值是源寄存器的函数的时候,指令被称为携带了从它们的源寄存器到它们的目的寄存器的依赖。
对于大多数指令,这意味着,目的寄存器携带了一个来自所有源寄存器的依赖。然而,也有一些著名的例外。在内存指令的情形中,写入目的寄存器的值最终来自于内存系统,而不是直接来自源寄存器,并因此这样打破了所携带的来自源寄存器的依赖链。
在无条件跳转的情形中,写入目的寄存器的值来自于当前的{\tt pc}(它永远不会被内存模型认为是一个源寄存器),并因此类似地,JALR(仅有的带有源寄存器的跳转)不会携带一个从{\em rs1}到{\em rd}的依赖。
% The terms in Section~\ref{sec:memorymodel:dependencies} work as follows.
% Instructions are said to carry dependencies from their source register(s) to their destination register(s) whenever the value written into each destination register is a function of the source register(s).
% For most instructions, this means that the destination register(s) carry a dependency from all source register(s).
% However, there are a few notable exceptions.
% In the case of memory instructions, the value written into the destination register ultimately comes from the memory system rather than from the source register(s) directly, and so this breaks the chain of dependencies carried from the source register(s).
% In the case of unconditional jumps, the value written into the destination register comes from the current {\tt pc} (which is never considered a source register by the memory model), and so likewise, JALR (the only jump with a source register) does not carry a dependency from {\em rs1} to {\em rd}.

\begin{verbbox}
(a) fadd  f3,f1,f2
(b) fadd  f6,f4,f5
(c) csrrs a0,fflags,x0
\end{verbbox}
\begin{figure}[h!]
  \centering\small
  \theverbbox
  \caption{通过{\tt fflags},一个(a)和(b)都会隐含地累积进去的目的寄存器,(c)有一个关于(a)和(b)的句法依赖。
    % (c) has a syntactic dependency on both (a) and (b) via {\tt fflags}, a destination register that both (a) and (b) implicitly accumulate into
    }
  \label{fig:litmus:fflags}
\end{figure}

累积到一个目的寄存器,而不是写入它,这个概念反应了类似{\tt fflags}的CSR的行为。特别地,累积进一个寄存器不会冲击到任何先前的写入或对相同寄存器的累积。
例如,在图~\ref{fig:litmus:fflags}中,(c)有一个同时关于(a)和(b)的句法依赖。
% The notion of accumulating into a destination register rather than writing into it reflects the behavior of CSRs such as {\tt fflags}.
% In particular, an accumulation into a register does not clobber any previous writes or accumulations into the same register.
% For example, in Figure~\ref{fig:litmus:fflags}, (c) has a syntactic dependency on both (a) and (b).

类似其它现代内存模型,RVWMO内存模型使用句法依赖,而不是语义依赖。换句话说,这个定义依赖于正在被不同指令访问的寄存器的标识,
而不是那些寄存器的实际的内容。这意味着地址依赖、控制依赖、或者内存依赖必须是强制采用的,即使计算看起来是可以“优化掉”的。
这个选择确保了RVWMO保留了与使用这些假句法依赖作为轻量级排序机制的代码的兼容性。
% Like other modern memory models, the RVWMO memory model uses syntactic rather than semantic dependencies.
% In other words, this definition depends on the identities of the
% registers being accessed by different instructions, not the actual
% contents of those registers.  This means that an address, control, or
% data dependency must be enforced even if the calculation could seemingly
% be ``optimized away''.
% This choice ensures that RVWMO remains compatible with code that uses these false syntactic dependencies as a lightweight ordering mechanism.

\begin{verbbox}
ld  a1,0(s0)
xor a2,a1,a1
add s1,s1,a2
ld  a5,0(s1)
\end{verbbox}
\begin{figure}[h!]
  \centering\small
  \theverbbox
  \caption{一个句法地址依赖  
  % A syntactic address dependency
  }
  \label{fig:litmus:address}
\end{figure}

例如,在图~\ref{fig:litmus:address}中,存在一个从第一个指令生成的内存操作到最后一个指令生成的内存操作的句法地址依赖,即使{\tt a1} XOR {\tt a1}是零,
并因此不会影响到第二个加载所访问的地址。
% For example, there is a syntactic address
% dependency from the memory operation generated by the first instruction to the memory operation generated by the last instruction in
% Figure~\ref{fig:litmus:address}, even though {\tt a1} XOR {\tt a1} is zero and
% hence has no effect on the address accessed by the second load.

使用依赖作为轻量级同步机制的好处是,排序的强制性需求仅仅被限制在上述提及的特定的两个指令。
其它非依赖的指令可以被激进的实现自由地重排。一个替代方案是使用一个加载-获取,但是这将强制第一个加载相对于所有后继指令的次序。
另一个替代方案是使用FENCE R, R,但是这将包含所有先前的加载和{\em 所有}后继的加载,使这个选择更加昂贵。
% The benefit of using dependencies as a lightweight synchronization mechanism is that the ordering enforcement requirement is limited only to the specific two instructions in question.
% Other non-dependent instructions may be freely reordered by aggressive implementations.
% One alternative would be to use a load-acquire, but this would enforce ordering for the first load with respect to {\em all} subsequent instructions.
% Another would be to use a FENCE~R,R, but this would include all previous and all subsequent loads, making this option more expensive.

\begin{verbbox}
      lw  x1,0(x2)
      bne x1,x0,next
      sw  x3,0(x4)
next: sw  x5,0(x6)
\end{verbbox}
\begin{figure}[h!]
  \centering\small
  \theverbbox
  \caption{一个句法控制依赖 
  % A syntactic control dependency
  }
  \label{fig:litmus:control1}
\end{figure}

一个控制依赖总是扩展到程序次序中跟在原始目标之后的所有的指令,在这个意义上,控制依赖的表现不同于地址依赖和数据依赖。
考虑表~\ref{fig:litmus:control1}:尽管{\tt 下}一个指令将总是执行,但是由上一个指令生成的内存操作仍然有一个来自第一个指令生成的内存操作的控制依赖。
% Control dependencies behave differently from address and data dependencies in the sense that a control dependency always extends to all instructions following the original target in program order.
% Consider Figure~\ref{fig:litmus:control1}: the instruction at {\tt next} will always execute, but the memory operation generated by that last instruction nevertheless still has a control dependency from the memory operation generated by the first instruction.

\begin{verbbox}
        lw  x1,0(x2)
        bne x1,x0,next
  next: sw  x3,0(x4)
\end{verbbox}
\begin{figure}[h!]
  \centering\small
  \theverbbox
  \caption{另一个句法控制依赖  
  % Another syntactic control dependency
  }
  \label{fig:litmus:control2}
\end{figure}

类似地,考虑图~\ref{fig:litmus:control2}。
即使两个分支的最终结果都有相同的目标,仍然存在从这个片段的第一个指令生成的内存操作,到最后一个指令生成的内存操作的一个控制依赖。控制依赖的这个定义比其它环境中(例如,C++)可能看到的要强一些,但是它符合文献中控制依赖的标准定义。
% Likewise, consider Figure~\ref{fig:litmus:control2}.
% Even though both branch outcomes have the same target, there is still a control dependency from the memory operation generated by the first instruction in this snippet to the memory operation generated by the last instruction.
% This definition of control dependency is subtly stronger than what might be seen in other contexts (e.g., C++), but it conforms with standard definitions of control dependencies in the literature.

显然,PPO规则\ref{ppo:addr} - \ref{ppo:ctrl}也是有意设计的,以尊重来源于成功的存储条件指令的输出的依赖。
通常,一个SC指令将跟随一个检测输出结果是否成功的条件分支;这暗示了将会有一个从SC指令生成的存储操作到分支随后的任何内存操作的控制依赖。
PPO规则~\ref{ppo:ctrl}反过来暗示了,任何后继的存储操作将在全局内存次序中比SC生成的存储操作出现得更晚。
然而,由于控制依赖、地址依赖和数据依赖是定义在内存操作上的,并且由于一次不成功的SC不会生成内存操作,所以在不成功的SC和它的依赖指令之间不会强制排序。并且,由于只有当SC成功的时候,SC才被定义为携带从它的源寄存器到{\em rd}的依赖,一次不成功的SC不会影响全局内存次序。
% Notably, PPO rules \ref{ppo:addr}--\ref{ppo:ctrl} are also intentionally designed to respect dependencies that originate from the output of a successful store-conditional instruction.
% Typically, an SC instruction will be followed by a conditional branch checking whether the outcome was successful; this implies that there will be a control dependency from the store operation generated by the SC instruction to any memory operations following the branch.
% PPO rule~\ref{ppo:ctrl} in turn implies that any subsequent store operations will appear later in the global memory order than the store operation generated by the SC.
% However, since control, address, and data dependencies are defined over memory operations, and since an unsuccessful SC does not generate a memory operation, no order is enforced between unsuccessful SC and its dependent instructions.
% Moreover, since SC is defined to carry dependencies from its source registers to {\em rd} only when the SC is successful, an unsuccessful SC has no effect on the global memory order.


\begin{figure}[h!]
  \centering
  \begin{tabular}{m{.4\linewidth}m{0.05\linewidth}m{.4\linewidth}}
  {
    \tt\small
    \begin{tabular}{cl||cl}
    \multicolumn{4}{c}{初始值: 0(s0)=1; 0(s2)=1} \\
    \\
    \multicolumn{2}{c}{Hart 0} & \multicolumn{2}{c}{Hart 1} \\
    \hline
      (a) & ld a0,0(s0)    & (e) & ld a3,0(s2) \\
      (b) & lr a1,0(s1)    & (f) & sd a3,0(s0) \\
      (c) & sc a2,a0,0(s1) &                    \\
      (d) & sd a2,0(s2)    &                    \\
      \hline
      \multicolumn{4}{c}{输出结果: {\tt a0=0}, {\tt a3=0}}
    \end{tabular}
  }
  & &
  \input{figs/litmus_lb_lrsc.pdf_t}
  \end{tabular}
  \caption{LB石蕊测试的一种变体(禁止的输出结果)  
  % A variant of the LB litmus test (outcome forbidden)
  }
  \label{fig:litmus:successdeps}
\end{figure}

此外,选择尊重源自于存储-条件指令的依赖确保了,特定的类似无中生有的行为会被阻止。
考虑图~\ref{fig:litmus:successdeps}。假设一个假想的实现可以偶然地做到提前保证存储-条件操作将会成功。
在这种情形中,(c)将提前返回0到{\tt a2}(在实际执行之前),从而允许序列(d)、(e)、(f)、(a)、然后是(b)的执行,接着(c)可能只在那一点(成功地)执行。这将表示(c)把它自己的成功的值写到了{\tt 0(s1)}!幸运的是,由于RVWMO尊重源自于由成功的SC指令生成的存储的依赖的事实,这个情形和其它类似的情形被阻止了。
% In addition, the choice to respect dependencies originating at store-conditional instructions ensures that certain out-of-thin-air-like behaviors will be prevented.
% Consider Figure~\ref{fig:litmus:successdeps}.
% Suppose a hypothetical implementation could occasionally make some early guarantee that a store-conditional operation will succeed.
% In this case, (c) could return 0 to {\tt a2} early (before actually executing), allowing the sequence (d), (e), (f), (a), and then (b) to execute, and then (c) might execute (successfully) only at that point.
% This would imply that (c) writes its own success value to {\tt 0(s1)}!
% Fortunately, this situation and others like it are prevented by the fact that RVWMO respects dependencies originating at the stores generated by successful SC instructions.

我们也注意到,指令之间的句法依赖,只有当它们采取句法地址依赖、句法控制依赖,和/或句法数据依赖的形式时才会有效力。
例如,~\ref{sec:source-dest-regs}节中,在两个“F”指令之间通过“累积CSR”形成的句法依赖并{\em 不}表示这两个“F”指令必须按次序执行。这种依赖将只会用于之后最终建立从两个“F”指令到之后的上述提及的访问CSR标志的CSR指令的依赖。
% We also note that syntactic dependencies between instructions only have any force when they take the form of a syntactic address, control, and/or data dependency.
% For example: a syntactic dependency between two ``F'' instructions via one of the ``accumulating CSRs'' in Section~\ref{sec:source-dest-regs} does {\em not} imply that the two ``F'' instructions must be executed in order.
% Such a dependency would only serve to ultimately set up later a dependency from both ``F'' instructions to a later CSR instruction accessing the CSR flag in question.

\subsection{流水线依赖(规则~\ref{ppo:addrdatarfi}-\ref{ppo:addrpo})}
% \subsection{Pipeline Dependencies (Rules~\ref{ppo:addrdatarfi}--\ref{ppo:addrpo})}
\label{sec:memory:ppopipeline}
\begin{tabular}{p{1cm}|p{12cm}}
  & 规则 \ref{ppo:addrdatarfi}: \ppoaddrdatarfi \\
  & 规则 \ref{ppo:addrpo}: \ppoaddrpo \\
%  & Rule \ref{ppo:ctrlcfence}: \ppoctrlcfence \\
%  & Rule \ref{ppo:addrpocfence}: \ppoaddrpocfence \\
\end{tabular}

\begin{figure}[h!]
  \centering
  \begin{tabular}{m{.4\linewidth}m{.05\linewidth}m{.4\linewidth}}
  {
    \tt\small
    \begin{tabular}{cl||cl}
    \multicolumn{2}{c}{Hart 0} & \multicolumn{2}{c}{Hart 1} \\
    \hline
          & li t1, 1    & (d) & lw a0, 0(s1)   \\
      (a) & sw t1,0(s0) & (e) & sw a0, 0(s2)   \\
      (b) & fence w, w  & (f) & lw a1, 0(s2)   \\
      (c) & sw t1,0(s1) &     & xor a2,a1,a1   \\
          &             &     & add s0,s0,a2   \\
          &             & (g) & lw a3,0(s0)    \\   
      \hline
      \multicolumn{4}{c}{输出结果: {\tt a0=1}, {\tt a3=0}}
    \end{tabular}
  } & &
  \input{figs/litmus_datarfi.pdf_t}
  \end{tabular}

  \caption{根据PPO规则~\ref{ppo:addrdatarfi}和从(d)到(e)的数据依赖,在全局内存次序中,(d)必须也先于(f)(禁止的输出结果)
    % Because of PPO rule~\ref{ppo:addrdatarfi} and the data dependency from (d) to (e), (d) must also precede (f) in the global memory order (outcome forbidden)
    }
  \label{fig:litmus:addrdatarfi}
\end{figure}

PPO规则~\ref{ppo:addrdatarfi}和\ref{ppo:addrpo}反应了几乎所有的真实的处理器流水线实现的行为。规则~\ref{ppo:addrdatarfi}陈述了一个加载不能从一个存储转发,
直到那个存储的地址和数据是已知的时候。考虑图~\ref{fig:litmus:addrdatarfi}:在(e)的数据被决定之前,(f)不能执行,因为(f)必须返回由(e)写入的值(或者在全局内存次序中由某些更加靠后的所写入的值),
而在(d)有机会执行之前,旧的值必须不能被(e)的写回所冲击。因此,(f)将永远不会在(d)的执行之前执行。
% PPO rules~\ref{ppo:addrdatarfi} and \ref{ppo:addrpo} reflect behaviors of almost all real processor pipeline implementations.
% Rule~\ref{ppo:addrdatarfi} states that a load cannot forward from a store until the address and data for that store are known.
% Consider Figure~\ref{fig:litmus:addrdatarfi}:
% (f) cannot be executed until the data for (e) has been resolved, because (f) must return the value written by (e) (or by something even later in the global memory order), and the old value must not be clobbered by the writeback of (e) before (d) has had a chance to perform.
% Therefore, (f) will never perform before (d) has performed.

\begin{figure}[h!]
  \centering
  \begin{tabular}{m{.4\linewidth}m{.05\linewidth}m{.4\linewidth}}
  {
    \tt\small
    \begin{tabular}{cl||cl}
    \multicolumn{2}{c}{Hart 0} & \multicolumn{2}{c}{Hart 1} \\
    \hline
          & li t1, 1    &     & li t1, 1       \\
      (a) & sw t1,0(s0) & (d) & lw a0, 0(s1)   \\
      (b) & fence w, w  & (e) & sw a0, 0(s2)   \\
      (c) & sw t1,0(s1) & (f) & sw t1, 0(s2)   \\
          &             & (g) & lw a1, 0(s2)   \\
          &             &     & xor a2,a1,a1   \\
          &             &     & add s0,s0,a2   \\
          &             & (h) & lw a3,0(s0)    \\   

      \hline
      \multicolumn{4}{c}{输出结果: {\tt a0=1}, {\tt a3=0}}
    \end{tabular}
  } & &
  \input{figs/litmus_datacoirfi.pdf_t}
  \end{tabular}

  \caption{根据PPO规则12和从(d)到(e)的数据依赖,在全局内存次序中,(d)必须也先于(f)(禁止的输出结果)
    % Because of the extra store between (e) and (g), (d) no longer necessarily precedes (g) (outcome permitted)
    }
  \label{fig:litmus:addrdatarfi_no}
\end{figure}

如果在(e)和(f)之间,有另一个针对相同地址的存储,就像图~\ref{fig:litmus:addrdatarfi_no}中的那样,那么(f)将不再依赖于(e)正在被决定的数据,并因此(f)关于为(e)生产数据的(d)的依赖将被打破。
% If there were another store to the same address in between (e) and (f), as in Figure~\ref{fig:litmus:addrdatarfi_no}, then (f) would no longer be dependent on the data of (e) being resolved, and hence the dependency of (f) on (d), which produces the data for (e), would be broken.

规则~\ref{ppo:addrpo}制定了一个和之前规则相似的观点:一个存储不能在内存执行,直到所有的先前可能访问相同地址的加载自身已经被执行。
这样的加载看起来必须在存储之前执行,但是如果存储在加载有机会读取旧值之前就覆写了内存中的值,它就不能这么做了。类似地,一个存储通常不能执行,直到它已知了先前的指令不会由于地址解析失败而引发异常,而从这个意义上讲,规则~\ref{ppo:addrpo}可以被视作规则~\ref{ppo:ctrl}的某种特殊情况。
% Rule~\ref{ppo:addrpo} makes a similar observation to the previous rule: a store cannot be performed at memory until all previous loads that might access the same address have themselves been performed.
% Such a load must appear to execute before the store, but it cannot do so if the store were to overwrite the value in memory before the load had a chance to read the old value.
% Likewise, a store generally cannot be performed until it is known that preceding instructions will not cause an exception due to failed address resolution, and in this sense, rule~\ref{ppo:addrpo} can be seen as somewhat of a special case of rule~\ref{ppo:ctrl}.

\begin{figure}[h!]
  \centering
  \begin{tabular}{m{.4\linewidth}m{.05\linewidth}m{.4\linewidth}}
    \tt\small
    \begin{tabular}{cl||cl}
    \multicolumn{2}{c}{Hart 0} & \multicolumn{2}{c}{Hart 1} \\
    \hline
        &             &     & li t1, 1       \\
    (a) & lw a0,0(s0) & (d) & lw a1, 0(s1)   \\
    (b) & fence rw,rw & (e) & lw a2, 0(a1)   \\
    (c) & sw s2,0(s1) & (f) & sw t1, 0(s0)   \\
    \hline
    \multicolumn{4}{c}{输出结果: {\tt a0=1}, {\tt a1=t}}
    \end{tabular}  
    & &
    \input{figs/litmus_addrpo.pdf_t}
  \end{tabular}
  \caption{因为在(e)和(g)之间的额外的存储,(d)不再必须先于(g)了(允许的输出结果)
    % Because of the address dependency from (d) to (e), (d) also precedes (f) (outcome forbidden)
    }
  \label{fig:litmus:addrpo}
\end{figure}

考虑图~\ref{fig:litmus:addrpo}:在(e)的地址被决定之前,(f)不能执行,因为它可能会导致地址匹配;也就是说,{\tt a1=s0}。
因此,在(d)已经执行并证实地址是否确实重叠之前,(f)不能被送到内存。
% Consider Figure~\ref{fig:litmus:addrpo}:
% (f) cannot be executed until the address for (e) is resolved, because it may turn out that the addresses match; i.e., that {\tt a1=s0}.  Therefore, (f) cannot be sent to memory before (d) has executed and confirmed whether the addresses do indeed overlap.

\section{超出主存范围}
% \section{Beyond Main Memory}

RVWMO当前不会尝试正式地描述FENCE.I、SFENCE.VMA、I/O屏障和PMA的行为如何。所有这些的行为都将在未来的形式化中描述。
与之同时,FENCE.I的行为描述在第~\ref{chap:zifencei}章中,SFENCE.VMA的行为描述在RISC-V指令集特权架构手册之中,而I/O屏障的行为和PMA的效果将在下面描述。
% RVWMO does not currently attempt to formally describe how FENCE.I, SFENCE.VMA, I/O fences, and PMAs behave.
% All of these behaviors will be described by future formalizations.
% In the meantime, the behavior of FENCE.I is described in Chapter~\ref{chap:zifencei}, the behavior of SFENCE.VMA is described in the RISC-V Instruction Set Privileged Architecture Manual, and the behavior of I/O fences and the effects of PMAs are described below.

\subsection{一致性和可缓存性}
% \subsection{Coherence and Cacheability}

RISC-V特权ISA定义了物理内存属性(PMA),除此之外,它指定了地址空间的各部分是否是一致的和/或可缓存的。完整的细节见RISC-V特权ISA规范。这里,我们简单地讨论每个PMA中的各种细节是如何关系到内存模型的:
% The RISC-V Privileged ISA defines Physical Memory Attributes (PMAs) which specify, among other things, whether portions of the address space are coherent and/or cacheable.
% See the RISC-V Privileged ISA Specification for the complete details.
% Here, we simply discuss how the various details in each PMA relate to the memory model:

\begin{itemize}
  \item 主内存vs I/O,以及I/O内存次序PMA:定义的内存模型适用于主内存区域。I/O次序在下面讨论。
  % Main memory vs.\@ I/O, and I/O memory ordering PMAs: the memory model as defined applies to main memory regions.  I/O ordering is discussed below.
  \item 支持的访问类型和原子性PMA:内存模型简单地被应用在每个区域所支持的任何原语之上。
  % Supported access types and atomicity PMAs: the memory model is simply applied on top of whatever primitives each region supports.
  \item 可缓存性PMA:可缓存性PMA总体上不会影响内存模型。非可缓存的区域可能比可缓存的区域有更多的限制性行为,但是所允许的行为集合无论如何都不会再变化。然而,一些平台相关的和/或设备相关的可缓存性设置可能有区别。
  %  PMAs: the cacheability PMAs in general do not affect the memory model.  Non-cacheable regions may have more restrictive behavior than cacheable regions, but the set of allowed behaviors does not change regardless.  However, some platform-specific and/or device-specific cacheability settings may differ.
  \item 一致性PMA:在PMA中标记为非一致性的内存区域,其内存一致性模型当前是平台相关的和/或设备相关的:加载值公理、原子性公理和进程公理都可能被非一致性内存所违背。然而请注意,一致性内存不需要硬件缓存一致性协议。RISC-V特权ISA规范建议,主内存的硬件不一致区域是不鼓励的,但是内存模型与硬件一致性、软件一致性、由只读性内存导致的隐含一致性、由只有一个拥有权限的代理导致的隐含一致性,或者其它的一致性,相兼容。
  % Coherence PMAs: The memory consistency model for memory regions marked as non-coherent in PMAs is currently platform-specific and/or device-specific: the load-value axiom, the atomicity axiom, and the progress axiom all may be violated with non-coherent memory.  Note however that coherent memory does not require a hardware cache coherence protocol.  The RISC-V Privileged ISA Specification suggests that hardware-incoherent regions of main memory are discouraged, but the memory model is compatible with hardware coherence, software coherence, implicit coherence due to read-only memory, implicit coherence due to only one agent having access, or otherwise.
  \item 幂等性PMA:幂等性PMA被用于指定那些加载和/或存储可能有副作用的内存区域,而这反过来被微架构用来决定,例如,预取是否合法。这个区别不影响内存模型。
  % Idempotency PMAs: Idempotency PMAs are used to specify memory regions for which loads and/or stores may have side effects, and this in turn is used by the microarchitecture to determine, e.g., whether prefetches are legal.  This distinction does not affect the memory model.
\end{itemize}

\subsection{I/O排序}
% \subsection{I/O Ordering}

对于I/O,通常不会应用加载值公理和原子性公理,因为读和写都可能有设备相关的副作用,并可能把由最近的存储所“写”的值以外的值返回到相同的地址。
无论如何,下面保留的程序次序规则通常仍然适用于对I/O内存的访问:在全局内存次序中,内存访问$a$先于内存访问$b$,如果在程序次序中$a$先于$b$,并且满足如下的一个或多个条件:
% For I/O, the load value axiom and atomicity axiom in general do not apply, as both reads and writes might have device-specific side effects and may return values other than the value ``written'' by the most recent store to the same address.
% Nevertheless, the following preserved program order rules still generally apply for accesses to I/O memory:
% memory access $a$ precedes memory access $b$ in global memory order if $a$ precedes $b$ in program order and one or more of the following holds:
\begin{enumerate}
  \item 在如第~\ref{ch:memorymodel}章中定义的保留的程序次序中,$a$先于$b$,除了只适用于从一个内存操作到另一个内存操作、和从一个I/O操作到另一个I/O操作的获取和释放次序注释的例外,但从一个内存操作到一个I/O操作,或者反过来,则不是。
  % $a$ precedes $b$ in preserved program order as defined in Chapter~\ref{ch:memorymodel}, with the exception that acquire and release ordering annotations apply only from one memory operation to another memory operation and from one I/O operation to another I/O operation, but not from a memory operation to an I/O nor vice versa
  \item $a$和$b$是对于一个I/O区域中重叠地址的访问 
  % $a$ and $b$ are accesses to overlapping addresses in an I/O region
  \item $a$和$b$是对于相同的强排序的I/O区域的访问
  % $a$ and $b$ are accesses to the same strongly ordered I/O region
  \item $a$和$b$是对于I/O区域的访问,且关联到被$a$或$b$所访问的I/O区域的通道是通道1
  % $a$ and $b$ are accesses to I/O regions, and the channel associated with the I/O region accessed by either $a$ or $b$ is channel 1
  \item $a$和$b$是对于关联到相同通道(除了通道0)的I/O区域的访问
  % $a$ and $b$ are accesses to I/O regions associated with the same channel (except for channel 0)
\end{enumerate}

注意FENCE指令在其前驱集和后继集之中区分了主内存操作和I/O操作。
为了强制在I/O操作和主内存操作之间排序,代码必须使用一个带有PI、PO、SI和/或SO,加上PR、PW、SR和/或SW的FENCE。
例如,为了强制在一个对主内存的写和一个对设备寄存器的I/O写之间排序,需要一个FENCE W, O或者更强的指令。
% Note that the FENCE instruction distinguishes between main memory operations and I/O operations in its predecessor and successor sets.
% To enforce ordering between I/O operations and main memory operations, code must use a FENCE with PI, PO, SI, and/or SO, plus PR, PW, SR, and/or SW.
% For example, to enforce ordering between a write to main memory and an I/O write to a device register, a FENCE~W,O or stronger is needed.

\begin{verbbox}
  sd t0, 0(a0)
  fence w,o
  sd a0, 0(a1)
\end{verbbox}
\begin{figure}[h!]
  \centering\small
  \theverbbox
  \caption{有序的内存和I/O访问}
  \label{fig:litmus:wo}
\end{figure}

当一个屏障被实际使用时,实现必须假定设备可能尝试在接收到MMIO信号之后立即访问内存,以及来自该设备到内存的后继的内存访问必须观察到所有次序优先于该MMIO操作的访问的影响。
换句话说,在图~\ref{fig:litmus:wo}中,假设{\tt 0(a0)}是在主内存中的,而{\tt 0(a1)}是在I/O内存中的一个设备寄存器的地址。如果设备在接收到MMIO写的时候访问{\tt 0(a0)},那么根据RVWMO内存模型的规则,在概念上,该加载必须出现在第一个对0(a0)的存储之后。
在一些实现中,确保这一点的仅有的方法将是要求第一个存储确实在MMIO写被发出之前完成。其它实现可能找到了更加激进的方式,同时其它实现仍然可能不需要对I/O和主内存访问做任何完全不同的事。
无论如何,RVWMO内存模型不在这些选项中做区分;它只是简单地提供了一种与实现无关的机制来指定必须强制采用的次序。
% When a fence is in fact used, implementations must assume that the device may attempt to access memory immediately after receiving the MMIO signal, and subsequent memory accesses from that device to memory must observe the effects of all accesses ordered prior to that MMIO operation.
% In other words, in Figure~\ref{fig:litmus:wo}, suppose {\tt 0(a0)} is in main memory and {\tt 0(a1)} is the address of a device register in I/O memory.
% If the device accesses {\tt 0(a0)} upon receiving the MMIO write, then that load must conceptually appear after the first store to {\tt 0(a0)} according to the rules of the RVWMO memory model.
% In some implementations, the only way to ensure this will be to require that the first store does in fact complete before the MMIO write is issued.
% Other implementations may find ways to be more aggressive, while others still may not need to do anything different at all for I/O and main memory accesses.
% Nevertheless, the RVWMO memory model does not distinguish between these options; it simply provides an implementation-agnostic mechanism to specify the orderings that must be enforced.

许多架构包括了“次序”和“完成”屏障的独立概念,尤其是当它与I/O相关时(与常规的主内存相反)。
次序屏障简单地确保了内存操作保持有序,而完成屏障确保了,在任何后继的访问变得可见之前,前趋的访问都已经被完成。
RISC-V不会明确地区分次序屏障和完成屏障。反之,这种区分是从FENCE位的不同用法简单地推断出来的。
% Many architectures include separate notions of ``ordering'' and ``completion'' fences, especially as it relates to I/O (as opposed to regular main memory).
% Ordering fences simply ensure that memory operations stay in order, while completion fences ensure that predecessor accesses have all completed before any successors are made visible.
% RISC-V does not explicitly distinguish between ordering and completion fences.
% Instead, this distinction is simply inferred from different uses of the FENCE bits.

对于遵守RISC-V Unix平台规范的实现,I/O设备和DMA操作被要求一致性地访问内存,并且通过强排序的I/O通道完成。
因此,访问常规的主内存区域,如果该区域同时被外部设备访问,那么也可以使用标准同步机制。
不遵守Unix平台规范的实现和/或在不会一致性地访问内存的设备中,将需要使用机制(这目前是平台相关的或者设备相关的)来强制一致性。
% For implementations that conform to the RISC-V Unix Platform Specification, I/O devices and DMA operations are required to access memory coherently and via strongly ordered I/O channels.
% Therefore, accesses to regular main memory regions that are concurrently accessed by external devices can also use the standard synchronization mechanisms.
% Implementations that do not conform to the Unix Platform Specification and/or in which devices do not access memory coherently will need to use mechanisms (which are currently platform-specific or device-specific) to enforce coherency.

地址空间中的I/O区域应当被考虑为在那些区域的PMA中的非可缓存的区域。这种区域可以被PMA认为是一致性的,如果它们还没有被任何代理缓存的话。
% I/O regions in the address space should be considered non-cacheable regions in the PMAs for those regions.  Such regions can be considered coherent by the PMA if they are not cached by any agent.

这一节中的次序保证可能不适用于在RISC-V核心和设备之间的平台相关的边界。特别地,经过外部总线(例如,PCIe)发送的I/O访问可能在它们到达它们的最终目的地之前被重新排序。在那种情景中,必须根据那些外部设备和总线的平台相关的规则来强制实行排序。
% The ordering guarantees in this section may not apply beyond a platform-specific boundary between the RISC-V cores and the device.  In particular, I/O accesses sent across an external bus (e.g., PCIe) may be reordered before they reach their ultimate destination.  Ordering must be enforced in such situations according to the platform-specific rules of those external devices and buses.

\section{代码移植和映射指南}
\label{sec:memory:porting}

\begin{table}[h!]
  \centering
  \begin{tabular}{|l|l|}
    \hline
    x86/TSO 操作 & RVWMO 映射 \\
    \hline
    \hline
    加载             & \tt l\{b|h|w|d\}; fence r,rw               \\
    \hline
    存储             & \tt fence rw,w; s\{b|h|w|d\}               \\
    \hline
    \multirow{2}{*}{原子 RMW}
    & \tt amo<op>.\{w|d\}.aqrl \textrm{OR} \\
    & \tt loop:\@ lr.\{w|d\}.aq; <op>; sc.\{w|d\}.aqrl; bnez loop \\
    \hline
    屏障             & \tt fence rw,rw \\
    \hline
  \end{tabular}
  \caption{从TSO操作到RISC-V操作的映射  
  % Mappings from TSO operations to RISC-V operations
  }
  \label{tab:tsomappings}
\end{table}

表~\ref{tab:tsomappings}提供了一份从TSO内存操作到RISC-V内存指令的映射。
通常的x86加载和存储都是固有的acquire-RCpc和release-RCpc操作:TSO默认强制所有的加载-加载、加载-存储,和存储-存储排序。
因此,在RVWMO下,所有的TSO加载必须被映射到随后跟有FENCE R, RW的加载,而所有的TSO存储必须被映射到跟在存储之后的FENCE RW, W。
TSO原子读-修改-写和使用LOCK前缀的x86指令是完全排序的,并且可以或者通过一个同时设置了{\em aq}和{\em rl}的AMO实现,或者通过一个设置了{\em aq}的LR、上述提及的算数操作、一个同时设置了{\em aq}和{\em rl}的SC,还有一个检查成功条件的条件分支来实现。
在最后一种情况中,在LR上的{\em rl}注释(由于不明的原因)是多余的,并且可以被省略。
% Table~\ref{tab:tsomappings} provides a mapping from TSO memory operations onto RISC-V memory instructions.
% Normal x86 loads and stores are all inherently acquire-RCpc and release-RCpc operations: TSO enforces all load-load, load-store, and store-store ordering by default.
% Therefore, under RVWMO, all TSO loads must be mapped onto a load followed by FENCE~R,RW, and all TSO stores must be mapped onto FENCE~RW,W followed by a store.
% TSO atomic read-modify-writes and x86 instructions using the LOCK prefix are fully ordered and can be implemented either via an AMO with both {\em aq} and {\em rl} set, or via an LR with {\em aq} set, the arithmetic operation in question, an SC with both {\em aq} and {\em rl} set, and a conditional branch checking the success condition.
% In the latter case, the {\em rl} annotation on the LR turns out (for non-obvious reasons) to be redundant and can be omitted.

表~\ref{tab:tsomappings}的替代方案也是可行的。一个TSO存储可以被映射到设置了{\em rl}的AMOSWAP上。
然而,由于RVWMO PPO规则~\ref{ppo:amoforward}禁止值从AMO到后继加载的转发,对存储使用AMOSWAP可能对性能产生负面的影响。
一个TSO加载可以使用设置了{\em aq}的LR来映射:所有的这种LR指令将是无配对的,但是事实上本身并不排除使用LR进行加载。
然而,再次强调,这种映射也可能对性能有负面影响,如果它把比最初意图更多的压力放在了保留机制上的话。
% Alternatives to Table~\ref{tab:tsomappings} are also possible.
% A TSO store can be mapped onto AMOSWAP with {\em rl} set.
% However, since RVWMO PPO Rule~\ref{ppo:amoforward} forbids forwarding of values from AMOs to subsequent loads, the use of AMOSWAP for stores may negatively affect performance.
% A TSO load can be mapped using LR with {\em aq} set: all such LR instructions will be unpaired, but that fact in and of itself does not preclude the use of LR for loads.
% However, again, this mapping may also negatively affect performance if it puts more pressure on the reservation mechanism than was originally intended.

\begin{table}[h!]
  \centering
  \begin{tabular}{|l|l|}
    \hline
    Power 操作 & RVWMO 映射 \\
    \hline
    \hline
    加载              & \tt l\{b|h|w|d\}  \\
    \hline
    加载-保留      & \tt lr.\{w|d\}  \\
    \hline
    存储             & \tt s\{b|h|w|d\}  \\
    \hline
    存储-条件 & \tt sc.\{w|d\}  \\
    \hline
    \tt lwsync        & \tt fence.tso \\
    \hline
    \tt sync          & \tt fence rw,rw \\
    \hline
    \tt isync         & \tt fence.i; fence r,r \\
    \hline
  \end{tabular}
  \caption{从Power操作到RISC-V操作的映射}
  \label{tab:powermappings}
\end{table}

表~\ref{tab:powermappings}提供了一份从Power内存操作到RISC-V内存指令的映射。
Power ISYNC在RISC-V上映射到一个后跟有FENCE R, R的FENCE.I上;后一个屏障是必须的,因为ISYNC被用于定义一种“控制+控制屏障”的依赖,而它在RVWMO中是不存在的。
% Table~\ref{tab:powermappings} provides a mapping from Power memory operations onto RISC-V memory instructions.
% Power ISYNC maps on RISC-V to a FENCE.I followed by a FENCE~R,R; the latter fence is needed because ISYNC is used to define a ``control+control fence'' dependency that is not present in RVWMO.

\begin{table}[h!]
  \centering
  \begin{tabular}{|l|l|}
    \hline
    ARM 操作             & RVWMO 映射 \\
    \hline
    \hline
    加载                      & \tt l\{b|h|w|d\}  \\
    \hline
    加载-获取              & \tt fence rw, rw; l\{b|h|w|d\}; fence r,rw  \\
    \hline
    加载-独占            & \tt lr.\{w|d\}  \\
    \hline
    加载-获取-独占    & \tt lr.\{w|d\}.aqrl \\
    \hline
    存储                     & \tt s\{b|h|w|d\}  \\
    \hline
    存储-释放             & \tt fence rw,w; s\{b|h|w|d\}  \\
    \hline
    存储-独占           & \tt sc.\{w|d\}  \\
    \hline
    存储-释放-独占   & \tt sc.\{w|d\}.rl  \\
    \hline
    \tt dmb                   & \tt fence rw,rw \\
    \hline
    \tt dmb.ld                & \tt fence r,rw \\
    \hline
    \tt dmb.st                & \tt fence w,w \\
    \hline
    \tt isb                   & \tt fence.i; fence r,r \\
    \hline
  \end{tabular}
  \caption{从ARM操作到RISC-V操作的映射}
  \label{tab:armmappings}
\end{table}

表~\ref{tab:armmappings}提供了一份从ARM内存操作到RISC-V内存指令的映射。由于RISC-V目前没有带{\em aq}或{\em rl}注释的不修饰的加载和存储的操作码,ARM加载-获取和存储-释放操作应当代之以使用屏障来映射。
而且,为了强制采用存储-释放到加载-获取的次序,在存储-释放和加载-获取之间必须有一个FENCE RW, RW;
表~\ref{tab:armmappings}通过把屏障放置在每个获取操作之前,强制采用了这个次序。
ARM的load-exclusive和store-exclusive指令可以类似地映射到与它们对等的RISC-V LR和SC上,但是并非把FENCE RW, RW放在设置了{\em aq}的LR之前,而是我们也简单地用设置{\em rl}代替。
ARM ISB在RISC-V上映射到后跟有FENCE R, R的FENCE.I上,类似于ISYNC映射Power的方式。
% Table~\ref{tab:armmappings} provides a mapping from ARM memory operations onto RISC-V memory instructions.
% Since RISC-V does not currently have plain load and store opcodes with {\em aq} or {\em rl} annotations, ARM load-acquire and store-release operations should be mapped using fences instead.
% Furthermore, in order to enforce store-release-to-load-acquire ordering, there must be a FENCE~RW,RW between the store-release and load-acquire; Table~\ref{tab:armmappings} enforces this by always placing the fence in front of each acquire operation.
% ARM load-exclusive and store-exclusive instructions can likewise map onto their RISC-V LR and SC equivalents, but instead of placing a FENCE~RW,RW in front of an LR with {\em aq} set, we simply also set {\em rl} instead.
% ARM ISB maps on RISC-V to FENCE.I followed by FENCE~R,R similarly to how ISYNC maps for Power.

\begin{table}[h!]
  \centering
  \begin{tabular}{|l|l|}
    \hline
    Linux 操作           & RVWMO 映射 \\
    \hline
    \hline
    \tt smp\_mb()             & \tt fence rw,rw \\
    \hline
    \tt smp\_rmb()            & \tt fence r,r \\
    \hline
    \tt smp\_wmb()            & \tt fence w,w \\
    \hline
    \tt dma\_rmb()            & \tt fence r,r \\
    \hline
    \tt dma\_wmb()            & \tt fence w,w \\
    \hline
    \tt mb()                  & \tt fence iorw,iorw \\
    \hline
    \tt rmb()                 & \tt fence ri,ri \\
    \hline
    \tt wmb()                 & \tt fence wo,wo \\
    \hline
    \tt smp\_load\_acquire()   & \tt l\{b|h|w|d\}; fence r,rw \\
    \hline
    \tt smp\_store\_release()  & \tt fence.tso; s\{b|h|w|d\}  \\
    \hline
    \hline
    Linux 构造            & RVWMO AMO 映射        \\
    \hline
    \tt atomic\_<op>\_relaxed  & \tt amo<op>.\{w|d\}      \\
    \hline
    \tt atomic\_<op>\_acquire  & \tt amo<op>.\{w|d\}.aq   \\
    \hline
    \tt atomic\_<op>\_release  & \tt amo<op>.\{w|d\}.rl   \\
    \hline
    \tt atomic\_<op>           & \tt amo<op>.\{w|d\}.aqrl \\
    \hline
    \hline
    Linux 构造            & RVWMO LR/SC 映射\\
    \hline
    \tt atomic\_<op>\_relaxed  & \tt loop:\@ lr.\{w|d\}; <op>; sc.\{w|d\}; bnez loop \\
    \hline
    \tt atomic\_<op>\_acquire  & \tt loop:\@ lr.\{w|d\}.aq; <op>; sc.\{w|d\}; bnez loop \\
    \hline
    \multirow{2}{*}{\tt atomic\_<op>\_release}
      & \tt loop:\@ lr.\{w|d\}; <op>; sc.\{w|d\}.aqrl$^*$; bnez loop \textrm{OR} \\
      & \tt fence.tso; loop:\@ lr.\{w|d\}; <op>; sc.\{w|d\}$^*$; bnez loop \\
    \hline
    \tt atomic\_<op>           & \tt loop:\@ lr.\{w|d\}.aq; <op>; sc.\{w|d\}.aqrl; bnez loop \\
    \hline
  \end{tabular}
  \caption{从Linux内存原语到RISC-V原语的映射。其它的构造(例如自旋锁)应当相应地服从。非一致性DMA的平台或设备可能需要额外的同步(例如缓存冲刷或无效性机制);当前任何这样的额外同步都将是设备相关的。
    % Mappings from Linux memory primitives to RISC-V primitives.  Other constructs (such as spinlocks) should follow accordingly.  Platforms or devices with non-coherent DMA may need additional synchronization (such as cache flush or invalidate mechanisms); currently any such extra synchronization will be device-specific.
    }
  \label{tab:linuxmappings}
\end{table}

表~\ref{tab:linuxmappings}提供了一份Linux内存排序宏到RISC-V内存指令的映射。
Linux屏障{\tt dma\_rmb()}和{\tt dma\_wmb()}分别映射到FENCE R, R和FENCE W, W,因为RISC-V Unix平台需要一致性DMA,但是在非一致性DMA平台上将分别被映射到FENCE RI, RI和FENCE WO, WO。
非一致性DMA的平台也可以要求一种“缓存行可以被冲刷和/或无效化”的机制。这种机制将是设备相关的、和/或在未来对ISA的扩展中标准化的。
% Table~\ref{tab:linuxmappings} provides a mapping of Linux memory ordering macros onto RISC-V memory instructions.
% The Linux fences {\tt dma\_rmb()} and {\tt dma\_wmb()} map onto FENCE~R,R and FENCE~W,W, respectively, since the RISC-V Unix Platform requires coherent DMA, but would be mapped onto FENCE~RI,RI and FENCE~WO,WO, respectively, on a platform with non-coherent DMA.
% Platforms with non-coherent DMA may also require a mechanism by which cache lines can be flushed and/or invalidated.
% Such mechanisms will be device-specific and/or standardized in a future extension to the ISA.

Linux对于释放操作的映射可能看起来比必要的更强,但是需要这些映射去覆盖某些Linux需要比更直观的映射将提供的更强的次序的情况。
特别地,在本文正在被编写的时候,Linux正在积极地讨论,在一个临界区中的访问和相同硬件线程中的一个后继的临界区中的访问之间,
是否需要加载-加载、加载-存储,和存储-存储的次序,并由相同的同步对象进行保护。
不是所有的FENCE RW, W/FENCE R, RW映射和{\em aq}/{\em rl}映射的组合都能提供这种次序。围绕这个问题有这样一些方法,包括:
% The Linux mappings for release operations may seem stronger than necessary, but these mappings are needed to cover some cases in which Linux requires stronger orderings than the more intuitive mappings would provide.
% In particular, as of the time this text is being written, Linux is actively debating whether to require load-load, load-store, and store-store orderings between accesses in one critical section and accesses in a subsequent critical section in the same hart and protected by the same synchronization object.
% Not all combinations of FENCE~RW,W/FENCE~R,RW mappings with {\em aq}/{\em rl} mappings combine to provide such orderings.
% There are a few ways around this problem, including:
\begin{enumerate}
  \item 永远使用FENCE RW, W/FENCE R, RW,并且永远不使用{\em aq}/{\em rl}。这是足够的,但是不可取,因为它违背了{\em aq}/{\em rl}修饰符的目的。
  % Always use FENCE~RW,W/FENCE~R,RW, and never use {\em aq}/{\em rl}.  This suffices but is undesirable, as it defeats the purpose of the {\em aq}/{\em rl} modifiers.
  \item 永远使用{\em aq}/{\em rl},并永远不使用FENCE RW, W/FENCE R, RW。这目前不会起作用,因为缺少带有{\em aq}和{\em rl}修饰符的加载和存储操作码。
  % Always use {\em aq}/{\em rl}, and never use FENCE~RW,W/FENCE~R,RW.  This does not currently work due to the lack of load and store opcodes with {\em aq} and {\em rl} modifiers.
  \item 加强释放操作的映射,使得它们将在现有的任何种类的获取映射中强制采用充分的次序。这是当前推荐的方案,该方案展示在表~\ref{tab:linuxmappings}中。
  % Strengthen the mappings of release operations such that they would enforce sufficient orderings in the presence of either type of acquire mapping.  This is the currently recommended solution, and the one shown in Table~\ref{tab:linuxmappings}.
\end{enumerate}

\begin{figure}[h!]
  \centering\small
  \begin{verbbox}
Linux 代码:
(a)  int r0 = *x;
(bc) spin_unlock(y, 0);
     ...
     ...
(d)  spin_lock(y);
(e)  int r1 = *z;
  \end{verbbox}
  \theverbbox
  ~~~~~~~~~~
  \begin{verbbox}
RVWMO 映射:
(a) lw           a0, 0(s0)
(b) fence.tso  // vs. fence rw,w
(c) sd           x0,0(s1)
    ...
    loop:
(d) amoswap.d.aq a1,t1,0(s1)
    bnez         a1,loop
(e) lw           a2,0(s2)
  \end{verbbox}
  \theverbbox
  \caption{Linux中临界区之间的次序}
  \label{fig:litmus:lkmm_ll}
\end{figure}

例如,Linux社区当前正在讨论临界区次序规则,该规则将要求图~\ref{fig:litmus:lkmm_ll}中的(a)被排序在(e)之前。如果确实那样要求,那么把(b)映射为FENCE RW, W将是不充分的。也就是说,随着Linux内核内存模型的演化,这些映射也将随之变化。
% For example, the critical section ordering rule currently being debated by the Linux community would require (a) to be ordered before (e) in Figure~\ref{fig:litmus:lkmm_ll}.
% If that will indeed be required, then it would be insufficient for (b) to map as FENCE~RW,W.
% That said, these mappings are subject to change as the Linux Kernel Memory Model evolves.

\begin{table}[h!]
  \centering
  \begin{tabular}{|l|l|}
    \hline
    C/C++ 构造                            & RVWMO 映射 \\
    \hline
    \hline
    Non-atomic load                            & \tt l\{b|h|w|d\}               \\
    \hline
    \tt atomic\_load(memory\_order\_relaxed)   & \tt l\{b|h|w|d\}               \\
    \hline
    %\tt atomic\_load(memory\_order\_consume)   & \multicolumn{2}{l|}{\tt l\{b|h|w|d\}; fence r,rw}   \\
    %\hline
    \tt atomic\_load(memory\_order\_acquire)   & \tt l\{b|h|w|d\}; fence r,rw    \\
    \hline
    \tt atomic\_load(memory\_order\_seq\_cst)  & \tt fence rw,rw; l\{b|h|w|d\}; fence r,rw       \\
    \hline
    \hline
    非原子性存储                           & \tt s\{b|h|w|d\}               \\
    \hline
    \tt atomic\_store(memory\_order\_relaxed)  & \tt s\{b|h|w|d\}               \\
    \hline
    \tt atomic\_store(memory\_order\_release)  & \tt fence rw,w; s\{b|h|w|d\}  \\
    \hline
    \tt atomic\_store(memory\_order\_seq\_cst) & \tt fence rw,w; s\{b|h|w|d\}  \\
    \hline
    \hline
    \tt atomic\_thread\_fence(memory\_order\_acquire)  & \tt fence r,rw \\
    \hline
    \tt atomic\_thread\_fence(memory\_order\_release)  & \tt fence rw,w \\
    \hline
    \tt atomic\_thread\_fence(memory\_order\_acq\_rel) & {\tt fence.tso} \\
    \hline
    \tt atomic\_thread\_fence(memory\_order\_seq\_cst) & \tt fence rw,rw \\
    \hline
    \hline
    C/C++ 构造                          & RVWMO AMO 映射        \\
    \hline
    \tt atomic\_<op>(memory\_order\_relaxed)  & \tt amo<op>.\{w|d\}      \\
    \hline
    \tt atomic\_<op>(memory\_order\_acquire)  & \tt amo<op>.\{w|d\}.aq   \\
    \hline
    \tt atomic\_<op>(memory\_order\_release)  & \tt amo<op>.\{w|d\}.rl   \\
    \hline
    \tt atomic\_<op>(memory\_order\_acq\_rel) & \tt amo<op>.\{w|d\}.aqrl \\
    \hline
    \tt atomic\_<op>(memory\_order\_seq\_cst) & \tt amo<op>.\{w|d\}.aqrl \\
    \hline
    \hline
    C/C++ 构造                           & RVWMO LR/SC 映射\\
    \hline
    \multirow{2}{*}{\tt atomic\_<op>(memory\_order\_relaxed)}
      & \tt loop:\@ lr.\{w|d\}; <op>; sc.\{w|d\}; \\
      & \tt bnez loop \\
    \hline
    \multirow{2}{*}{\tt atomic\_<op>(memory\_order\_acquire)}
      & \tt loop:\@ lr.\{w|d\}.aq; <op>; sc.\{w|d\}; \\
      & \tt bnez loop \\
    \hline
    \multirow{2}{*}{\tt atomic\_<op>(memory\_order\_release)}
      & \tt loop:\@ lr.\{w|d\}; <op>; sc.\{w|d\}.rl; \\
      & \tt bnez loop \\
    \hline
    \multirow{2}{*}{\tt atomic\_<op>(memory\_order\_acq\_rel)}
      & \tt loop:\@ lr.\{w|d\}.aq; <op>; sc.\{w|d\}.rl; \\
      & \tt bnez loop \\
    \hline
    \multirow{2}{*}{\tt atomic\_<op>(memory\_order\_seq\_cst)}
      & \tt loop:\@ lr.\{w|d\}.aqrl; <op>; \\
      & \tt sc.\{w|d\}.rl; bnez loop \\
    \hline
  \end{tabular}
  \caption{从C/C++原语到RISC-V原语的映射。
    % Mappings from C/C++ primitives to RISC-V primitives.
    }
  \label{tab:c11mappings}
\end{table}

\begin{table}[h!]
  \centering
  \begin{tabular}{|l|l|}
    \hline
    C/C++ 构造                            & RVWMO 映射 \\
    \hline
    \hline
    非原子性加载                            & \tt l\{b|h|w|d\}               \\
    \hline
    \tt atomic\_load(memory\_order\_relaxed)   & \tt l\{b|h|w|d\}               \\
    \hline
    \tt atomic\_load(memory\_order\_acquire)   & \tt l\{b|h|w|d\}.aq  \\
    \hline
    \tt atomic\_load(memory\_order\_seq\_cst)  & \tt l\{b|h|w|d\}.aq  \\
    \hline
    \hline
    非原子性存储                          & \tt s\{b|h|w|d\}               \\
    \hline
    \tt atomic\_store(memory\_order\_relaxed)  & \tt s\{b|h|w|d\}               \\
    \hline
    \tt atomic\_store(memory\_order\_release)  & \tt s\{b|h|w|d\}.rl  \\
    \hline
    \tt atomic\_store(memory\_order\_seq\_cst) & \tt s\{b|h|w|d\}.rl \\
    \hline
    \hline
    \tt atomic\_thread\_fence(memory\_order\_acquire)  & \tt fence r,rw \\
    \hline
    \tt atomic\_thread\_fence(memory\_order\_release)  & \tt fence rw,w \\
    \hline
    \tt atomic\_thread\_fence(memory\_order\_acq\_rel) & {\tt fence.tso} \\
    \hline
    \tt atomic\_thread\_fence(memory\_order\_seq\_cst) & \tt fence rw,rw \\
    \hline
    \hline
    C/C++ 构造                           & RVWMO AMO 映射    \\
    \hline
    \tt atomic\_<op>(memory\_order\_relaxed)  & \tt amo<op>.\{w|d\}      \\
    \hline
    \tt atomic\_<op>(memory\_order\_acquire)  & \tt amo<op>.\{w|d\}.aq   \\
    \hline
    \tt atomic\_<op>(memory\_order\_release)  & \tt amo<op>.\{w|d\}.rl   \\
    \hline
    \tt atomic\_<op>(memory\_order\_acq\_rel) & \tt amo<op>.\{w|d\}.aqrl \\
    \hline
    \tt atomic\_<op>(memory\_order\_seq\_cst) & \tt amo<op>.\{w|d\}.aqrl \\
    \hline
    \hline
    C/C++ 构造                           & RVWMO LR/SC 映射\\
    \hline
    \tt atomic\_<op>(memory\_order\_relaxed)  & \tt lr.\{w|d\}; <op>; sc.\{w|d\} \\
    \hline
    \tt atomic\_<op>(memory\_order\_acquire)  & \tt lr.\{w|d\}.aq; <op>; sc.\{w|d\} \\
    \hline
    \tt atomic\_<op>(memory\_order\_release)  & \tt lr.\{w|d\}; <op>; sc.\{w|d\}.rl \\
    \hline
    \tt atomic\_<op>(memory\_order\_acq\_rel) & \tt lr.\{w|d\}.aq; <op>; sc.\{w|d\}.rl \\
    \hline
    \tt atomic\_<op>(memory\_order\_seq\_cst) & \tt lr.\{w|d\}.aq$^*$; <op>; sc.\{w|d\}.rl \\
    \hline
    \multicolumn{2}{l}{为了能与表~\ref{tab:c11mappings}中的各个代码映射相互操作,$^*$必须是{\tt lr.\{w|d\}.aqrl}  
    % $^*$must be {\tt lr.\{w|d\}.aqrl} in order to interoperate with code mapped per Table~\ref{tab:c11mappings}
    }
  \end{tabular}
  \caption{假设的从C/C++原语到RISC-V原语的映射,如果引入了原生的加载-获取和存储释放操作码的话。
    % Hypothetical mappings from C/C++ primitives to RISC-V primitives, if native load-acquire and store-release opcodes are introduced.
    }
  \label{tab:c11mappings_hypothetical}
\end{table}

表~\ref{tab:c11mappings}提供了一份C11/C++11原子操作到RISC-V内存指令的映射。
如果引入了带有{\em aq}和{\em rl}修饰符的加载和存储操作码,那么表~\ref{tab:c11mappings_hypothetical}中的映射就将足够了。
然而要注意,只有当原子的{\tt atomic\_<op>(内存\_次序\_seq\_cst)}被使用一个同时设置了{\em aq}和{\em rl}的LR映射时,这两个映射才会正确地互通。
% Table~\ref{tab:c11mappings} provides a mapping of C11/C++11 atomic operations onto RISC-V memory instructions.
% If load and store opcodes with {\em aq} and {\em rl} modifiers are introduced, then the mappings in Table~\ref{tab:c11mappings_hypothetical} will suffice.
% Note however that the two mappings only interoperate correctly if {\tt atomic\_<op>(memory\_order\_seq\_cst)} is mapped using an LR that has both {\em aq} and {\em rl} set.

任何AMO可以通过一个LR/SC对来模拟,但是必须注意确保任何源自于LR的PPO次序也是源自于SC的,并且任何在SC终止的PPO次序也会使得在LR处终止。
例如,LR必须也要服从任何AMO拥有的数据依赖,使得加载操作否则将没有任何数据依赖的概念。类似地,相同硬件线程中的其它地方的一个FENCE R, R的影响也必须适用于SC,否则将不会遵从该屏障。
模拟器可以系通过简单地把AMO映射到{\tt lr.aq;~<op>;~sc.aqrl}上来达成这个效果,与其它地方的用于全排序原子性的映射相匹配。
% Any AMO can be emulated by an LR/SC pair, but care must be taken to ensure that any PPO orderings that originate from the LR are also made to originate from the SC, and that any PPO orderings that terminate at the SC are also made to terminate at the LR.
% For example, the LR must also be made to respect any data dependencies that the AMO has, given that load operations do not otherwise have any notion of a data dependency.
% Likewise, the effect a FENCE~R,R elsewhere in the same hart must also be made to apply to the SC, which would not otherwise respect that fence.
% The emulator may achieve this effect by simply mapping AMOs onto {\tt lr.aq;~<op>;~sc.aqrl}, matching the mapping used elsewhere for fully ordered atomics.

这些C11/C++11映射需要平台为所有内存提供下列物理内存属性(正如RISC-V特权ISAS中定义的那样):
% These C11/C++11 mappings require the platform to provide the following Physical Memory Attributes (as defined in the RISC-V Privileged ISA) for all memory:
\begin{itemize}
  \item 主内存
  \item 连贯性
  \item AMOArithmetic
  \item RsrvEventual
\end{itemize}
具有不同属性的平台可能需要不同的映射,或者需要平台相关的SW(例如,内存映射I/O)。
% Platforms with different attributes may require different mappings, or require platform-specific SW (e.g., memory-mapped I/O).

\section{实现指南}
% \section{Implementation Guidelines}

RVWMO和RVTSO内存模型绝不排除微架构采用复杂的推测技术或其它形式的优化来提供更高的性能。
模型也不采用任何需要使用任何一个特定的缓存层次的需求,甚至一点也不使用缓存一致性协议。
相反,这些模型只指定了可以暴露给软件的行为。微架构可以自由地使用任何流水线设计,任何一致性或非一致性缓存层次,任何片上互连,等等,
只要设计只认可满足内存模型规则的执行。也就是说,为了帮助人们理解内存模型的实际实现,本节中我们提供了一些关于架构师和编程人员应当如何解释模型的规则的指导。
% The RVWMO and RVTSO memory models by no means preclude microarchitectures from employing sophisticated speculation techniques or other forms of optimization in order to deliver higher performance.
% The models also do not impose any requirement to use any one particular cache hierarchy, nor even to use a cache coherence protocol at all.
% Instead, these models only specify the behaviors that can be exposed to software.
% Microarchitectures are free to use any pipeline design, any coherent or non-coherent cache hierarchy, any on-chip interconnect, etc., as long as the design only admits executions that satisfy the memory model rules.
% That said, to help people understand the actual implementations of the memory model, in this section we provide some guidelines on how architects and programmers should interpret the models' rules.

RVWMO和RVTSO都是多重拷贝原子性(或者“其它多重拷贝原子性”)的:任何对一个硬件线程(除了最初发出它的那个硬件线程)可见的存储的值,在概念上必须也对系统中的所有的其它硬件线程可见。
换句话说,硬件线程可以从它们自己的先前的存储进行转发,时机在那些存储变得对所有的硬件线程全局可见之前,但是提前在硬件线程之间进行转发是不被允许的。
多重拷贝原子性可以通过多种方式被采用。它可能由于缓存和存储缓冲区的物理设计而固有地存在,它可以通过一种单一写者/多重读者的缓存一致性协议来采用,或者它可以由于某些其它机制而存在。
% Both RVWMO and RVTSO are multi-copy atomic (or ``other-multi-copy-atomic''): any store value that is visible to a hart other than the one that originally issued it must also be conceptually visible to all other harts in the system.
% In other words, harts may forward from their own previous stores before those stores have become globally visible to all harts, but no early inter-hart forwarding is permitted.
% Multi-copy atomicity may be enforced in a number of ways.
% It might hold inherently due to the physical design of the caches and store buffers, it may be enforced via a single-writer/multiple-reader cache coherence protocol, or it might hold due to some other mechanism.

尽管多重拷贝原子性的确在微架构上采用了一些限制,但是它是保持内存模型免于变得极度复杂的关键属性之一。
例如,一个硬件线程不可以从一个邻居硬件线程的私有存储缓冲区合法地转发一个值(当然,除非它这么做,不会有新的非法行为变得架构可见)。
一个缓存一致性协议也不能在其已经无效化了所有来自其它缓存的更旧的拷贝之前,从一个硬件线程向另一个硬件线程转发一个值。
当然,微架构可以(并且高性能实现可能会)通过推测或其它的优化来暗中违背这些规则,只要任何不合规的行为都不会暴露给编程人员。
% Although multi-copy atomicity does impose some restrictions on the microarchitecture, it is one of the key properties keeping the memory model from becoming extremely complicated.
% For example, a hart may not legally forward a value from a neighbor hart's private store buffer (unless of course it is done in such a way that no new illegal behaviors become architecturally visible).
% Nor may a cache coherence protocol forward a value from one hart to another until the coherence protocol has invalidated all older copies from other caches.
% Of course, microarchitectures may (and high-performance implementations likely will) violate these rules under the covers through speculation or other optimizations, as long as any non-compliant behaviors are not exposed to the programmer.

作为一份解释RVWMO中的PPO规则的粗略指南,从软件的角度,我们期待看到以下情形:
% As a rough guideline for interpreting the PPO rules in RVWMO, we expect the following from the software perspective:
\begin{itemize}
  \item 编程人员将有规律地和积极地使用PPO规则\ref{ppo:->st}和\ref{ppo:fence}-\ref{ppo:pair}。
  % programmers will use PPO rules \ref{ppo:->st} and \ref{ppo:fence}--\ref{ppo:pair} regularly and actively.
  \item 专业的编程人员将使用PPO规则\ref{ppo:addr}-\ref{ppo:ctrl}来加速重要数据结构的关键路径。
  % expert programmers will use PPO rules \ref{ppo:addr}--\ref{ppo:ctrl} to speed up critical paths of important data structures.
  \item 即使是专业的编程人员也将很少直接使用PPO规则\ref{ppo:rdw}-\ref{ppo:amoforward}和\ref{ppo:addrdatarfi}-\ref{ppo:addrpo},如果它们有的话。
  % even expert programmers will rarely if ever use PPO rules \ref{ppo:rdw}--\ref{ppo:amoforward} and \ref{ppo:addrdatarfi}--\ref{ppo:addrpo} directly.  These are included to facilitate common microarchitectural optimizations (rule~\ref{ppo:rdw}) and the operational formal modeling approach (rules \ref{ppo:amoforward} and \ref{ppo:addrdatarfi}--\ref{ppo:addrpo}) described in Section~\ref{sec:operational}.  They also facilitate the process of porting code from other architectures that have similar rules.
\end{itemize}

% -------------  word文档里有这一段,但是tex源文件中没有,可能是这版tex去掉了这段话,需要再三审核。--------------
% 包含这些规则是为了方便在B.3节中描述的常见微架构优化(规则2)和操作形式化模型方法(规则3和12-13)。它们也有助于从拥有相似规则的其它架构移植代码的进程。


从硬件的角度,我们也希望看到下列情况:
% We also expect the following from the hardware perspective:
\begin{itemize}
  \item PPO规则\ref{ppo:->st}和\ref{ppo:amoforward}-\ref{ppo:release}反映了好理解的规则,应当不会给架构师带来什么惊喜。
  % PPO rules \ref{ppo:->st} and \ref{ppo:amoforward}--\ref{ppo:release} reflect well-understood rules that should pose few surprises to architects.
  \item PPO规则\ref{ppo:rdw}反映了一个自然的和常见的硬件优化,但那是一个非常微妙的优化,因此值得仔细地复查。
  % PPO rule \ref{ppo:rdw} reflects a natural and common hardware optimization, but one that is very subtle and hence is worth double checking carefully.
  \item PPO规则\ref{ppo:rcsc}可能对于架构师来说不会立刻感到显然,但是它是标准内存模型的需求。
  % PPO rule \ref{ppo:rcsc} may not be immediately obvious to architects, but it is a standard memory model requirement
  \item 加载值公理、原子性公理,和PPO规则\ref{ppo:pair}-\ref{ppo:addrpo}反映了大多数硬件实现将自然地采用的规则,除非它们包含了极度的优化。当然,尽管如此,实现应当确保复查这些规则。硬件也必须确保句法依赖不会“被优化掉”。
  % The load value axiom, the atomicity axiom, and PPO rules \ref{ppo:pair}--\ref{ppo:addrpo} reflect rules that most hardware implementations will enforce naturally, unless they contain extreme optimizations.  Of course, implementations should make sure to double check these rules nevertheless.  Hardware must also ensure that syntactic dependencies are not ``optimized away''.
\end{itemize}

架构可以自由地实现任何的内存模型规则,正如它们选择的那样保守。例如,一个硬件实现可以选择做到下列中的任何或者所有:
% Architectures are free to implement any of the memory model rules as conservatively as they choose.  For example, a hardware implementation may choose to do any or all of the following:
  \begin{itemize}
    \item 无论各位实际如何设置,把所有的屏障都按它们是FENCE RW, RW(或者FENCE IORW, IORW,如果涉及了I/O)来解释
    % interpret all fences as if they were FENCE~RW,RW (or FENCE~IORW,IORW, if I/O is involved), regardless of the bits actually set
    \item 把所有的带PW和SR的屏障都按它们是FENCE RW, RW(或者FENCE IORW, IORW,如果涉及了I/O)来实现,因为无论如何,带SR的PW都是四个可能的主内存次序组件中最昂贵的
    % implement all fences with PW and SR as if they were FENCE~RW,RW (or FENCE~IORW,IORW, if I/O is involved), as PW with SR is the most expensive of the four possible main memory ordering components anyway
    \item 像~\ref{sec:memory:porting}节中描述的那样模拟{\em aq}和{\em rl}
    % emulate {\em aq} and {\em rl} as described in Section~\ref{sec:memory:porting}
    \item 强制所有的相同地址的加载-加载次序,即使存在在诸如“fri-rfi”和“RSW”等式样
    %  all same-address load-load ordering, even in the presence of patterns such as ``fri-rfi'' and ``RSW''
    \item 禁止任何从存储缓冲区中的存储到相同地址的后继的AMO或LR的值的转发
    % forbid any forwarding of a value from a store in the store buffer to a subsequent AMO or LR to the same address
    \item 禁止任何从存储缓冲区中的AMO或SC到相同地址的后继的加载的值的转发
    % forbid any forwarding of a value from an AMO or SC in the store buffer to a subsequent load to the same address
    \item 在所有内存访问上实现TSO,并忽略任何不包括PW和SR次序的主内存屏障(例如,就像Ztso实现会做的那样)
    % implement TSO on all memory accesses, and ignore any main memory fences that do not include PW and SR ordering (e.g., as Ztso implementations will do)
    \item 把所有的原子性实现为RCsc或者甚至是全排序的,无论注释如何
    % implement all atomics to be RCsc or even fully ordered, regardless of annotation
  \end{itemize}

实现了RVTSO的架构可以安全地做到下列事情:
% Architectures that implement RVTSO can safely do the following:
\begin{itemize}
  \item 忽略所有不同时具有PW和SR的屏障(除非该屏障也排序了I/O)
  % Ignore all fences that do not have both PW and SR (unless the fence also orders I/O)
  \item 忽略所有除了规则\ref{ppo:fence}至\ref{ppo:rcsc}之外的PPO规则,因为在RVTSO的假设下,剩下的规则与其它的PPO规则是多余的
  % Ignore all PPO rules except for rules \ref{ppo:fence} through \ref{ppo:rcsc}, since the rest are redundant with other PPO rules under RVTSO assumptions
\end{itemize}

其它的一般注意事项:
% Other general notes:

\begin{itemize}
  \item 从内存模型的观点来看,静默存储(例如,写入与内存位置已经存在的值相同的值的存储)的行为与任何其它的存储相像。
  类似地,不实际改变内存中的值的AMO(例如,一个{\em rs2}中的值比内存中当前的值更小的AMOMAX)仍然在语义上被认为是存储操作。
  尝试实现静默存储的微架构必须小心地确保仍然服从内存模型,特别是在诸如RSW(章节~\ref{sec:memory:overlap})的情形中,它们往往与静默存储不兼容。
  % Silent stores (i.e., stores that write the same value that already exists at a memory location) behave like any other store from a memory model point of view.  Likewise, AMOs which do not actually change the value in memory (e.g., an AMOMAX for which the value in {\em rs2} is smaller than the value currently in memory) are still semantically considered store operations.  Microarchitectures that attempt to implement silent stores must take care to ensure that the memory model is still obeyed, particularly in cases such as RSW (Section~\ref{sec:memory:overlap}) which tend to be incompatible with silent stores.
  \item 写可以被融合(即,对相同地址的两个连续的写可以被融合)或归并(即,对于相同地址的两个背靠背的写,可以省略较早的那个),只要结果行为不会违反内存模型语义。
  % Writes may be merged (i.e., two consecutive writes to the same address may be merged) or subsumed (i.e., the earlier of two back-to-back writes to the same address may be elided) as long as the resulting behavior does not otherwise violate the memory model semantics.
\end{itemize}

写归并的问题可以从下面的例子来理解:
% The question of write subsumption can be understood from the following example:
\begin{figure}[h!]
  \centering
  \begin{tabular}{m{.4\linewidth}m{.1\linewidth}m{.4\linewidth}}
    \tt\small
    \begin{tabular}{cl||cl}
    \multicolumn{2}{c}{Hart 0} & \multicolumn{2}{c}{Hart 1} \\
    \hline
        & li t1, 3    &     & li  t3, 2    \\
        & li t2, 1    &     &              \\
    (a) & sw t1,0(s0) & (d) & lw  a0,0(s1) \\
    (b) & fence w, w  & (e) & sw  a0,0(s0) \\
    (c) & sw t2,0(s1) & (f) & sw  t3,0(s0) \\
    \end{tabular}
  & &
    \input{figs/litmus_subsumption.pdf_t}
  \end{tabular}
  \caption{写归并石蕊测试,允许的执行。
  %  Write subsumption litmus test, allowed execution.
  }
  \label{fig:litmus:subsumption}
\end{figure}

如前所写,如果加载(d)读取值$1$,然后(a)在全局内存次序中必须先于(f):
% As written, if the load ~(d) reads value~$1$, then (a) must precede (f) in the global memory order:
\begin{itemize}
  \item 由于规则2,在全局内存次序中,(a)先于(c)
  % (a) precedes (c) in the global memory order because of rule 2
  \item 由于加载值公理,在全局内存次序中,(c)先于(d)
  % (c) precedes (d) in the global memory order because of the Load Value axiom
  \item 由于规则7,在全局内存次序中,(d)先于(e)
  % (d) precedes (e) in the global memory order because of rule 7
  \item 由于规则1,在全局内存次序中,(e)先于(f)
  % (e) precedes (f) in the global memory order because of rule 1
\end{itemize}
换句话说,地址在{\tt s0}中的内存位置的最终取值必定是$2$(由存储(f)写入的值),而不能是$3$(由存储(a)写入的值)。
% In other words the final value of the memory location whose address is in {\tt s0} must be~$2$ (the value written by the store~(f)) and cannot be~$3$ (the value written by the store~(a)).

一个非常激进的微架构可能会错误地决定丢弃(e),因为(f)取代了它,而这可能反过来导致微架构打破(d)和(f)之间现在已消除的依赖
(并因此也会打破(a)和(f)之间的)。这将违反内存模型规则,因而它是被禁止的。
写归并可能在其他情形中是合法的,如果,例如在(d)和(e)之间没有数据依赖的话。
% A very aggressive microarchitecture might erroneously decide to discard (e), as (f) supersedes it, and this may in turn lead the microarchitecture to break the now-eliminated dependency between (d) and (f) (and hence also between (a) and (f)).
% This would violate the memory model rules, and hence it is forbidden.
% Write subsumption may in other cases be legal, if for example there were no data dependency between (d) and (e).

\subsection{未来可能的扩展}
% \subsection{Possible Future Extensions}

我们希望任何或所有的下列可能的未来扩展都将与RVWMO内存模型兼容:
% We expect that any or all of the following possible future extensions would be compatible with the RVWMO memory model:

\begin{itemize}
  \item ‘V’向量ISA扩展  
  % `V' vector ISA extensions
  \item ‘J’JIT扩展 
  % `J' JIT extension
  \item 用于设置了{\em aq}和{\em rl}的加载和存储操作码的原生编码
  % Native encodings for load and store opcodes with {\em aq} and {\em rl} set
  \item 限制到特定地址的屏障
  % Fences limited to certain addresses
  \item 缓存的写回/冲刷/无效化/等等……指令
  % Cache writeback/flush/invalidate/etc.\@ instructions
\end{itemize}

\section{已知问题}
% \section{Known Issues}
\label{sec:memory:discrepancies}

\subsection{混合尺寸的RSW}
% \subsection{Mixed-size RSW}
\label{sec:memory:discrepancies:mixedrsw}

\begin{figure}[h!]
  \centering\small
  {\tt
    \begin{tabular}{cl||cl}
    \multicolumn{2}{c}{Hart 0} & \multicolumn{2}{c}{Hart 1} \\
    \hline
          & li t1, 1    &     & li t1, 1    \\
      (a) & lw a0,0(s0) & (d) & lw a1,0(s1) \\
      (b) & fence rw,rw & (e) & amoswap.w.rl a2,t1,0(s2) \\
      (c) & sw t1,0(s1) & (f) & ld a3,0(s2) \\
          &             & (g) & lw a4,4(s2) \\
          &             &     & xor a5,a4,a4  \\
          &             &     & add s0,s0,a5  \\
          &             & (h) & sw a2,0(s0)   \\
      \hline
      \multicolumn{4}{c}{输出结果: {\tt a0=1}, {\tt a1=1}, {\tt a2=0}, {\tt a3=1}, {\tt a4=0}}
    \end{tabular}
  }
  \caption{混合尺寸的差异(被公理模型所允许,被操作模型所禁止)
    % Mixed-size discrepancy (permitted by axiomatic models, forbidden by operational model)
    }
  \label{fig:litmus:discrepancy:rsw1}
\end{figure}

\begin{figure}[h!]
  \centering\small
  {\tt
    \begin{tabular}{cl||cl}
    \multicolumn{2}{c}{Hart 0} & \multicolumn{2}{c}{Hart 1} \\
    \hline
          & li t1, 1    &     & li t1, 1      \\
      (a) & lw a0,0(s0) & (d) & ld a1,0(s1)   \\
      (b) & fence rw,rw & (e) & lw a2,4(s1)   \\
      (c) & sw t1,0(s1) &     & xor a3,a2,a2  \\
          &             &     & add s0,s0,a3  \\
          &             & (f) & sw a2,0(s0)   \\
      \hline
      \multicolumn{4}{c}{输出结果: {\tt a0=0}, {\tt a1=1}, {\tt a2=0}}
    \end{tabular}
  }
  \caption{混合尺寸的差异(被公理模型所允许,被操作模型所禁止)
    % Mixed-size discrepancy (permitted by axiomatic models, forbidden by operational model)
    }
  \label{fig:litmus:discrepancy:rsw2}
\end{figure}

\begin{figure}[h!]
  \centering\small
  {\tt
    \begin{tabular}{cl||cl}
    \multicolumn{2}{c}{Hart 0} & \multicolumn{2}{c}{Hart 1} \\
    \hline
          & li t1, 1    &     & li t1, 1      \\
      (a) & lw a0,0(s0) & (d) & sw t1,4(s1)   \\
      (b) & fence rw,rw & (e) & ld a1,0(s1)   \\
      (c) & sw t1,0(s1) & (f) & lw a2,4(s1)   \\
          &             &     & xor a3,a2,a2  \\
          &             &     & add s0,s0,a3  \\
          &             & (g) & sw a2,0(s0)   \\
      \hline
      \multicolumn{4}{c}{Outcome: {\tt a0=1}, {\tt a1=0x100000001}, {\tt a1=1}}
    \end{tabular}
  }
  \caption{混合尺寸的差异(被公理模型所允许,被操作模型所禁止)}
  \label{fig:litmus:discrepancy:rsw3}
\end{figure}

在混合尺寸的RSW变体家族中,操作规范和公理规范之间有一种已知的差异性,显示在表~\ref{fig:litmus:discrepancy:rsw1}—\ref{fig:litmus:discrepancy:rsw3}中。
为了解决这个差异性,我们可以选择添加一些像是下列新的PPO规则的东西:在保留的程序次序中(并因此也在全局内存次序中),
内存操作$a$先于内存操作$b$,如果在程序次序中a先于b,a和b都访问常规主内存(而不是I/O区域),$a$是一个加载,$b$是一个存储,
那么在$a$和$b$之间存在一个加载$m$,$a$和$m$都读取一个位$x$,在$a$和$m$之间没有写$x$的存储,且在PPO中$m$先于$b$。换句话说,在{\sf herd}语法中,
我们可以选择向PPO添加“{\tt (po-loc \& rsw);ppo;[W]}”。许多实现已经自然地采用了这种次序。严格来说,即使这个规则不是官方的,
尽管如此,我们也推荐实现者采用它,以便确保未来可能把这个规则添加到RVWMO的向前兼容性。
% There is a known discrepancy between the operational and axiomatic specifications within the family of mixed-size RSW variants shown in Figures~\ref{fig:litmus:discrepancy:rsw1}--\ref{fig:litmus:discrepancy:rsw3}.
% To address this, we may choose to add something like the following new PPO rule:
% Memory operation $a$ precedes memory operation $b$ in preserved program order (and hence also in the global memory order) if $a$ precedes $b$ in program order, $a$ and $b$ both access regular main memory (rather than I/O regions), $a$ is a load, $b$ is a store, there is a load $m$ between $a$ and $b$, there is a byte $x$ that both $a$ and $m$ read, there is no store between $a$ and $m$ that writes to $x$, and $m$ precedes $b$ in PPO.
% In other words, in {\sf herd} syntax, we may choose to add ``{\tt (po-loc \& rsw);ppo;[W]}'' to PPO.
% Many implementations will already enforce this ordering naturally.
% As such, even though this rule is not official, we recommend that implementers enforce it nevertheless in order to ensure forwards compatibility with the possible future addition of this rule to RVWMO.


\chapter{形式化的内存模型规范(0.1版本)}
% \chapter{Formal Memory Model Specifications, Version 0.1}
为了便于对RVWMO的形式化分析,这章使用不同的工具和建模方法呈现了一组形式化。
任何差异性都是无意的;希望这些模型确实描述了相同的合法行为集合。
% To facilitate formal analysis of RVWMO, this chapter presents a set of formalizations using different tools and modeling approaches.  Any discrepancies are unintended; the expectation is that the models describe exactly the same sets of legal behaviors.

这个附录应当被视为注释;所有的规范材料都提供在第~\ref{ch:memorymodel}章和ISA规范主体的剩余部分中。第~\ref{sec:memory:discrepancies}节中列出了所有当前已知的差异性。任何其它的差异性都是无意的。
% This appendix should be treated as commentary; all normative material is provided in Chapter~\ref{ch:memorymodel} and in the rest of the main body of the ISA specification.
% All currently known discrepancies are listed in Section~\ref{sec:memory:discrepancies}.
% Any other discrepancies are unintentional.

\clearpage
\section{Alloy中的形式公理规范}
% \section{Formal Axiomatic Specification in Alloy}
\label{sec:alloy}

\lstdefinelanguage{alloy}{
  morekeywords={abstract, sig, extends, pred, fun, fact, no, set, one, lone, let, not, all, iden, some, run, for},
  morecomment=[l]{//},
  morecomment=[s]{/*}{*/},
  commentstyle=\color{green!40!black},
  keywordstyle=\color{blue!40!black},
  moredelim=**[is][\color{red}]{@}{@},
  escapeinside={!}{!},
}
\lstset{language=alloy}
\lstset{aboveskip=0pt}
\lstset{belowskip=0pt}

我们在Alloy(\url{http://alloy.mit.edu})中呈现了RVWMO内存模型的一种形式规范。
这个模型可以从\url{https://github.com/daniellustig/riscv-memory-model}在线获得。
% We present a formal specification of the RVWMO memory model in Alloy (\url{http://alloy.mit.edu}).
% This model is available online at \url{https://github.com/daniellustig/riscv-memory-model}.

该线上材料也包含了一些石蕊测试和一些关于Alloy可以怎样被用于对第~\ref{sec:memory:porting}节中的一些映射进行模型检查的例子。
% The online material also contains some litmus tests and some examples of how Alloy can be used to model check some of the mappings in Section~\ref{sec:memory:porting}.

\begin{figure}[h!]
  {
  \tt\bfseries\centering\footnotesize
  \begin{lstlisting}
////////////////////////////////////////////////////////////////////////////////
// =RVWMO PPO=

// 保留的程序次序
fun ppo : Event->Event {
  // same-address ordering
  po_loc :> Store
  + rdw
  + (AMO + StoreConditional) <: rfi

  // 显式同步
  + ppo_fence
  + Acquire <: ^po :> MemoryEvent
  + MemoryEvent <: ^po :> Release
  + RCsc <: ^po :> RCsc
  + pair

  // 句法依赖
  + addrdep
  + datadep
  + ctrldep :> Store

  // 管道依赖
  + (addrdep+datadep).rfi
  + addrdep.^po :> Store
}

// 全局内存次序尊重保留的程序次序
fact { ppo in ^gmo }
\end{lstlisting}}
  \caption{Alloy中的形式化RVWMO内存模型(1/5:PPO)  
  % The RVWMO memory model formalized in Alloy (1/5: PPO)
  }
  \label{fig:alloy1}
\end{figure}
\begin{figure}[h!]
  {
  \tt\bfseries\centering\footnotesize
  \begin{lstlisting}
////////////////////////////////////////////////////////////////////////////////
// =RVWMO 公理=

// 加载值公理
fun candidates[r: MemoryEvent] : set MemoryEvent {
  (r.~^gmo & Store & same_addr[r]) // 在gmo中写前趋r
  + (r.^~po & Store & same_addr[r]) // 在po中写前趋r
}

fun latest_among[s: set Event] : Event { s - s.~^gmo }

pred LoadValue {
  all w: Store | all r: Load |
    w->r in rf <=> w = latest_among[candidates[r]]
}

// 原子性公理
pred Atomicity {
  all r: Store.~pair |            // 从lr开始,
    no x: Store & same_addr[r] |  // 对于相同的地址,没有存储x
      x not in same_hart[r]       // 使得x来自不同的硬件线程,
      and x in r.~rf.^gmo         // 在gmo中x跟着(存储r的“读从”),
      and r.pair in x.^gmo        // 以及在gmo中,r跟着x
}

// 进程公理被隐去:Alloy只考虑有限的执行

pred RISCV_mm { LoadValue and Atomicity /* and Progress */ }

\end{lstlisting}}
  \caption{Alloy中的形式化RVWMO内存模型(2/5:公理)
    % The RVWMO memory model formalized in Alloy (2/5: Axioms)
    }
  \label{fig:alloy2}
\end{figure}
\begin{figure}[h!]
  {
  \tt\bfseries\centering\footnotesize
  \begin{lstlisting}
////////////////////////////////////////////////////////////////////////////////
// 内存的基础模型

sig Hart {  // 硬件线程
  start : one Event
}
sig Address {}
abstract sig Event {
  po: lone Event // 程序次序
}

abstract sig MemoryEvent extends Event {
  address: one Address,
  acquireRCpc: lone MemoryEvent,
  acquireRCsc: lone MemoryEvent,
  releaseRCpc: lone MemoryEvent,
  releaseRCsc: lone MemoryEvent,
  addrdep: set MemoryEvent,
  ctrldep: set Event,
  datadep: set MemoryEvent,
  gmo: set MemoryEvent,  // 全局内存次序
  rf: set MemoryEvent
}
sig LoadNormal extends MemoryEvent {} // l{b|h|w|d}
sig LoadReserve extends MemoryEvent { // lr
  pair: lone StoreConditional
}
sig StoreNormal extends MemoryEvent {}       // s{b|h|w|d}
// 模型中所有的StoreConditionals 都假定是成功的
sig StoreConditional extends MemoryEvent {}  // sc
sig AMO extends MemoryEvent {}               // amo
sig NOP extends Event {}

fun Load : Event { LoadNormal + LoadReserve + AMO }
fun Store : Event { StoreNormal + StoreConditional + AMO }

sig Fence extends Event {
  pr: lone Fence, // 操作位
  pw: lone Fence, // 操作位
  sr: lone Fence, // 操作位
  sw: lone Fence  // 操作位
}
sig FenceTSO extends Fence {}

/* Alloy编码细节:操作码位要么被设置(被编码,例如 as f.pr in iden),
 * 要么不被设置(f.pr not in iden)。这些位不能被用作其它任何用途 */
fact { pr + pw + sr + sw in iden }
// 对排序注释也是类似的
fact { acquireRCpc + acquireRCsc + releaseRCpc + releaseRCsc in iden }
// 不要试图通过pr/pw/sr/sw来编码FenceTSO;只把它当做is那样使用
fact { no FenceTSO.(pr + pw + sr + sw) }
\end{lstlisting}}
  \caption{Alloy中的形式化RVWMO内存模型(3/5:内存的模型)}
  \label{fig:alloy3}
\end{figure}

\begin{figure}[h!]
  {
  \tt\bfseries\centering\footnotesize
  \begin{lstlisting}
////////////////////////////////////////////////////////////////////////////////
// =基本模型规则 =

// 次序注释组
fun Acquire : MemoryEvent { MemoryEvent.acquireRCpc + MemoryEvent.acquireRCsc }
fun Release : MemoryEvent { MemoryEvent.releaseRCpc + MemoryEvent.releaseRCsc }
fun RCpc : MemoryEvent { MemoryEvent.acquireRCpc + MemoryEvent.releaseRCpc }
fun RCsc : MemoryEvent { MemoryEvent.acquireRCsc + MemoryEvent.releaseRCsc }

// 没有像store - acquire或者load - release那样的东西,除非两者结合
fact { Load & Release in Acquire }
fact { Store & Acquire in Release }

// FENCE PPO
fun FencePRSR : Fence { Fence.(pr & sr) }
fun FencePRSW : Fence { Fence.(pr & sw) }
fun FencePWSR : Fence { Fence.(pw & sr) }
fun FencePWSW : Fence { Fence.(pw & sw) }

fun ppo_fence : MemoryEvent->MemoryEvent {
    (Load  <: ^po :> FencePRSR).(^po :> Load)
  + (Load  <: ^po :> FencePRSW).(^po :> Store)
  + (Store <: ^po :> FencePWSR).(^po :> Load)
  + (Store <: ^po :> FencePWSW).(^po :> Store)
  + (Load  <: ^po :> FenceTSO) .(^po :> MemoryEvent)
  + (Store <: ^po :> FenceTSO) .(^po :> Store)
}

// 辅助定义
fun po_loc : Event->Event { ^po & address.~address }
fun same_hart[e: Event] : set Event { e + e.^~po + e.^po }
fun same_addr[e: Event] : set Event { e.address.~address }

// 初始化存储
fun NonInit : set Event { Hart.start.*po }
fun Init : set Event { Event - NonInit }
fact { Init in StoreNormal }
fact { Init->(MemoryEvent & NonInit) in ^gmo }
fact { all e: NonInit | one e.*~po.~start }  // 各事件都确实地在one硬件线程中
fact { all a: Address | one Init & a.~address } // one初始化各地址的存储
fact { no Init <: po and no po :> Init }
\end{lstlisting}}
  \caption{Alloy中的形式化RVWMO内存模型(4/5:基础模型规则)}
  \label{fig:alloy4}
\end{figure}

\begin{figure}[h!]
  {
  \tt\bfseries\centering\footnotesize
  \begin{lstlisting}
// po
fact { acyclic[po] }

// gmo
fact { total[^gmo, MemoryEvent] } // gmo是在所有MemoryEvents上的总次序

//rf
fact { rf.~rf in iden } // 每个读返回唯一写的值
fact { rf in Store <: address.~address :> Load }
fun rfi : MemoryEvent->MemoryEvent { rf & (*po + *~po) }

//dep
fact { no StoreNormal <: (addrdep + ctrldep + datadep) }
fact { addrdep + ctrldep + datadep + pair in ^po }
fact { datadep in datadep :> Store }
fact { ctrldep.*po in ctrldep }
fact { no pair & (^po :> (LoadReserve + StoreConditional)).^po }
fact { StoreConditional in LoadReserve.pair } // 假设所有的SC都成功了

// rdw
fun rdw : Event->Event {
  (Load <: po_loc :> Load)  // start with all same_address load-load pairs,
  - (~rf.rf)                // subtract pairs that read from the same store,
  - (po_loc.rfi)            // and subtract out "fri-rfi" patterns
}

// 过滤出冗余的实例和/或可视化效果
fact { no gmo & gmo.gmo } // 保持可视化效果整洁
fact { all a: Address | some a.~address }

////////////////////////////////////////////////////////////////////////////////
// =可选:操作码编码约束=

// 神圣的屏障列表
fact { Fence in
  Fence.pr.sr
  + Fence.pw.sw
  + Fence.pr.pw.sw
  + Fence.pr.sr.sw
  + FenceTSO
  + Fence.pr.pw.sr.sw
}

pred restrict_to_current_encodings {
  no (LoadNormal + StoreNormal) & (Acquire + Release)
}

////////////////////////////////////////////////////////////////////////////////
// =Alloy捷径 =
pred acyclic[rel: Event->Event] { no iden & ^rel }
pred total[rel: Event->Event, bag: Event] {
  all disj e, e': bag | e->e' in rel + ~rel
  acyclic[rel]
}
\end{lstlisting}}
  \caption{Alloy中的形式化RVWMO内存模型(5/5:辅助内容)
    % The RVWMO memory model formalized in Alloy (5/5: Auxiliaries)
    ]}
  \label{fig:alloy5}
\end{figure}



\clearpage
\input{memory-model-herd.tex}

\clearpage
\section{一个内存操作模型}
% \section{An Operational Memory Model}
\label{sec:operational}
这是操作风格中对RVWMO内存模型的一个备选的表示。
它旨在确实地承认与公理化表示相同的扩展行为:对于任何给定的程序,承认一个执行,当且仅当公理化表示也允许它。
% This is an alternative presentation of the RVWMO memory model in
% operational style.
%
% It aims to admit exactly the same extensional behavior as the
% axiomatic presentation: for any given program, admitting an execution
% if and only if the axiomatic presentation allows it.

公理化表示被定义为关于完整候选执行的谓词。相比之下,这种操作表示具有一种抽象的微架构风味:它被表示为一个状态机,
其中状态是对于硬件机器状态的一种抽象表示,并且带有显式的乱序执行和推测性执行(但是从更实现相关的微架构细节中抽象出来,
例如寄存器重命名、存储缓冲区、缓存层次、缓存协议,等等)。尽管如此,它可以提供有用的直觉。它也可以增量地构造执行,
使交互地和随机地探索更大样例的行为成为可能,同时公理化模型需要完整的候选执行,在此之上,公理可以得到检查。
% The axiomatic presentation is defined as a predicate on complete
% candidate executions.  In contrast, this operational presentation has
% an abstract microarchitectural flavor: it is expressed as a state
% machine, with states that are an abstract representation of hardware
% machine states, and with explicit out-of-order and speculative
% execution
% (but abstracting from more implementation-specific microarchitectural
% details such as register renaming, store buffers, cache hierarchies, cache protocols, etc.).
% As such, it can provide useful intuition.
% It can also
% construct executions incrementally, making it possible to
% interactively and randomly explore the behavior of larger examples,
% while the axiomatic model requires complete candidate executions
% over which the axioms can be checked.

操作表示覆盖了混合尺寸的执行,可能带有2的不同乘幂字节尺寸的重叠的内存访问。未对齐的访问被打断为单字节访问。
% The operational presentation covers mixed-size execution, with
% potentially overlapping memory accesses of different power-of-two byte
% sizes.  Misaligned accesses are broken up into single-byte accesses.

操作模型,与RISC-V ISA语义片段(RV64I和A)一起,被集成进{\tt rmem}探究工具中(url{https://github.com/rems-project/rmem})。
{\tt rmem}可以彻底地、伪随机地和交互地探究石蕊测试(见\ref{sec:litmustests})和小型ELF二进制文件。
在{\tt rmem}中,ISA语义使用Sail显示表达(关于Sail语言,见url{https://github.com/rems-project/sail},以及RISC-V ISA模型,
见url{https://github.com/rems-project/sail-riscv}),
而并发语义使用Lem表达(关于Lem语言,见url{https://github.com/rems-project/lem})。
% The operational model, together with a fragment of the RISC-V ISA
% semantics (RV64I and A), are integrated into the {\tt rmem} exploration
% tool (\url{https://github.com/rems-project/rmem}).  {\tt rmem} can
% explore litmus tests (see \ref{sec:litmustests}) and small ELF
% binaries exhaustively,
% pseudo-randomly and interactively.  In {\tt rmem}, the ISA semantics
% is expressed explicitly in Sail (see
% \url{https://github.com/rems-project/sail} for the Sail language, and
% \url{https://github.com/rems-project/sail-riscv} for the RISC-V ISA
% model), and the concurrency semantics is expressed in Lem (see
% \url{https://github.com/rems-project/lem} for the Lem language).

{\tt rmem}有一个命令行接口和一个网络接口。网络接口完全运行在客户端,
并且与一个石蕊测试库一起在线提供:\url{http//www.cl.cam.ac.uk/~pes20/rmem}。
命令行接口比网络接口更快,尤其是在穷举模式中。
% {\tt rmem} has a command-line interface and a web-interface.
% The web-interface runs entirely on the client side, and is provided
% online together with a library of litmus tests:
% \url{http://www.cl.cam.ac.uk/~pes20/rmem}.  The command-line interface
% is faster than the web-interface, specially in exhaustive mode.

% A library of RISC-V litmus tests can be downloaded from
% \url{https://github.com/litmus-tests/litmus-tests-riscv}.
% This repository also provides instructions on how to run the litmus
% tests on RISC-V hardware and how to compare the results with the
% operational and axiomatic models.  The library is also available
% through the web-interface.


% TODO: compare with the herd and alloy versions

下面是关于模型状态和过渡的一个非正式的介绍。正式的模型描述在下一个小节开始。
% Below is an informal introduction of the model states and transitions.
% The description of the formal model starts in the next subsection.

术语:与公理化表示相对,这里每个内存操作都或者是一个加载,或者是一个存储。
因此,AMO产生了两种截然不同的内存操作,加载和存储。当与“指令”连起来使用时,术语“加载”和“存储”指代产生此种内存操作的指令。
因此,二者都包括AMO指令。术语“获取”指代一种带有acquire-RCpc或acquire-RCsc注释的指令(或者它的内存操作)。
术语“释放”指代一种带有release-RCpc或release-RCsc注释的指令(或者它的内存操作)。
% Terminology: In contrast to the axiomatic presentation, here every memory operation is either a load or a store.
% Hence, AMOs give rise to two distinct memory operations, a load and a store.
% When used in conjunction with ``instruction'', the terms ``load'' and ``store'' refer to instructions that give rise to such memory operations.
% As such, both include AMO instructions.
% The term ``acquire'' refers to an instruction (or its memory operation) with the acquire-RCpc or acquire-RCsc annotation.
% The term ``release'' refers to an instruction (or its memory operation) with the release-RCpc or release-RCsc annotation.

\paragraph{模型状态}
一个模型状态由一份共享内存和一组硬件线程状态组成。
% A model state consists of a shared memory and a tuple of hart states.
\begin{center}
\sffamily
\begin{tabular}{ccc}
\cline{1-1}\cline{3-3}
\multicolumn{1}{|c|}{Hart 0} & \bf \dots & \multicolumn{1}{|c|}{Hart $n$} \\
\cline{1-1}\cline{3-3}
$\big\uparrow$ $\big\downarrow$ & & $\big\uparrow$ $\big\downarrow$ \\
\hline
\multicolumn{3}{|c|}{Shared Memory} \\
\hline
\end{tabular}
\end{center}
共享内存状态记录所有的至今已经传播的内存存储操作,以它们传播的次序记录(这可以变得更加高效,但是为了简化表示,我们采用了这种方式)。
% The shared memory state records all the memory store operations that have propagated so far, in the order they propagated (this can be made more efficient, but for simplicity of the presentation we keep it this way).

每个硬件线程状态主要由一个指令实例树组成,其中一些已经\emph{完成}了,而另一些还没有完成。未完成的指令实例可能被\emph{重新启动},例如,如果它们依赖于一个乱序的或者推测的加载,结果该加载出现了错误。
% Each hart state consists principally of a tree of instruction instances, some of which have been \emph{finished}, and some of which have not.
% Non-finished instruction instances can be subject to \emph{restart}, e.g.~if they depend on an out-of-order or speculative load that turns out to be unsound.

在指令树中,条件分支和间接跳转指令可能多个后继。当这种指令完成时,任何没有被采用的备选路径都被丢弃。
% Conditional branch and indirect jump instructions may have multiple successors in the instruction tree.
% When such instruction is finished, any un-taken alternative paths are discarded.

指令树中的每个指令实例都有一个状态,它包括了内部指令语义(用于该指令的ISA伪代码)的执行状态。
模型在Sail中对内部指令语义使用了一种形式化描述。可以将一条指令的执行状态想成伪代码控制状态、伪代码调用栈和局部变量值的一种表示。
一个指令实例状态也包括了关于该实例的内存和寄存器足迹、它的寄存器读写、它的内存操作、它是否完成,等等的信息。
% Each instruction instance in the instruction tree has a state that
% includes an execution state of the intra-instruction semantics (the
% ISA pseudocode for this instruction).
% The model uses a formalization of the intra-instruction semantics in Sail.
% One can think of the execution state of an instruction as a representation of the pseudocode control state, pseudocode call stack, and local variable values.
% An instruction instance state also includes information about the instance's memory and register footprints, its register reads and writes, its memory operations, whether it is finished, etc.

\paragraph{模型过渡}
% \paragraph{Model transitions}
对于任何模型状态,模型都定义了允许的过渡集合,每个过渡都是一个到达新的抽象机器状态的单一原子步骤。
一条指令的执行往往将涉及许多过渡,而它们在操作模型执行中与来自其它指令引发的过渡进行交错。
每个过渡由一个单独的指令实例引发;它将改变那个实例的状态,并且它可能依赖于、或者改变它的余下的硬件线程状态和共享内存状态,
但是它不会依赖于其它硬件线程的状态,并且它也不会去改变它们:下面引入的过渡定义在~\ref{sec:omm:transitions}节中,
每个过渡都带有一个先决条件和一个过渡后模型状态的构造。
% The model defines, for any model state, the set of allowed transitions, each of which is a single atomic step to a new abstract machine state.
% Execution of a single instruction will typically involve many
% transitions, and they may be interleaved in operational-model
% execution with transitions arising from other instructions. 
% Each transition arises from a single instruction instance; it will
% change the state of that instance, and it may depend on or change the
% rest of its hart state and the shared memory state, but it does not depend on other hart states, and it will not change them.
% The transitions are introduced below and defined in Section~\ref{sec:omm:transitions}, with a precondition and a construction of the post-transition model state for each.

\noindent 用于所有指令的过渡:
\begin{itemize}
\item \nameref{omm:fetch}: 这个过渡代表了一个新指令实例的获取和解码,作为一个先前的获取指令实例(或者初始获取地址)的一个程序次序后继。
% This transition represents a fetch and
%   decode of a new instruction instance, as a program order successor
%   of a previously fetched instruction instance (or the initial fetch
%   address).

模型假定指令内存是固定的;它不描述自我修改的代码的行为。
特别地,\nameref{omm:fetch}过渡不会生成内存加载操作,而在过渡中也不会涉及共享内存。
反而,模型依赖于在给定一个内存位置时提供操作码的外部指示。
% The model assumes the instruction memory is fixed; it does not
% describe the behavior of self-modifying code. 
% In particular, the \nameref{omm:fetch} transition does not generate memory load operations, and the shared memory is not involved in the transition.
% Instead, the model depends on an external oracle that provides an opcode when given a memory location.
%


\item[$\circ$] \nameref{omm:reg_write}: 这是对一个寄存器值的写。
% This is a write of a register value.
\item[$\circ$] \nameref{omm:reg_read}: 这是对一个寄存器值的写,该值来自写该寄存器的最近的程序次序前趋指令实例。
% This is a read of a register
  % value from the most recent program-order-predecessor instruction instance that writes to that register.
\item[$\circ$] \nameref{omm:sail_interp}: 这覆盖了伪代码内部运算:算数、函数调用,等等。
% This covers pseudocode internal computation: arithmetic, function calls, etc.
\item[$\circ$] \nameref{omm:finish}: 在指令伪代码完成的这一时刻,指令不能被重启,内存访问不能被丢弃,而所有的内存效果已经发生。
对于条件分支和间接跳转指令,任何从未被写到{\em pc}寄存器的地址获取的程序次序后继,与它们之后的指令实例的子树一起,都被丢弃,
% At this point the instruction pseudocode is done, the instruction cannot be restarted, memory accesses cannot be discarded, and all memory effects have taken place.
% For conditional branch and indirect jump instructions, any program order successors that were fetched from an address that is not the one that was written to the {\em pc} register are discarded, together with the sub-tree of instruction instances below them.
\end{itemize}

\noindent 专用于加载指令的过渡:
\begin{itemize}
\item[$\circ$] \nameref{omm:initiate_load}: 在这一时刻,加载指令的内存足迹是暂时已知的(它可能发生改变,如果较早的指令被重启的话),
  并且它的单个内存加载指令被满足而可以开始。
  % At this point the memory
  % footprint of the load instruction is provisionally known (it could change if
  % earlier instructions are restarted) and its individual memory load operations can start being satisfied.
\item \nameref{omm:sat_by_forwarding}: 通过转发,这会从程序次序中之前的内存存储操作,部分地或完全地满足单个内存加载操作。
% This partially or entirely
%   satisfies a single memory load operation by forwarding, from
%   program-order-previous memory store operations.
\item \nameref{omm:sat_from_mem}: 这会从内存完全地满足单个内存加载操作的未完成部分。
% This entirely satisfies the outstanding slices of a single memory load operation, from memory.
\item[$\circ$] \nameref{omm:complete_loads}: 在这一时刻,指令的所有的内存加载操作都已经被完全地满足,且指令的伪代码可以继续执行。
  一个加载指令可以被重启,直到\nameref{omm:finish}过渡。但是,在某些条件下,模型可能把一个加载指令视为不可重启的,即使是在它结束之前(例如,见\nameref{omm:prop_store})。
% At this point all the memory load operations of the instruction have been entirely satisfied and the instruction pseudocode can continue executing.
% A load instruction can be subject to being restarted until the \nameref{omm:finish} transition.
% But, under some conditions, the model might treat a load instruction as non-restartable even before it is finished (e.g.~see \nameref{omm:prop_store}).
\end{itemize}

\noindent 专用于存储指令的过渡:
\begin{itemize}
\item[$\circ$] \nameref{omm:initiate_store_footprint}: 在这一时刻,存储的内存足迹是暂时已知的。
% At this point the memory footprint of the store is provisionally known.
\item[$\circ$] \nameref{omm:instantiate_store_value}: 在这一时刻,内存存储操作具有它们的值,并且程序次序后继的内存加载操作可以通过从这些值转发而被满足。
% At this point the memory store operations have their values and program-order-successor memory load operations can be satisfied by forwarding from them.
\item[$\circ$] \nameref{omm:commit_stores}: 在这一时刻,存储指令被保证发生(指令不再会被重启或丢弃),而它们可以通过被传播到内存而开始。
% At this point the store operations are guaranteed to happen (the instruction can no longer be restarted or discarded), and they can start being propagated to memory.
\item \nameref{omm:prop_store}:这会把单个的内存存储操作传播到内存。
%  This propagates a single memory store operation to memory.
\item[$\circ$] \nameref{omm:complete_stores}: 在这一时刻,指令的所有的内存存储操作都已经被传播到内存,而指令的伪代码可以继续执行。
% At this point all the memory store operations of the instruction have been propagated to memory, and the instruction pseudocode can continue executing.
\end{itemize}

\noindent 专用于{\tt sc}指令的过渡:
\begin{itemize}
\item \nameref{omm:early_sc_fail}: 这会引起{\tt sc}失败,或者是一个自发的失败,或者是因为没有与程序次序之前的{\tt lr}配对。
% This causes the {\tt sc} to fail, either a spontaneous fail or because it is not paired with a program-order-previous {\tt lr}.
\item \nameref{omm:paired_sc}: 这个过渡表示{\tt sc}与一个{\tt lr}配对了,并且可能会成功。
% This transition indicates the {\tt sc} is paired with an {\tt lr} and might succeed.
\item \nameref{omm:commit_sc}: 这是对提交\nameref{omm:commit_stores}和\nameref{omm:prop_store}操作过渡的一个原子性的执行,它只有当存储来自于{\tt lr}读取的内容没有被覆写时才会被启用。
% This is an atomic execution of the transitions \nameref{omm:commit_stores} and \nameref{omm:prop_store}, it is enabled only if the stores from which the {\tt lr} read from have not been overwritten.
\item \nameref{omm:late_sc_fail}: 这会引起{\tt sc}失败,或者是一个自发的失败,或者是因为存储来自于{\tt lr}读取的内容已经被覆写。
% This causes the {\tt sc} to fail, either a spontaneous fail or because the stores from which the {\tt lr} read from have been overwritten.
\end{itemize}

\noindent 专用于AMO指令的过渡: 
\begin{itemize}
\item \nameref{omm:do_amo}: 这是一个所有需要去满足加载操作、执行必要的算术、和传播存储操作的过渡的原子性操作。
% This is an atomic execution of all the transitions needed to satisfy the load operation, do the required arithmetic, and propagate the store operation.
\end{itemize}

\noindent 专用于屏障指令的过渡: % Transitions specific to fence instructions:
\begin{itemize}
\item[$\circ$] \nameref{omm:commit_fence}
\end{itemize}

标记有~$\circ$标签的过渡,只要它们的先决条件被满足,那么总是可以立即执行,而不需要排除其它的行为;而$\bullet$不行。
尽管\nameref{omm:fetch}指令被标记为$\bullet$,但是只要它不是无限地多次执行,它就可以被立即执行。
% The transitions labeled~$\circ$ can always be taken eagerly, as soon as their precondition is satisfied, without excluding other behavior; the $\bullet$ cannot.
% Although \nameref{omm:fetch} is marked with a $\bullet$, it can be taken eagerly as long as it is not taken infinitely many times.

一个非AMO加载指令的实例,在被获取之后,往往将按这个次序经历如下的过渡:
% An instance of a non-AMO load instruction, after being fetched, will typically experience the following transitions in this order:
\begin{enumerate}
\item \nameref{omm:reg_read}
\item \nameref{omm:initiate_load}
\item \nameref{omm:sat_by_forwarding} 和/或 \nameref{omm:sat_from_mem} (与满足实例的所有加载操作的需求数目相同)
\item \nameref{omm:complete_loads}
\item \nameref{omm:reg_write}
\item \nameref{omm:finish}
\end{enumerate}
在上述过渡之前、之间和之后,可能出现任意数目的伪代码内部步骤过渡。此外,在下一个程序位置中的用于获取指令的Fetch指令过渡将一直可用,直到其被执行为止。
% Before, between and after the transitions above, any number of \nameref{omm:sail_interp} transitions may appear.
% In addition, a \nameref{omm:fetch} transition for fetching the instruction in the next program location will be available until it is taken.

这样结束了关于操作模型的非正式描述。接下来的章节描述了正式的操作模型。
% This concludes the informal description of the operational model.
% The following sections describe the formal operational model.

\subsection{指令内的伪码执行}\label{sec:omm:pseudocode_exec}
% \subsection{Intra-instruction Pseudocode Execution}\label{sec:omm:pseudocode_exec}
每个指令实例的指令内语义被表达为一个状态机,它本质上运行指令伪代码。
给定一个伪代码执行状态,它会计算下一个状态。大多数状态标识了一个由伪代码所请求的、挂起的内存或寄存器状态,这是内存模型必须做的。
这些状态是(这是一个带标签的联合;标签用小写字母表示):
% The intra-instruction semantics for each instruction instance is expressed as a state machine, essentially running the instruction pseudocode.
% Given a pseudocode execution state, it computes the next state.  Most
% states identify a pending memory or register operation, requested by
% the pseudocode, which the memory model has to do.  The
% states are (this is a tagged union; tags in small-caps):

\begin{center}
\begin{tabular}{l@{ \quad-\quad }l}
{\sc Load\_mem}({\it kind}, {\it address}, {\it size}, {\it load\_continuation})
    & 内存加载操作  \\
{\sc Early\_sc\_fail}({\it res\_continuation})
    & 允许 {\tt sc} 提前失败 \\
{\sc Store\_ea}({\it kind}, {\it address}, {\it size}, {\it next\_state})
    & 内存存储有效地址  \\
{\sc Store\_memv}({\it mem\_value}, {\it store\_continuation})
    & 内存存储值 \\
{\sc Fence}({\it kind}, {\it next\_state})
    & 屏障  \\
{\sc Read\_reg}({\it reg\_name}, {\it read\_continuation})
    & 寄存器读  \\
{\sc Write\_reg}({\it reg\_name}, {\it reg\_value}, {\it next\_state})
    & 寄存器写  \\
{\sc Internal}({\it next\_state})
    & 伪代码内部步骤  \\
{\sc Done}
    & 伪代码的结束  \\
\end{tabular}
\end{center}
Here:
\begin{tightlist}
\item {\it mem\_value}和{\it reg\_value}是字节的列表;  %   {\it mem\_value} and {\it reg\_value} are lists of bytes;
\item {\it address}是一个XLEN位的整数;   %  {\it address} is an integer of XLEN bits;
\item 对于加载/存储,{\it kind}标识了它是否是{\tt lr/sc}、acquire-RCpc/release-RCpc、acquire-RCsc/release-RCsc、acquire-release-RCsc;  
%  for load/store, {\it kind} identifies whether it is {\tt lr/sc}, acquire-RCpc/release-RCpc, acquire-RCsc/release-RCsc, acquire-release-RCsc;
\item 对于屏障,{\it kind}标识了它是否是一个普通的屏障还是TSO屏障,以及(对于普通屏障)前趋和后继次序的位;
% for fence, {\it kind} identifies whether it is a normal or TSO, and (for normal fences) the predecessor and successor ordering bits;
\item {\it reg\_name}标识了一个寄存器和它的一部分(开始和结束位的索引);以及
% {\it reg\_name} identifies a register and a slice thereof (start and
%   end bit indices); and
\item continuation描述了指令实例将如何继续处理每个可能由周围内存模型所提供的值({\it load\_continuation}和{\it read\_continuation}取得从内存加载的值和读取从先前寄存器写入的值,{\it store\_continuation}对于一个失败的sc取假,而在所有其它情形中取真,而如果{\tt sc}{\it 失败},{\it res\_continuation}取假,否则取{\it 真})。
% the continuations describe how the instruction instance will continue for each value that might be provided by the surrounding memory model (the {\it load\_continuation} and {\it read\_continuation} take the value loaded from memory and read from the previous register write, the {\it store\_continuation} takes {\it false} for an {\tt sc} that failed and {\it true} in all other cases, and {\it res\_continuation} takes {\it false} if the {\tt sc} fails and {\it true} otherwise).
\end{tightlist}

\begin{commentary}
  例如,给定加载指令\verb!lw x1,0(x2)!,一次执行往往将如下运行。初始执行状态将从给定的操作码的伪代码计算。
  这可能预计是{\sc Read\_reg}({\tt x2}, {\it read\_continuation})。
  把寄存器{\tt x2}最近写入的值(如果有必要,指令语义将被阻塞,直到寄存器的值可用),也就是{\tt 0x4000},输入到{\it read\_continuation},
  返回{\sc Load\_mem}({\tt plain\_load}, {\tt 0x4000}, {\tt 4}, {\it load\_continuation})。把从内存地址{\tt 0x4000}加载的{\tt 4}字节的值,也就是{\tt 0x42},
  输入到{\it load\_continuation},返回{\sc Write\_reg}({\tt x1}, {\tt 0x42}, {\sc Done})。在上述状态之间可能出现许多{\sc Internal}({\it next\_state})状态。
% For example, given the load instruction \verb!lw x1,0(x2)!,
% an execution will typically go as follows.
% The initial execution state will be computed from the pseudocode for the given opcode.
% This can be expected to be {\sc Read\_reg}({\tt x2}, {\it read\_continuation}).
% Feeding the most recently written value of register {\tt x2} (the instruction semantics will be blocked if necessary until the register value is available), say {\tt 0x4000}, to {\it read\_continuation} returns {\sc Load\_mem}({\tt plain\_load}, {\tt 0x4000}, {\tt 4}, {\it load\_continuation}).
% Feeding the 4-byte value loaded from memory location {\tt 0x4000}, say {\tt 0x42}, to {\it load\_continuation} returns
% {\sc Write\_reg}({\tt x1}, {\tt 0x42}, {\sc Done}).
% Many {\sc Internal}({\it next\_state}) states may appear before and between the states above.
\end{commentary}

注意,写内存被分为两个步骤,{\sc Store\_ea}和{\sc Store\_memv}:第一个步骤制造存储暂时已知的内存足迹,而第二个步骤添加要被存储的值。我们确保这些在伪代码中是成对的({\sc Store\_ea}后面跟着{\sc Store\_memv}),但是在它们之间可能会有其它的步骤。
% Notice that writing to memory is split into two steps, {\sc Store\_ea} and {\sc Store\_memv}: the first one makes the memory footprint of the store provisionally known, and the second one adds the value to be stored.
% We ensure these are paired in the pseudocode ({\sc Store\_ea} followed by {\sc Store\_memv}), but there may be other steps between them.
\begin{commentary}
  可以观察到,{\sc Store\_ea}可以发生在要存储的值被决定之前。例如,对于操作模型所允许的石蕊测试LB+fence.r.rw+data-po(正如RVWMO所允许的那样),硬件线程1中的第一个存储必须在它的值被决定之前就采取{\sc Store\_ea}步骤,以便第二个存储可以看到它是一个非重叠的内存足迹,以允许第二个存储被乱序提交而不违反一致性。
% It is observable that the {\sc Store\_ea} can occur before the value to be stored is determined.
% For example, for the litmus test LB+fence.r.rw+data-po to be allowed by the operational model (as it is by RVWMO), the first store in Hart 1 has to take the {\sc Store\_ea} step before its value is determined, so that the second store can see it is to a non-overlapping memory footprint, allowing the second store to be committed out of order without violating coherence.
\end{commentary}

除了明确执行一次加载和一次存储的AMO,每个指令的伪代码最多执行一次存储或一次加载。那些内存访问然后通过硬件线程语义被分割为架构上的原子性单元(见下面的\nameref{omm:initiate_load}和\nameref{omm:initiate_store_footprint})。
% The pseudocode of each instruction performs at most one store or one load, except for AMOs that perform exactly one load and one store.
% Those memory accesses are then split apart into the architecturally atomic units by the hart semantics (see \nameref{omm:initiate_load} and \nameref{omm:initiate_store_footprint} below).

非正式地,一个寄存器读的每一位都应当由来自最近的(以程序次序)可以写该位的指令实例的一个寄存器写(或者来自硬件线程的初始寄存器状态,如果没有这样的写的话)来满足。因此,有必要知道每个指令实例的寄存器写的足迹,我们在指令实例被创建时(见下面的\nameref{omm:fetch}的动作)计算这个足迹。我们确保在伪代码中,每个指令对每个寄存器位最多执行一次寄存器写,并且也保证不会尝试读取它刚刚写入的寄存器的值。
% Informally, each bit of a register read should be satisfied from a register write by the most recent (in program order) instruction instance that can write that bit (or from the hart's initial register state if there is no such write).
% Hence, it is essential to know the register write footprint of each instruction instance, which we calculate when the instruction instance is created (see the action of \nameref{omm:fetch} below).
% We ensure in the pseudocode that each instruction does at most one register write to each register bit, and also that it does not try to read a register value it just wrote.

每个寄存器读必须等待合适的寄存器写被执行(正如上面描述的那样),从这个事实浮现了模型中的数据流依赖(地址和数据)。
% Data-flow dependencies (address and data) in the model emerge from the
% fact that each register read has to wait for the appropriate register write to be executed (as described above).

\subsection{指令实例状态}\label{sec:omm:inst_state}
% \subsection{Instruction Instance State}\label{sec:omm:inst_state}
每个指令实例$i$都拥有一个状态,包括:
% Each instruction instance $i$ has a state comprising:
\begin{itemize}
\item {\it program\_loc},指令被从此获取的内存地址; % , the memory address from which the instruction was fetched;
\item {\it instruction\_kind},识别这是否是一个加载、存储、AMO、屏障、分支/跳转,或者一个‘简单的’指令(这也包括了一种类似于描述伪指令执行状态的类型); % identifying whether this is a load, store, AMO, fence, branch/jump or a `simple' instruction (this also includes a {\it kind} similar to the one described for the pseudocode execution states);
\item {\it src\_regs},源{\it reg\_name}的集合(包括系统寄存器),由指令的伪代码静态决定; %  , the set of source {\it reg\_name}s (including system registers), as statically determined from the pseudocode of the instruction;
\item {\it dst\_regs},目的{\it reg\_name}(包括系统寄存器),由指令的伪代码静态决定; % , the destination {\it reg\_name}s (including system registers), as statically determined from the pseudocode of the instruction;
\item {\it pseudocode\_state}(或者有时只简写为‘state’),如下之一(这是一个带标签的联合;标签用小写字母表示):  %  (or sometimes just `state' for short), one of (this is a tagged union; tags in small-caps):
  \begin{center}
  \begin{tabular}{l@{ \quad-\quad }l}
  {\sc Plain}({\it isa\_state})
    & 准备制造一个伪代码过渡 \\
  {\sc Pending\_mem\_loads}({\it load\_continuation})
    & 请求内存加载操作 \\
  {\sc Pending\_mem\_stores}({\it store\_continuation})
    & 请求内存存储操作 \\
%   {\sc Pending\_exception}({\it exception}) & performing an exception;
  \end{tabular}
  \end{center}

\item {\it reg\_reads},寄存器读实例已经被执行,对于每个实例,包括寄存器写它从中读取的片段; % , the register reads the instance has performed, including, for each one, the register write slices it read from;
\item {\it reg\_writes},寄存器写实例已经被执行; % , the register writes the instance has performed;
\item {\it mem\_loads},一组内存加载操作的集合,并且对于每个加载操作,是尚未满足的片段(还没有被满足的字节的索引),而对于已满足的片段,则是满足它的存储片段(每个片段由一个内存存储操作和它的字节索引的子集组成)。  % , a set of memory load operations, and for each one
  % the as-yet-unsatisfied slices (the byte indices that have not been
  % satisfied yet), and, for the satisfied slices, the store slices
  % (each consisting of a memory store operation and subset of its byte indices) that satisfied it.
% ------   下面两条word里没有,可能是tex新版本刚加入的内容 ----------
% \item {\it mem\_stores}, a set of memory store operations, and for each one a flag that indicates whether it has been propagated (passed to the shared memory) or not.
% \item information recording whether the instance is committed, finished, etc.
\end{itemize}

每个内存加载操作包括了一个内存足迹(地址和尺寸)。每个内存存储操作包括了一个内存足迹和(当值可用时)一个值。
% Each memory load operation includes a memory footprint (address and size).
% Each memory store operations includes a memory footprint, and, when available, a value.

一个带有非空{\it mem\_loads}的加载指令实例,如果所有的加载操作都被满足(换句话说,没有未满足的加载片段),那么被称之为被完全满足的。
% A load instruction instance with a non-empty {\it mem\_loads}, for which all the load operations are satisfied (i.e.~there are no unsatisfied load slices) is said to be {\it entirely satisfied}.

非正式地,一个指令实例被称为具有{\it 完全决定的}数据,如果为它的源寄存器提供输入的加载(和{\tt sc})指令被完成了。
类似地,它被称为具有完全决定的内存足迹,如果为它的内存操作地址寄存器提供输入的加载(和{\tt sc})指令被完成了。
正式地讲,我们首先定义了{\it 完全决定的寄存器写}的概念:一个来自指令实例$i$的{\it reg\_writes}的$w$被称之为{\it 完全决定的},如果满足了下列条件之一:
% Informally, an instruction instance is said to have {\it fully determined data} if the load (and {\tt sc}) instructions feeding its source registers are finished.
% Similarly, it is said to have a {\it fully determined memory footprint} if the load (and {\tt sc}) instructions feeding its memory operation address register are finished.
%
% Formally, we first define the notion of {\it fully determined register write}: a register write $w$ from {\it reg\_writes} of instruction instance $i$ is said to be {\it fully determined} if one of the following conditions hold:
\begin{enumerate}
\item i被完成了;或者 %  $i$ is finished; or
\item w所写的值不会被i制造的内存操作所影响(即,一个从内存加载的值或者sc的结果),并且,对于i已经制造的每个影响w的寄存器读,源于i读取的寄存器写都是完全决定的(或者i从初始寄存器状态读取)。 
% the value written by $w$ is not affected by a memory operation that $i$ has made (i.e. a value loaded from memory or the result of {\tt sc}), and, for every register read that $i$ has made, that affects $w$, the register write from which $i$ read is fully determined (or $i$ read from the initial register state).
\end{enumerate}
现在,一个指令实例$i$被称为具有{\it 完全决定的}数据,如果对于每个来自{\it reg\_reads}的寄存器读$r$,$r$从中读取的寄存器写都是完全决定的。
一个指令实例$i$被称为{\it 具有完全决定的内存足迹},如果对于来自{\it reg\_reads}的每个输入到$i$的内存操作地址的寄存器读$r$,$r$从中读取的寄存器写都是完全决定的。
% Now, an instruction instance $i$ is said to have  {\it fully determined data} if for every register read $r$ from {\it reg\_reads}, the register writes that $r$ reads from are fully determined.
% An instruction instance $i$ is said to have a {\it fully determined memory footprint} if for every register read $r$ from {\it reg\_reads} that feeds into $i$'s memory operation address, the register writes that $r$ reads from are fully determined.
\begin{commentary}
  对于每次寄存器写, {\tt rmem}工具都会记录,在执行写的这一时刻,该指令已经读取的来自其它指令的寄存器写的集合。通过小心地安排由工具覆盖的指令的伪代码,我们能够做到这点,使得这确实是这次写所依赖的寄存器写的集合。
% The {\tt rmem} tool records, for every register write, the set of register writes from other instructions that have been read by this instruction at the point of performing the write.
% By carefully arranging the pseudocode of the instructions covered by the tool we were able to make it so that this is exactly the set of register writes on which the write depends on.
\end{commentary}


\subsection{硬件线程状态}
% \subsection{Hart State}
单个硬件线程的模型状态包括:
% The model state of a single hart comprises:
\begin{itemize}
\item {\it hart\_id},一个关于硬件线程的唯一的标识符; % , a unique identifier of the hart;
%\item {\it register\_data}, the name, bit width, and start bit index for each register;
\item {\it initial\_register\_state},各寄存器的初始寄存器状态;  % , the initial register value for each register;
\item {\it initial\_fetch\_address},初始指令获取地址; % , the initial instruction fetch address;
\item {\it instruction\_tree},以程序次序,一个已经被获取到的(并且没有被丢弃的)指令实例的树。 % , a tree of the instruction instances that have been fetched (and not discarded), in program order.
\end{itemize}


\subsection{共享内存状态}
% \subsection{Shared Memory State}
共享内存的模型状态包括一个内存存储操作的列表,按它们传播到共享内存的次序排序。
% The model state of the shared memory comprises a list of memory store operations, in the order they propagated to the shared memory.

当一个存储操作被传播到共享内存时,它被简单地添加到列表的末端。当一个加载操作从内存被满足时,对于加载操作的每一个字节,将返回最近对应的存储片段。
% When a store operation is propagated to the shared memory it is simply added to the end of the list.
% When a load operation is satisfied from memory, for each byte of the load operation, the most recent corresponding store slice is returned.

\begin{commentary}
  对于大多数目的,将共享内存想象为一个数组会更简单,即,一个从内存位置到内存存储操作片段的映射,这里每个内存位置被映射到一个最近的针对该位置的内存存储操作的1字节的片段。
  然而,对于适当地处理{\tt sc}指令,这种抽象不够详细。RVWMO\nameref{rvwmo:ax:atom}允许来自与{\tt sc}相同的硬件线程的存储操作,以在{\tt sc}的存储操作和与之配对的{\tt lr}将读取自的存储操作之间进行干预。
  为了允许这种存储操作进行干预,而禁止其它的,数组抽象必须被扩展以记录更多的信息。这里,我们使用一个列表,因为它很简单,但是更有效和可扩展的实现应当更可能使用某些更好的东西。
% For most purposes, it is simpler to think of the shared memory as an array, i.e., a map from memory locations to memory store operation slices, where each memory location is mapped to a one-byte slice of the most recent memory store operation to that location.
% However, this abstraction is not detailed enough to properly handle the {\tt sc} instruction.
% The RVWMO \nameref{rvwmo:ax:atom} allows store operations from the same hart as the {\tt sc} to intervene between the store operation of the {\tt sc} and the store operations the paired {\tt lr} read from.
% To allow such store operations to intervene, and forbid others, the array abstraction must be extended to record more information.
% Here, we use a list as it is very simple, but a more efficient and scalable implementations should probably use something better.
\end{commentary}


\subsection{过渡}\label{sec:omm:transitions}

下面的每个段落都描述了一种系统过渡。描述开始于一个基于当前系统状态之上的条件。
只有当条件被满足时,过渡可以发生在当前状态。条件后面跟着一个行动,它在过渡发生时应用到该状态,以生成新的系统状态。
% Each of the paragraphs below describes a single kind of system transition.
% The description starts with a condition over the current system state.
% The transition can be taken in the current state only if the condition is satisfied.
% The condition is followed by an action that is applied to that state when the transition is taken, in order to generate the new system state.

\paragraph{获取指令}\label{omm:fetch}
可以从地址{\it loc}获取指令实例$i$的一个可能的程序次序后继,如果:
% A possible program-order-successor of instruction instance $i$ can be fetched from address {\it loc} if:
\begin{enumerate}
\item 它还没有被获取到,即,在硬件线程的{\it 指令树}中,$i$没有直接的来自{\it loc}的后继;并且 % it has not already been fetched, i.e., none of the immediate successors of $i$ in the hart's {\it instruction\_tree} are from {\it loc}; and
\item 如果$i$的伪代码已经向{\em pc}写入了一个地址,那么{\it loc}必须是那个地址,否则{\it loc}就是:  % if $i$'s pseudocode has already written an address to {\em pc}, then {\it loc} must be that address, otherwise {\it loc} is:
  \begin{itemize}
  \item 对于一个条件分支,是后继地址或分支目标地址;  % for a conditional branch, the successor address or the branch target address;
  \item 对于一个(直接的)跳转和链接指令({\tt jal}),是目标地址;  % for a (direct) jump and link instruction ({\tt jal}), the target address;
  \item 对于一个间接跳转指令({\tt jalr}),是任何地址;以及  % for an indirect jump instruction ({\tt jalr}), any address; and
  \item 对于任何其它的指令,是$i.\textit{program\_loc}+4$。 %  for any other instruction, $i.\textit{program\_loc}+4$.
  \end{itemize}
\end{enumerate}
% \fixme{Does an instruction at the end of memory need special-case treatment?}

行动:在{\it loc}处的程序内存中为指令构建一个新的初始化的指令实例$i’$,它带有从指令伪代码计算的状态{\sc Plain}({\it isa\_state}),包括从伪代码可以获得的静态信息,例如它的{\it instruction\_kind},{\it src\_regs},和{\it dst\_regs},并把$i’$添加到硬件线程的指令树作为$i$的一个后继。
% Action: construct a freshly initialized instruction instance $i'$ for the instruction in the program memory at {\it loc}, with state {\sc Plain}({\it isa\_state}), computed from the instruction pseudocode, including the static information available from the pseudocode such as its {\it instruction\_kind}, {\it src\_regs}, and {\it dst\_regs}, and add $i'$ to the hart's {\it instruction\_tree} as a successor of $i$.

\begin{commentary}

  可能的下一个获取地址({\it loc})在获取$i$之后是立即可用的,并且模型不需要等待伪代码写到{\em pc};这允许乱序执行,并且推测过去的条件分支和跳转。对于大多数指令,这些地址很容易从指令的伪代码获得。
  唯一的例外是间接跳转指令({\tt jalr}),那里地址依赖于寄存器中持有的值。
  原则上,数学模型应当允许这里推测到任意的地址。在{\tt rmem}工具中的穷举搜索通过多次运行穷举搜索来处理这个问题,对于每个间接跳转,都带有不断增长的、可能的下一个获取地址的集合。
  最初的搜索使用空集合,因此在间接跳转指令之后没有获取,直到指令的伪代码写到{\em pc},然后我们使用该值来获取下一条指令。在开始穷举搜索的下一次迭代之前,我们为每个间接跳转(按代码位置分组)收集了它在先前搜索迭代中的所有执行中写到{\em pc}的值的集合,并使用它作为指令的可能的下一个获取的地址。
  当没有检测到新的获取地址时,这个过程终止。
% The possible next fetch addresses ({\it loc}) are available immediately after fetching $i$ and the model does not need to wait for the pseudocode to write to {\em pc}; this allows out-of-order execution, and speculation past conditional branches and jumps.
% For most instructions these addresses are easily obtained from the instruction pseudocode.
% The only exception to that is the indirect jump instruction ({\tt jalr}), where the address depends on the value held in a register.
% %
% In principle the mathematical model should allow speculation to
% arbitrary addresses here. 
% %
% The exhaustive search in the {\tt rmem} tool handles this by running the exhaustive search multiple times with a growing set of possible next fetch addresses for each indirect jump.
% The initial search uses empty sets, hence there is no fetch after indirect jump instruction until the pseudocode of the instruction writes to {\em pc}, and then we use that value for fetching the next instruction.
% Before starting the next iteration of exhaustive search, we collect for each indirect jump (grouped by code location) the set of values it wrote to {\em pc} in all the executions in the previous search iteration, and use that as possible next fetch addresses of the instruction.
% This process terminates when no new fetch addresses are detected.
\end{commentary}

\paragraph{初始化内存加载操作}\label{omm:initiate_load}
一个状态{\sc Plain}({\sc Load\_mem}({\it kind}, {\it address}, {\it size}, {\it load\_continuation}))中的指令实例$i$总是可以初始化对应的内存加载操作。
% An instruction instance $i$ in state {\sc Plain}({\sc Load\_mem}({\it kind}, {\it address}, {\it size}, {\it load\_continuation})) can always initiate the corresponding memory load operations.
行动:
% Action:
\begin{enumerate}
\item 构造合适的内存加载操作$mlos$:  % Construct the appropriate memory load operations $mlos$:
  \begin{itemize}
  \item 如果{\it address}被对齐到{\it size},那么$mlos$是一个从{\it address}开始的{\it size}字节的单个内存加载操作; %  if {\it address} is aligned to {\it size} then $mlos$ is a single memory load operation of {\it size} bytes from {\it address};
  \item 否则,$mlos$是一组数量为{\it size}的内存加载操作的集合,每个加载操作负责地址$\textit{address}\ldots\textit{address}+\textit{size}-1$中的一个字节。  % otherwise, $mlos$ is a set of {\it size} memory load operations, each of one byte, from the addresses $\textit{address}\ldots\textit{address}+\textit{size}-1$.
  \end{itemize}
\item 把$i$的{\it mem\_loads}设置到$mlos$;并且  % set {\it mem\_loads} of $i$ to $mlos$; and
\item 更新$i$的状态为{\sc Pending\_mem\_loads}({\it load\_continuation})。  % update the state of $i$ to {\sc Pending\_mem\_loads}({\it load\_continuation}).
\end{enumerate}

\begin{commentary}
  在~\ref{sec:rvwmo:primitives}节中说到,未对齐的内存访问可能在任意粒度被分解。这里我们把它们分解为一个字节的访问,因为这个粒度包含了所有其它的粒度。
% In Section~\ref{sec:rvwmo:primitives} it is said that misaligned memory accesses may be decomposed at any granularity.
% Here we decompose them to one-byte accesses as this granularity subsumes all others.
\end{commentary}

\paragraph{通过从未传播的状态转发来满足内存加载操作}\label{omm:sat_by_forwarding}
对于一个状态{\sc Pending\_mem\_loads}({\it load\_continuation})中的非AMO加载指令实例$i$,和$i.\textit{mem\_loads}$中的一个具有未满足片段的内存加载操作$mlo$,内存加载操作可以通过从未被程序次序之前的存储指令实例所传播的内存存储操作转发,而被部分地或完全地满足,如果:
% For a non-AMO load instruction instance $i$ in state {\sc Pending\_mem\_loads}({\it load\_continuation}), and a memory load operation $mlo$ in $i.\textit{mem\_loads}$ that has unsatisfied slices, the memory load operation can be partially or entirely satisfied by forwarding from unpropagated memory store operations by store instruction instances that are program-order-before $i$ if:
\begin{enumerate}
\item 所有的程序次序在先的设置了{\tt .sr}和{\tt .pw}的{\tt fence}指令都完成了;  % all program-order-previous {\tt fence} instructions with {\tt .sr} and {\tt .pw} set are finished;
\item 对于每个程序次序在先的设置了{\tt .sr}和{\tt .pr},但是没有设置{\tt .pw}的{\tt fence}指令,$f$,如果$f$没有完成,那么所有的程序次序在$f$之前的加载指令都被完全地满足了;  % for every program-order-previous {\tt fence} instruction, $f$, with {\tt .sr} and {\tt .pr} set, and {\tt .pw} not set, if $f$ is not finished then all load instructions that are program-order-before $f$ are entirely satisfied;
\item 对于每个没有完成的程序次序在先的{\tt fence.tso}指令,$f$,所有的程序次序先于$f$的加载指令都被完全地满足了;  % for every program-order-previous {\tt fence.tso} instruction, $f$, that is not finished, all load instructions that are program-order-before $f$ are entirely satisfied;
% \item all program-order-previous {\tt fence.i} instructions are finished;
\item 如果$i$是一个load-acquire-RCsc,所有的程序次序在先的store-relaase-RCsc都完成了;  % if $i$ is a load-acquire-RCsc, all program-order-previous store-releases-RCsc are finished;
\item 如果$i$是一个load-acquire-release,所有的程序次序在先的指令都完成了;  % if $i$ is a load-acquire-release, all program-order-previous instructions are finished;
\item 所有的未完成的程序次序在先的load-acquire指令都完全地被满足了;并且  % all non-finished program-order-previous load-acquire instructions are entirely satisfied; and
\item 所有的程序次序在先的store-acquire-release指令都被满足了;  % all program-order-previous store-acquire-release instructions are finished;
\end{enumerate}

令$msoss$是所有来自程序次序先于$i$的、并且已经计算出要存储的值的非{\tt sc}存储指令实例的未被转发的内存存储操作片段的集合,它们与$mlo$的未被满足的片段重叠,并且不会被干扰的存储操作、或从一个干扰的加载读取的存储操作所替代。最后一个条件要求,对于$msoss$中的每个来自指令$i’$的内存存储操作片段$msos$:
% Let $msoss$ be the set of all unpropagated memory store operation slices from non-{\tt sc} store instruction instances that are program-order-before $i$ and have already calculated the value to be stored, that overlap with the unsatisfied slices of $mlo$, and which are not superseded by intervening store operations or store operations that are read from by an intervening load.
% The last condition requires, for each memory store operation slice $msos$ in $msoss$ from instruction $i'$:
\begin{tightlist}
\item 没有程序次序在$i$和$i’$之间的存储指令带有与$msos$重叠的内存存储操作;并且 % that there is no store instruction program-order-between $i$ and $i'$ with a memory store operation overlapping $msos$; and
\item 没有程序次序在$i$和$i’$之间的加载指令被来自不同的硬件线程的重叠的内存存储操作片段所满足。  % that there is no load instruction program-order-between $i$ and $i'$ that was satisfied from an overlapping memory store operation slice from a different hart.
\end{tightlist}

行动:
% Action:
\begin{enumerate}
\item 更新$i.\textit{mem\_loads}$,以表示$mlo$被$msoss$满足了;并且
% update $i.\textit{mem\_loads}$ to indicate that $mlo$ was satisfied by $msoss$; and
\item 重启任何违背一致性的推测性指令,作为这个的结果,也就是说,对于每个未完成的作为i的程序次序后继的指令$i’$,和$i’$的每个由$msoss’$满足的内存加载操作$mlo’$,如果在$msoss’$中存在一个内存存储操作片段$msos’$,和一个来自$msoss$中一个不同的内存存储操作的重叠的内存存储操作片段,并且$msos’$并非来自$i$的一个程序次序后继指令,那么重新启动$i’$和它的{\em 重启依赖}。
% restart any speculative instructions which have violated coherence as a result of this, i.e., for every non-finished instruction $i'$ that is a program-order-successor of $i$, and every memory load operation $mlo'$ of $i'$ that was satisfied from $msoss'$, if there exists a memory store operation slice $msos'$ in $msoss'$, and an overlapping memory store operation slice from a different memory store operation in $msoss$, and $msos'$ is not from an instruction that is a program-order-successor of $i$, restart $i'$ and its {\em restart-dependents}.
\end{enumerate}
此处,指令$j$的{\em 重启依赖}是指:
% Where, the {\em restart-dependents} of instruction $j$ are:
\begin{tightlist}
\item $j$的程序次序后继,如果它有关于$j$的寄存器写的数据流依赖;  % program-order-successors of $j$ that have data-flow dependency on a register write of $j$;
\item $j$的程序次序后继,如果它有一个内存加载操作,其读取自j的一个内存存储操作(通过转发);  % program-order-successors of $j$ that have a memory load operation that reads from a memory store operation of $j$ (by forwarding);
\item 如果$j$是一个load-acquire,那么是$j$的所有的程序次序后继;  % if $j$ is a load-acquire, all the program-order-successors of $j$;
\item 如果$j$是一个加载,对于每个设置了{\tt .sr}和{\tt .pr},但是没有设置{\tt .pw}的{\tt fence},$f$,如果它是$j$的一个程序次序后继,那么是所有的是$f$的程序次序后继的加载指令;  %  if $j$ is a load, for every {\tt fence}, $f$, with {\tt .sr} and {\tt .pr} set, and {\tt .pw} not set, that is a program-order-successor of $j$, all the load instructions that are program-order-successors of $f$;
\item 如果$j$是一个加载,对于每个{\tt fence.tso},$f$,如果是$j$的一个程序次序后继,那么是所有的是$f$的程序次序后继的加载指令;以及  % if $j$ is a load, for every {\tt fence.tso}, $f$, that is a program-order-successor of $j$, all the load instructions that are program-order-successors of $f$;
% and 
\item(递归地)所有的上述指令实例的所有重启依赖。  %  (recursively) all the restart-dependents of all the instruction instances above.
\end{tightlist}

\begin{commentary}
  向一个内存加载转发内存存储操作可能只满足该加载的某些片段,而使其它片段仍是未满足的。
% Forwarding memory store operations to a memory load might satisfy only some slices of the load, leaving other slices unsatisfied.

一个程序次序在先的不可用的存储操作,当它变得可用时,在采用上面的过渡的时候,可能使得$msoss$暂时不可靠(违反一致性)。那样的存储将阻止加载的完成(见\nameref{omm:finish}),并将在存储操作被传播时,导致它重启(见\nameref{omm:prop_store})。
% A program-order-previous store operation that was not available when taking the transition above might make $msoss$ provisionally unsound (violating coherence) when it becomes available.
% That store will prevent the load from being finished (see \nameref{omm:finish}), and will cause it to restart when that store operation is propagated (see \nameref{omm:prop_store}).

上述过渡条件的结果就是,store-release-RCsc内存存储操作不能被转发到load-acquire-RCsc指令:$msoss$不包括来自已完成存储的内存存储操作(因为那些必定是已传播的内存存储操作),而且当加载是acquire-RCsc的时候,上述条件需要所有的程序次序在先的store-release-RCsc都被完成。
% A consequence of the transition condition above is that store-release-RCsc memory store operations cannot be forwarded to load-acquire-RCsc instructions:
% $msoss$ does not include memory store operations from finished stores (as those must be propagated memory store operations), and the condition above requires all program-order-previous store-releases-RCsc to be finished when the load is acquire-RCsc.
\end{commentary}


\paragraph{从内存满足内存加载操作}\label{omm:sat_from_mem}
对于一个非AMO加载指令或者一个“AMO的满足、提交和传播操作”过渡上下文中的AMO指令的指令实例$i$,任何$i.\textit{mem\_loads}$中的有未满足片段的内存加载操作mlo,可以从内存被满足,如果通过未传播的存储转发来满足内存加载操作的所有条件都被满足了的话。
行动:令$msoss$是来自覆盖了$mlo$的未满足片段的内存的内存存储操作片段,并应用通过未传播的存储转发满足内存加载操作的行为。
% For an instruction instance $i$ of a non-AMO load instruction or an AMO instruction in the context of the ``\nameref{omm:do_amo}'' transition, any memory load operation $mlo$ in $i.\textit{mem\_loads}$ that has unsatisfied slices, can be satisfied from memory if all the conditions of \nameref{omm:sat_by_forwarding} are satisfied.
% Action: let $msoss$ be the memory store operation slices from memory covering the unsatisfied slices of $mlo$, and apply the action of \nameref{omm:sat_by_forwarding}.

\begin{commentary}
  注意通过未传播的存储转发来满足内存加载操作可能会使内存加载操作的某些片段无法满足,那些片段将不得不通过再次采取过渡来满足,或者采取从内存满足内存加载操作来满足。另一方面,从内存满足内存加载操作将总是满足内存加载操作的所有的未满足片段。
% Note that \nameref{omm:sat_by_forwarding} might leave some slices of the memory load operation unsatisfied, those will have to be satisfied by taking the transition again, or taking \nameref{omm:sat_from_mem}.
% \nameref{omm:sat_from_mem}, on the other hand, will always satisfy all the unsatisfied slices of the memory load operation.
\end{commentary}


\paragraph{完整加载操作}\label{omm:complete_loads}
状态{\sc Pending\_mem\_loads}({\it load\_continuation})中的一个加载指令实例$i$可以是完整的(不要与完成相混淆),如果所有的内存加载操作$i.\textit{mem\_loads}$被完全地满足了(即,没有未满足的片段)。
行动:将i的状态更新为{\sc Plain}({\it load\_continuation(mem\_value)}),这里{\it mem\_value}是从所有满足$i.\textit{mem\_loads}$的内存存储操作片段集合而成的。
% A load instruction instance $i$ in state {\sc Pending\_mem\_loads}({\it load\_continuation}) can be completed (not to be confused with finished) if all the memory load operations $i.\textit{mem\_loads}$ are entirely satisfied (i.e.~there are no unsatisfied slices).
% Action: update the state of $i$ to {\sc Plain}({\it load\_continuation(mem\_value)}), where {\it mem\_value} is assembled from all the memory store operation slices that satisfied $i.\textit{mem\_loads}$.


\paragraph{早期{\tt sc}失败}\label{omm:early_sc_fail}
状态{\sc Plain}({\sc Early\_sc\_fail}({\it res\_continuation}))中的一个{\tt sc}指令实例$i$可以总是被造成失败。行动:把$i$的状态更新到{\sc Plain}({\it res\_continuation(false)})。
% An {\tt sc} instruction instance $i$ in state {\sc Plain}({\sc Early\_sc\_fail}({\it res\_continuation})) can always be made to fail.
% Action: update the state of $i$ to {\sc Plain}({\it res\_continuation(false)}).


\paragraph{配对的{\tt sc} }\label{omm:paired_sc}
状态{\sc Plain}({\sc Early\_sc\_fail}({\it res\_continuation}))中的一个{\tt sc}指令实例$i$可以继续它的(可能成功的)执行,如果$i$与一个{\tt lr}配对的话。行动:把$i$的状态更新到{\sc Plain}({\it res\_continuation(true)})。
% An {\tt sc} instruction instance $i$ in state {\sc Plain}({\sc Early\_sc\_fail}({\it res\_continuation})) can continue its (potentially successful) execution if $i$ is paired with an {\tt lr}.
% Action: update the state of $i$ to {\sc Plain}({\it res\_continuation(true)}).


\paragraph{初始化内存存储操作足迹}\label{omm:initiate_store_footprint}
状态{\sc Plain}({\sc Store\_ea}({\it kind}, {\it address}, {\it size}, {\it next\_state}))中的一个指令实例$i$可以总是宣布它的挂起的内存操作足迹。
行动:
% An instruction instance $i$ in state {\sc Plain}({\sc Store\_ea}({\it kind}, {\it address}, {\it size}, {\it next\_state})) can always announce its pending memory store operation footprint.
% Action:
\begin{enumerate}
\item 构造合适的内存存储操作$msos$(不带有存储的值): % construct the appropriate memory store operations $msos$ (without the store value):
  \begin{itemize}
  \item 如果{\it address}对齐到{\it size},那么$msos$是一个单独的针对{\it address}的{\it size}字节的内存存储操作; % if {\it address} is aligned to {\it size} then $msos$ is a single memory store operation of {\it size} bytes to {\it address};
  \item 否则,$msos$是一组数量为{\it size}的内存存储操作集合,每个长度1字节,针对地址$\textit{address}\ldots\textit{address}+\textit{size}-1$。  %  otherwise, $msos$ is a set of {\it size} memory store operations, each of one-byte size, to the addresses $\textit{address}\ldots\textit{address}+\textit{size}-1$.
  \end{itemize}
\item 把$i.\textit{mem\_stores}$设置为$msos$;并且  % set $i.\textit{mem\_stores}$ to $msos$; and
\item 把$i$的状态更新到{\sc Plain}({\it next\_state})。  % update the state of $i$ to {\sc Plain}({\it next\_state}).
\end{enumerate}

\begin{commentary}
  注意,在采取上述过渡之后,内存存储操作还没有拥有它们的值。把这个过渡从下面的过渡分离出来的重要性在于,它允许其它的程序次序后继的存储指令观察到这个指令的内存足迹,并且如果它们不重叠的话,尽可能早地乱序传播(即,在数据寄存器的值变得可用之前)。
% Note that after taking the transition above the memory store operations do not yet have their values.
% The importance of splitting this transition from the transition below is that it allows other program-order-successor store instructions to observe the memory footprint of this instruction, and if they don't overlap, propagate out of order as early as possible (i.e.~before the data register value becomes available).
\end{commentary}


\paragraph{初始化内存存储操作的值}\label{omm:instantiate_store_value}
状态{\sc Plain}({\sc Store\_memv}({\it mem\_value}中的一个指令实例i可以总是初始化内存存储操作$i.\textit{mem\_stores}$的值。
行动:
% An instruction instance $i$ in state {\sc Plain}({\sc Store\_memv}({\it mem\_value}, {\it store\_continuation})) can always instantiate the values of the memory store operations $i.\textit{mem\_stores}$.
% Action:
\begin{enumerate}
\item 在内存存储操作{\it mem\_value}之间分割出$i.\textit{mem\_stores}$;以及 % split {\it mem\_value} between the memory store operations $i.\textit{mem\_stores}$; and
\item 把$i$的状态更新到{\sc Pending\_mem\_stores}({\it store\_continuation})。 % update the state of $i$ to {\sc Pending\_mem\_stores}({\it store\_continuation}).
\end{enumerate}


\paragraph{提交存储指令}\label{omm:commit_stores}
一个非{\tt sc}存储指令,或者一个在“提交和传播一个{\tt sc}的存储操作”的上下文中的{\tt sc}指令,如果在状态{\sc Pending\_mem\_stores}({\it store\_continuation})之中,那么可以被提交(不要与传播相混淆),如果:
% An uncommitted instruction instance $i$ of a non-{\tt sc} store instruction or an {\tt sc} instruction in the context of the ``\nameref{omm:commit_sc}'' transition, in state {\sc Pending\_mem\_stores}({\it store\_continuation}), can be committed (not to be confused with propagated) if:
\begin{enumerate}
\item $i$具有完全决定的数据; % $i$ has fully determined data;
\item 所有的程序次序在先的条件分支和间接跳转指令都完成了; % all program-order-previous conditional branch and indirect jump instructions are finished;
\item 所有的程序次序在先的设置了{\tt .sw}的{\tt fence}指令都完成了; % all program-order-previous {\tt fence} instructions with {\tt .sw} set are finished;
\item 所有的程序次序在先的{\tt fence.tso}指令都完成了; % all program-order-previous {\tt fence.tso} instructions are finished;
% \item all program-order-previous {\tt fence.i} instructions are finished;
\item 所有的程序次序在先的load-acquire指令都完成了; % all program-order-previous load-acquire instructions are finished;
\item 所有的程序次序在先的store-acquire-release指令都完成了; %  all program-order-previous store-acquire-release instructions are finished;
\item 如果$i$是一个store-release,所有的程序次序在先的指令都完成了; %  if $i$ is a store-release, all program-order-previous instructions are finished;
\item\label{omm:commit_store:prev_addrs} 所有的程序次序在先的内存访问指令具有完全决定的内存足迹; % all program-order-previous memory access instructions have a fully determined memory footprint;
\item\label{omm:commit_store:prev_stores} 所有的程序次序在先的存储指令,除了失败的{\tt sc},都已经初始化并因此具有非空的{\it mem\_stores};并且  % all program-order-previous store instructions, except for {\tt sc} that failed, have initiated and so have non-empty {\it mem\_stores}; and
\item\label{omm:commit_store:prev_loads} 所有的程序次序在先的加载指令都已经初始化并因此具有非空的{\it mem\_loads}。 % all program-order-previous load instructions have initiated and so have non-empty {\it mem\_loads}.
\end{enumerate}
行动:记录i被提交了。
% Action: record that $i$ is committed.

\begin{commentary}
  注意,如果条件\ref{omm:commit_store:prev_addrs}被满足了,条件\ref{omm:commit_store:prev_stores}和\ref{omm:commit_store:prev_loads}也被满足,或者将在采取某些立即的过渡之后被满足。
  因此,对它们的要求并不会增强模型。通过要求它们,我们保证了先前的内存访问指令已经采取了足够的过渡,使得它们的内存操作对\nameref{omm:prop_store}的条件检查可见,这是指令将采取的下一个过渡,使得那个条件更加简单。
% Notice that if condition \ref{omm:commit_store:prev_addrs} is satisfied the conditions \ref{omm:commit_store:prev_stores} and \ref{omm:commit_store:prev_loads} are also satisfied, or will be satisfied after taking some eager transitions.
% Hence, requiring them does not strengthen the model.
% By requiring them, we guarantee that previous memory access instructions have taken enough transitions to make their memory operations visible for the condition check of \nameref{omm:prop_store}, which is the next transition the instruction will take, making that condition simpler.
\end{commentary}


\paragraph{传播存储操作}\label{omm:prop_store}
对于状态{\sc Pending\_mem\_stores}({\it store\_continuation})中的一个提交的指令实例$i$,和一个在$i.\textit{mem\_stores}$之中的未传播的内存存储操作$mso$,$mso$可以被传播,如果:
% For a committed instruction instance $i$ in state {\sc Pending\_mem\_stores}({\it store\_continuation}), and an unpropagated memory store operation $mso$ in $i.\textit{mem\_stores}$, $mso$ can be propagated if:
\begin{enumerate}
\item 所有的与$mso$重叠的程序次序在先的存储指令的内存存储操作都已经传播;  % all memory store operations of program-order-previous store instructions that overlap with $mso$ have already propagated;
\item 所有的与$mso$重叠的程序次序在先的加载指令的内存加载操作都已经被满足,并且(加载指令)是{\em 不可重启的}(见下面的定义);并且  % all memory load operations of program-order-previous load instructions that overlap with $mso$ have already been satisfied, and (the load instructions) are {\em non-restartable} (see definition below); and
\item 所有的通过转发$mso$被满足的内存加载操作都被完全地满足。  % all memory load operations that were satisfied by forwarding $mso$ are entirely satisfied.
\end{enumerate}
此处,一个{\em 未完成的}指令实例$i$是不可重启的,如果:
% Where a non-finished instruction instance $j$ is {\em non-restartable} if:
\begin{enumerate}
\item 不存在一个存储指令s和s的一个未传播的内存存储操作mso,使得对mso应用“传播存储操作”过渡的行为将导致j的重启;并且 % there does not exist a store instruction $s$ and an unpropagated memory store operation $mso$ of $s$ such that applying the action of the ``\nameref{omm:prop_store}'' transition to $mso$ will result in the restart of $j$; and
\item 不存在一个未完成的加载指令l和l的一个内存加载操作mlo,使得对mlo应用“通过从未传播的存储转发来满足内存加载操作”/“从内存满足内存加载操作”过渡(甚至mlo已经被满足)将导致j的重启。  % there does not exist a non-finished load instruction $l$ and a memory load operation $mlo$ of $l$ such that applying the action of the ``\nameref{omm:sat_by_forwarding}''/``\nameref{omm:sat_from_mem}'' transition (even if $mlo$ is already satisfied) to $mlo$ will result in the restart of $j$.
\end{enumerate}
行动:
% Action:
\begin{enumerate}
\item 使用$mso$更新共享内存的状态;  % update the shared memory state with $mso$;
\item 更新$i.\textit{mem\_stores}$,以表示$mso$被传播了;以及  % update $i.\textit{mem\_stores}$ to indicate that $mso$ was propagated; and
\item 重启任何因为这个的结果而已经违背了一致性的推测性指令,也就是说,对于每个程序次序晚于$i$的未完成的指令$i’$和$msoss’$中的与$mso$重叠而不来自$mso$的每个$i’$的内存加载操作$mlo’$,并且$msos’$不是来自$i$的一个程序次序后继,重启$i’$和它的{\em 重启依赖}(见:\nameref{omm:sat_by_forwarding})。  % restart any speculative instructions which have violated coherence as a result of this, i.e., for every non-finished instruction $i'$ program-order-after $i$ and every memory load operation $mlo'$ of $i'$ that was satisfied from $msoss'$, if there exists a memory store operation slice $msos'$ in $msoss'$ that overlaps with $mso$ and is not from $mso$, and $msos'$ is not from a program-order-successor of $i$, restart $i'$ and its {\em restart-dependents} (see \nameref{omm:sat_by_forwarding}).
\end{enumerate}


\paragraph{提交和传播一个{\tt sc}的存储操作}\label{omm:commit_sc}
一个来自硬件线程$h$的、状态{\sc Pending\_mem\_stores}({\it store\_continuation})中的、未提交的{\tt sc}指令实例$i$,带有一个配对的{\tt lr} $i’$,其已经被某些存储片段$msoss$所满足,可以被同时提交和传播,如果:
% An uncommitted {\tt sc} instruction instance $i$, from hart $h$, in state {\sc Pending\_mem\_stores}({\it store\_continuation}), with a paired {\tt lr} $i'$ that has been satisfied by some store slices $msoss$, can be committed and propagated at the same time if:
\begin{enumerate}
\item $i’$被完成; % $i'$ is finished;
\item 每个已经转发到$i’$的内存存储操作都是已传播的; % every memory store operation that has been forwarded to $i'$ is propagated;
\item \nameref{omm:commit_stores}的条件被满足; % the conditions of \nameref{omm:commit_stores} is satisfied;
\item \nameref{omm:prop_store}的条件被满足(注意一个{\tt sc}指令只能有一个内存存储操作);以及  % the conditions of \nameref{omm:prop_store} is satisfied (notice that an {\tt sc} instruction can only have one memory store operation); and
\item 在共享内存中,对于每个来自$msoss$的存储片段$msos$,从$msos$被传播到内存后的任何时刻,$msos$都还没有被来自不是$h$的硬件线程的存储覆写。  % for every store slice $msos$ from $msoss$, $msos$ has not been overwritten, in the shared memory, by a store that is from a hart that is not $h$, at any point since $msos$ was propagated to memory.
\end{enumerate}
行动:
% Action:
\begin{enumerate}
\item 应用\nameref{omm:commit_stores}的行动;以及  % apply the actions of \nameref{omm:commit_stores}; and
\item 应用\nameref{omm:prop_store}的行动。  %  apply the action of \nameref{omm:prop_store}.
\end{enumerate}


\paragraph{Late {\tt sc} fail}\label{omm:late_sc_fail}
状态{\sc Pending\_mem\_stores}({\it store\_continuation})中的一个{\tt sc}指令实例$i$,如果还没有传播它的内存存储操作,那么总是可以被造成失败。
行动:
% An {\tt sc} instruction instance $i$ in state {\sc Pending\_mem\_stores}({\it store\_continuation}), that has not propagated its memory store operation, can always be made to fail.
% Action:
\begin{enumerate}
\item 清除$i.\textit{mem\_stores}$;以及 % clear $i.\textit{mem\_stores}$; and
\item 把$i$的状态更新到{\sc Plain}({\it store\_continuation(false)})。  % update the state of $i$ to {\sc Plain}({\it store\_continuation(false)}).
\end{enumerate}

\begin{commentary}
  为了效率,{\tt rmem}工具只在不可能采用一个提交和传播一个{\tt sc}存储操作过渡的时候,才允许这个过渡。这不影响所允许的最终状态的集合,但是当交互式地探究时,如果sc应当失败,那么应当使用早期sc失败过渡,而不是期待这个过渡。
% For efficiency, the {\tt rmem} tool allows this transition only when it is not possible to take the \nameref{omm:commit_sc} transition.
% This does not affect the set of allowed final states, but when explored interactively, if the {\tt sc} should fail one should use the \nameref{omm:early_sc_fail} transition instead of waiting for this transition.
\end{commentary}

\paragraph{完整存储操作}\label{omm:complete_stores}
状态{\sc Pending\_mem\_stores}({\it store\_continuation})中的一个存储指令实例$i$,如果在$i.\textit{mem\_stores}$中的所有内存存储操作都已经被传播,那么可以总是完整的(不要与完成相混淆)。行动:把$i$的状态更新到{\sc Plain}({\it store\_continuation(true)})。
% A store instruction instance $i$ in state {\sc Pending\_mem\_stores}({\it store\_continuation}), for which all the memory store operations in $i.\textit{mem\_stores}$ have been propagated, can always be completed (not to be confused with finished).
% Action: update the state of $i$ to {\sc Plain}({\it store\_continuation(true)}).


\paragraph{满足、提交和传播一个AMO的操作}\label{omm:do_amo}
状态{\sc Pending\_mem\_loads}({\it load\_continuation})中的一个AMO指令实例$i$可以执行它的内存访问,如果可以执行下列顺序的过渡而没有干扰的过渡:
% An AMO instruction instance $i$ in state {\sc Pending\_mem\_loads}({\it load\_continuation}) can perform its memory access if it is possible to perform the following sequence of transitions with no intervening transitions:
\begin{enumerate}
\item \nameref{omm:sat_from_mem}
\item \nameref{omm:complete_loads}
\item \nameref{omm:sail_interp} (zero or more times)
\item \nameref{omm:instantiate_store_value}
\item \nameref{omm:commit_stores}
\item \nameref{omm:prop_store}
\item \nameref{omm:complete_stores}
\end{enumerate}
并且额外地,完成指令的条件,除了不要求$i$在状态{\sc Plain}({\sc Done}之中,其余在这些过渡之后保持不变。行动:一个接一个地、不加干扰过渡地,执行上述序列的过渡(这不包括完成指令),
% and in addition, the condition of \nameref{omm:finish}, with the exception of not requiring $i$ to be in state {\sc Plain}({\sc Done}), holds after those transitions.
% Action: perform the above sequence of transitions (this does not include \nameref{omm:finish}), one after the other, with no intervening transitions.

\begin{commentary}
  注意,程序次序在先的存储不能被转发到一个AMO的加载。这纯粹是因为上面的过渡序列不包括转发过渡。但是即使它确实包含了转发过渡,当尝试执行传播存储操作过渡的时候,该序列也将失败,因为这个过渡需要所有的程序次序在先的针对重叠的内存足迹的存储指令都被传播,而传播需要存储操作是未被传播的。
% Notice that program-order-previous stores cannot be forwarded to the load of an AMO.
% This is simply because the sequence of transitions above does not include the forwarding transition.
% But even if it did include it, the sequence will fail when trying to do the \nameref{omm:prop_store} transition, as this transition requires all program-order-previous store operations to overlapping memory footprints to be propagated, and forwarding requires the store operation to be unpropagated.

此外,一个AMO的存储不能被转发到一个程序次序后继的加载。在采取上面的过渡之前,AMO的存储指令没有拥有它的值,并因此不能被转发;在采取上面的过渡之后,存储操作被传播,并因此不能被转发。
% In addition, the store of an AMO cannot be forwarded to a program-order-successor load.
% Before taking the transition above, the store operation of the AMO does not have its value and therefore cannot be forwarded; after taking the transition above the store operation is propagated and therefore cannot be forwarded.
\end{commentary}


\paragraph{提交屏障}\label{omm:commit_fence}
在状态{\sc Plain}({\sc Fence}({\it kind}, {\it next\_state}))中的一个屏障指令实例$i$可以被提交,如果:
% A fence instruction instance $i$ in state {\sc Plain}({\sc Fence}({\it kind}, {\it next\_state})) can be committed if:
\begin{enumerate}
\item 如果$i$是一个普通fence,并且它设置了{\tt .pr},那么所有的程序次序在先的加载指令都是完成的; % if $i$ is a normal fence and it has {\tt .pr} set, all program-order-previous load instructions are finished;
\item 如果$i$是一个普通fence,并且它设置了{\tt .pw},那么所有的程序次序在先的存储指令都是完成的;以及 % if $i$ is a normal fence and it has {\tt .pw} set, all program-order-previous store instructions are finished; and
\item 如果$i$是一个{\tt fence.tso},那么所有的程序次序在先的加载和存储指令都是完成的。 % if $i$ is a {\tt fence.tso}, all program-order-previous load and store instructions are finished.
% \item if $i$ is a {\tt fence.i} instruction, all program-order-previous memory access instructions have fully determined memory footprints.
\end{enumerate}
行动:
% Action:
\begin{enumerate}
\item 记录$i$被提交了;以及 % record that $i$ is committed; and
\item 把$i$的状态更新到{\sc Plain}({\it next\_state})。 % update the state of $i$ to {\sc Plain}({\it next\_state}).
\end{enumerate}


\paragraph{寄存器读}\label{omm:reg_read}
% \paragraph{Register read}\label{omm:reg_read}
状态{\sc Plain}({\sc Read\_reg}({\it reg\_name}中的一个指令实例$i$可以执行一次{\it reg\_name}的寄存器读,如果它需要读取自的每个指令实例都已经执行了所期待的{\it reg\_name}寄存器写。
% An instruction instance $i$ in state {\sc Plain}({\sc Read\_reg}({\it reg\_name}, {\it read\_cont})) can do a register read of {\it reg\_name} if every instruction instance that it needs to read from has already performed the expected {\it reg\_name} register write.

对于{\it reg\_name}的每一位,令{\it read\_sources}包括,可以写到该位的最近(以程序次序)的指令实例对于该位的写入(如果有的话)。如果没有这样的指令,来源就是来自{\it initial\_register\_state}的初始的寄存器值。令{\it reg\_value}为从{\it read\_sources}集合的值。行动:
% Let {\it read\_sources} include, for each bit of {\it reg\_name}, the write to
% that bit by the most recent (in program order) instruction instance that can write to that bit, if any. If there is no such instruction, the source is the initial register value from {\it initial\_register\_state}.
% Let  {\it reg\_value} be the value assembled from {\it read\_sources}.
% Action:
\begin{enumerate}
\item 把{\it reg\_name}添加到带有{\it read\_sources}和{\it reg\_value}的$i.\textit{reg\_reads}$;以及 % add {\it reg\_name} to $i.\textit{reg\_reads}$ with {\it read\_sources} and {\it reg\_value}; and
\item 把$i$的状态更新为{\sc Plain}({\it read\_cont(reg\_value)})。 % update the state of $i$ to {\sc Plain}({\it read\_cont(reg\_value)}).
\end{enumerate}


\paragraph{寄存器写}\label{omm:reg_write}
状态{\sc Plain}({\sc Write\_reg}({\it reg\_name}中的一个指令实例$i$总是可以执行一个{\it reg\_name}寄存器写。行动:
% An instruction instance $i$ in state {\sc Plain}({\sc Write\_reg}({\it reg\_name}, {\it reg\_value}, {\it next\_state})) can always do a {\it reg\_name} register write.
% Action:
\begin{enumerate}
\item 把{\it reg\_name}添加到带有$deps$和{\it reg\_value}的$i.\textit{reg\_writes}$;以及  % add {\it reg\_name} to $i.\textit{reg\_writes}$ with $deps$ and {\it reg\_value}; and
\item 把$i$的状态更新到{\sc Plain}({\it next\_state})。 % update the state of $i$ to {\sc Plain}({\it next\_state}).
\end{enumerate}
此处$deps$是所有来自$i.\textit{reg\_reads}$的读取来源的集合,和一个标志的配对,这个标志为真,当且仅当$i$是一个已经被完全地满足的加载指令实例。
% where $deps$ is a pair of the set of all {\it read\_sources} from $i.\textit{reg\_reads}$, and a flag that is true iff $i$ is a load instruction instance that has already been entirely satisfied.


\paragraph{伪代码内部步骤}\label{omm:sail_interp}
状态{\sc Plain}({\sc Internal}({\it next\_state}))中的一个指令实例$i$总是可以执行那个伪代码内部的步骤。
行动:把$i$的状态更新到{\sc Plain}({\it next\_state})。
% An instruction instance $i$ in state {\sc Plain}({\sc Internal}({\it next\_state})) can always do that pseudocode-internal step.
% Action: update the state of $i$ to {\sc Plain}({\it next\_state}).


\paragraph{完成指令}\label{omm:finish}
状态{\sc Plain}({\sc Done})中的一个未完成的指令实例$i$可以被完成,如果:
% A non-finished instruction instance $i$ in state {\sc Plain}({\sc Done}) can be finished if:
\begin{enumerate}
\item 如果$i$是一个加载指令: %if $i$ is a load instruction:
  \begin{enumerate}
  \item 所有的程序次序在先的load-acquire指令都被完成了; % all program-order-previous load-acquire instructions are finished;
  \item 所有的程序次序在先的设置了.sr的{\tt fence}指令都被完成了; % all program-order-previous {\tt fence} instructions with {\tt .sr} set are finished;
  \item 对于每个程序次序在先的没有完成的{\tt fence.tso}指令,$f$,所有的程序次序在$f$之前的加载指令都被完成了;并且  % for every program-order-previous {\tt fence.tso} instruction, $f$, that is not finished, all load instructions that are program-order-before $f$ are finished; and
  \item 保证$i$的内存加载指令所读取的值将不会引起对一致性的违背,即,对于任何程序次序在先的指令实例$i’$,令$\textit{cfp}$是来自程序次序在$i$和$i’$之间的存储指令的已传播的内存存储操作,和从程序次序在$i$和$i’$之间、包括$i’$的存储指令转发到$i$的{\em 固定的内存存储操作},这二者的组合,并令$\overline{\textit{cfp}}$是$i$的内存足迹中的$\textit{cfp}$的补。
  如果$\overline{\textit{cfp}}$是不空的: 
  % it is guaranteed that the values read by the memory load operations of $i$ will not cause coherence violations, i.e., for any program-order-previous instruction instance $i'$, let $\textit{cfp}$ be the combined footprint of propagated memory store operations from store instructions program-order-between $i$ and $i'$, and {\em fixed memory store operations} that were forwarded to $i$ from store instructions program-order-between $i$ and $i'$ including $i'$, and let $\overline{\textit{cfp}}$ be the complement of $\textit{cfp}$ in the memory footprint of $i$.
  % If $\overline{\textit{cfp}}$ is not empty:
    \begin{enumerate}
    \item $i’$具有一个完全决定的内存足迹; %  $i'$ has a fully determined memory footprint;
    \item $i’$没有与$\overline{\textit{cfp}}$重叠的未传播的内存存储操作;并且  % $i'$ has no unpropagated memory store operations that overlap with $\overline{\textit{cfp}}$; and
    \item 如果$i’$是一个带有与$\overline{\textit{cfp}}$重叠的内存足迹的加载,那么与$\overline{\textit{cfp}}$重叠的$i’$的所有的内存加载操作都被满足,并且$i’$是不可重启的(对于如何决定一个指令是否是不可重启的,见传播存储操作过渡)。  
    % if $i'$ is a load with a memory footprint that overlaps with $\overline{\textit{cfp}}$, then all the memory load operations of $i'$ that overlap with $\overline{\textit{cfp}}$ are satisfied and $i'$ is {\em non-restartable} (see the \nameref{omm:prop_store} transition for how to determined if an instruction is non-restartable).
    \end{enumerate}
  这里,一个内存存储操作被称为固定的,如果存储指令具有完全决定的数据。
  % Here, a memory store operation is called fixed if the store instruction has fully determined data.
  \end{enumerate}
\item $i$具有一个完全决定的数据;并且  % $i$ has a fully determined data; and
\item 如果$i$不是一个屏障,那么所有的程序次序在先的条件分支和间接跳转指令都被完成了。  % if $i$ is not a fence, all program-order-previous conditional branch and indirect jump instructions are finished.
\end{enumerate}
行动:
% Action:
\begin{enumerate}
\item 如果$i$是一个条件分支或者间接跳转指令,丢弃任何未采取的执行路径,即,移除{\it instruction\_tree}中的所有的不可被分支/跳转采用而达到的指令实例;并且  % if $i$ is a conditional branch or indirect jump instruction, discard any untaken paths of execution, i.e., remove all instruction instances that are not reachable by the branch/jump taken in {\it instruction\_tree}; and
\item 记录该指令为完成的,即,设置{\it finished}为真。  % record the instruction as finished, i.e., set {\it finished} to {\it true}.
\end{enumerate}


\subsection{局限性}\label{sec:omm:limitations}
\begin{itemize}
\item 模型覆盖用户级RV64I和RV64A。
特别地,它不支持未对齐的原子性扩展“Zam”或者总存储排序扩展“Ztso”。
使模型适应RV32I/A和G、Q还有C扩展应当是轻而易举的,但是我们从未尝试过它。这将主要涉及,为指令写Sail代码,同时对并发模型的改变(如果有的话)最小。
% The model covers user-level RV64I and RV64A.
% In particular, it does not support the misaligned atomics extension ``Zam'' or the total store ordering extension ``Ztso''.
% It should be trivial to adapt the model to RV32I/A and to the G, Q and C extensions, but we have never tried it. This will involve, mostly, writing Sail code for the instructions, with minimal, if any, changes to the concurrency model.
\item 模型只覆盖了一般的内存访问(它不处理I/O访问)。  % The model covers only normal memory accesses (it does not handle I/O accesses).
\item 模型没有覆盖TLB相关的效果。  % The model does not cover TLB-related effects.
\item 模型假设指令内存是固定的。特别地,Fetch指令过渡不会生成内存加载操作,而共享内存不会在过渡中被涉及。反而,模型依赖于一个在给定内存位置时提供操作码的外部指示。 % The model assumes the instruction memory is fixed.
% In particular, the \nameref{omm:fetch} transition does not generate memory load operations, and the shared memory is not involved in the transition.
% Instead, the model depends on an external oracle that provides an opcode when given a memory location.
\item 模型没有覆盖异常、陷入和中断。 % The model does not cover exceptions, traps and interrupts.
\end{itemize}


