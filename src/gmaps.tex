\chapter{RV32/64G指令集列表}
% \chapter{RV32/64G Instruction Set Listings}

RISC-V项目的一个目标是,将它作为一个稳定的软件开发目标来使用。
为了这个目的,我们定义了一个基础ISA(RV32I或RV64I)加上选择的标准扩展(IMAFD、Zicsr、Zifencei)的组合,
作为一个“通用目的”的ISA,而且,我们使用缩写G来表示指令集扩展的IMAFDZicsr Zifencei组合。
本章展示了为RV32G和RV64G列出的操作码映射和指令集。
% One goal of the RISC-V project is that it be used as a stable software
% development target.  For this purpose, we define a combination of a
% base ISA (RV32I or RV64I) plus selected standard extensions (IMAFD, Zicsr, Zifencei) as
% a ``general-purpose'' ISA, and we use the abbreviation G for the IMAFDZicsr\_Zifencei
% combination of instruction-set extensions.    This chapter presents
% opcode maps and instruction-set listings for RV32G and RV64G.

\input{opcode-map}

表~\ref{opcodemap}显示了RVG的主要操作码的映射。设置了3或更低位的主要操作码被保留用于超过32位的指令长度。
标记为{\em 保留的}的操作码应当被避免用于自定义的指令集扩展,因为它们可能会用在未来的标准扩展中。
标记为{\em 自定义-0}和{\em 自定义-1}的主操作码应当被避免用于未来的标准扩展,并且建议用于具有基础32位指令格式的自定义指令集扩展。
标记为{\em 自定义-2/rv128}和{\em 自定义-3/rv128}的操作码被保留,以供RV128未来使用,
但如果不是RV128,那么它们将避免被用于标准扩展,并因此也可以被用于RV32和RV64中的自定义指令集扩展。
% Table~\ref{opcodemap} shows a map of the major opcodes for RVG.  Major
% opcodes with 3 or more lower bits set are reserved for instruction
% lengths greater than 32 bits.  Opcodes marked as {\em reserved} should
% be avoided for custom instruction-set extensions as they might be used
% by future standard extensions.  Major opcodes marked as {\em custom-0}
% and {\em custom-1} will be avoided by future standard extensions and
% are recommended for use by custom instruction-set extensions within
% the base 32-bit instruction format.  The opcodes marked {\em
%   custom-2/rv128} and {\em custom-3/rv128} are reserved for future use
% by RV128, but will otherwise be avoided for standard extensions and so
% can also be used for custom instruction-set extensions in RV32 and
% RV64.

我们相信RV32G和RV64G为大量通用目的计算提供了简单但完整的指令集。
可以添加可选的在~\ref{compressed}章中描述的压缩指令集(形成RV32GC和RV64GC)来改善性能、代码尺寸和能量效率,尽管这会带来额外的硬件复杂度。
% We believe RV32G and RV64G provide simple but complete instruction
% sets for a broad range of general-purpose computing.  The optional
% compressed instruction set described in Chapter~\ref{compressed} can
% be added (forming RV32GC and RV64GC) to improve performance, code
% size, and energy efficiency, though with some additional hardware
% complexity.

随着我们的脚步超越了IMAFDC,走进进一步的指令集扩展,添加的指令更加倾向于特定领域,
而只对严格的某类应用(例如,多媒体应用或者安全应用)提供收益。不像大多数商业化的ISA,
RISC-V ISA的设计将基础ISA和广泛可用的标准扩展与这些更特定化的额外部分清晰地分离开。
第~\ref{extensions}章对于向RISC-V ISA添加扩展的方法进行了更加广泛的讨论。
% As we move beyond IMAFDC into further instruction-set extensions, the
% added instructions tend to be more domain-specific and only provide
% benefits to a restricted class of applications, e.g., for multimedia
% or security.  Unlike most commercial ISAs, the RISC-V ISA design
% clearly separates the base ISA and broadly applicable standard
% extensions from these more specialized additions.
% Chapter~\ref{extensions} has a more extensive discussion of ways to
% add extensions to the RISC-V ISA.

\input{instr-table}

\FloatBarrier
表~\ref{rvgcsrnames}列出了当前被分配了CSR地址的CSR。计时器、计数器,和浮点CSR是这个规范中定义的仅有的CSR。
% Table~\ref{rvgcsrnames} lists the CSRs that have
% currently been allocated CSR addresses.  The timers, counters, and
% floating-point CSRs are the only CSRs defined in this specification.

\begin{table}[htb!]
\begin{center}
\begin{tabular}{|l|l|l|l|}
\hline
编号   & 权限 & 名称 & 描述 \\
\hline
\multicolumn{4}{|c|}{浮点控制和状态寄存器} \\
\hline
\tt 0x001 & 读/写  &\tt fflags     & 浮点增长异常。 \\
\tt 0x002 & 读/写  &\tt frm        & 浮点动态舍入模式。 \\
\tt 0x003 & 读/写  &\tt fcsr       & 浮点控制和状态寄存器({\tt frm} + {\tt fflags})。 \\
\hline
\multicolumn{4}{|c|}{计数器和计时器} \\
\hline
\tt 0xC00 & 只读  &\tt cycle      & 用于RDCYCLE指令的周期计数器。\\
\tt 0xC01 & 只读  &\tt time       & 用于RDTIME指令的计时器。 \\
\tt 0xC02 & 只读  &\tt instret    & 用于RDINSTRET指令的指令退场计数器。\\
\tt 0xC80 & 只读  &\tt cycleh     & {\tt cycle}的高32位,RV32I专用。 \\
\tt 0xC81 & 只读  &\tt timeh      & {\tt time}的高32位,RV32I专用。 \\
\tt 0xC82 & 只读  &\tt instreth   & {\tt instret}的高32位,RV32I专用。 \\
\hline
\end{tabular}
\end{center}
\caption{RISC-V控制和状态寄存器(CSR)地址映射。}
\label{rvgcsrnames}
\end{table}

% \begin{table}[htb!]
%   \begin{center}
%   \begin{tabular}{|l|l|l|l|}
%   \hline
%   Number    & Privilege & Name & Description \\
%   \hline
%   \multicolumn{4}{|c|}{Floating-Point Control and Status Registers} \\
%   \hline
%   \tt 0x001 & Read/write  &\tt fflags     & Floating-Point Accrued Exceptions. \\
%   \tt 0x002 & Read/write  &\tt frm        & Floating-Point Dynamic Rounding Mode. \\
%   \tt 0x003 & Read/write  &\tt fcsr       & Floating-Point Control and Status
%   Register ({\tt frm} + {\tt fflags}). \\
%   \hline
%   \multicolumn{4}{|c|}{Counters and Timers} \\
%   \hline
%   \tt 0xC00 & Read-only  &\tt cycle      & Cycle counter for RDCYCLE instruction. \\
%   \tt 0xC01 & Read-only  &\tt time       & Timer for RDTIME instruction. \\
%   \tt 0xC02 & Read-only  &\tt instret    & Instructions-retired counter for RDINSTRET instruction. \\
%   \tt 0xC80 & Read-only  &\tt cycleh     & Upper 32 bits of {\tt cycle}, RV32I only. \\
%   \tt 0xC81 & Read-only  &\tt timeh      & Upper 32 bits of {\tt time}, RV32I only. \\
%   \tt 0xC82 & Read-only  &\tt instreth   & Upper 32 bits of {\tt instret}, RV32I only. \\
%   \hline
%   \end{tabular}
%   \end{center}
%   \caption{RISC-V control and status register (CSR) address map.}
%   \label{rvgcsrnames}
%   \end{table}