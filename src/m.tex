\chapter{用于乘法和除法的“M”标准扩展(2.0版本)}

这章描述了标准整数乘法和除法指令扩展,命名为“M”,包含了将两个整数寄存器中持有的值相乘或相除的指令.
% This chapter describes the standard integer multiplication and
% division instruction extension, which is named ``M'' and contains
% instructions that multiply or divide values held in two integer
% registers.

\begin{commentary}

  我们将整数乘法和除法从基础中分离出来,以简化低端的实现,或者用于那些整数乘法和除法操作并不频繁的、或在相接的加速器中能更好地处理的应用。
% We separate integer multiply and divide out from the base to simplify
% low-end implementations, or for applications where integer multiply
% and divide operations are either infrequent or better handled in
% attached accelerators.
\end{commentary}

\section{乘法操作}
\label{multiplication-operations}

\vspace{-0.2in}
\begin{center}
\begin{tabular}{S@{}R@{}R@{}S@{}R@{}O}
\\
\instbitrange{31}{25} &
\instbitrange{24}{20} &
\instbitrange{19}{15} &
\instbitrange{14}{12} &
\instbitrange{11}{7} &
\instbitrange{6}{0} \\
\hline
\multicolumn{1}{|c|}{funct7} &
\multicolumn{1}{c|}{rs2} &
\multicolumn{1}{c|}{rs1} &
\multicolumn{1}{c|}{funct3} &
\multicolumn{1}{c|}{rd} &
\multicolumn{1}{c|}{opcode} \\
\hline
7 & 5 & 5 & 3 & 5 & 7 \\
MULDIV & 乘数 & 被乘数 & MUL/MULH[[S]U] & dest & OP    \\
MULDIV & 乘数 & 被乘数 & MULW           & dest & OP-32 \\
\end{tabular}
\end{center}

MUL在{\em rs1}和{\em rs2}上执行XLEN位$\times$XLEN位乘法,
并将低XLEN位放入目的寄存器。MULH、MULHU和MULHSU分别针对有符号数$\times$有符号数、无符号数$\times$无符号数、
有符号rs1$\times$无符号rs2执行相同的乘法,但是返回2$\times$XLEN位结果的高XLEN位。
如果同一个乘积的高位和低位都需要,那么推荐的代码序列是:MULH[[S]U] {\em rdh,rs1, rs2}; MUL {\em rdl, rs1, rs2}
(源寄存器标识符必须次序相同,且{\em rdh}不能与{\em rs1}或{\em rs2}相同)。然后微架构可以把这些代码融合进单独的一次乘法操作,而不是两次分离的乘法。

% MUL performs an XLEN-bit$\times$XLEN-bit multiplication
% of {\em rs1} by {\em rs2} and places the
% lower XLEN bits in the destination register.  MULH, MULHU, and MULHSU
% perform the same multiplication but return the upper XLEN bits of the
% full 2$\times$XLEN-bit product, for signed$\times$signed,
% unsigned$\times$unsigned, and \wunits{signed}{\em rs1}$\times$\wunits{unsigned}{\em rs2} multiplication,
% respectively.  If both the high and low bits of the same product are
% required, then the recommended code sequence is: MULH[[S]U] {\em rdh,
%   rs1, rs2}; MUL {\em rdl, rs1, rs2} (source register specifiers must
% be in same order and {\em rdh} cannot be the same as {\em rs1} or {\em
%   rs2}).  Microarchitectures can then fuse these into a single
% multiply operation instead of performing two separate multiplies.

\begin{commentary}
  MULHSU被用在多字有符号乘法中,将被乘数(包含符号位)的最高位有效字与乘数(无符号的)较低位有效字相乘。
% MULHSU is used in multi-word signed multiplication to multiply the
% most-significant word of the multiplicand (which contains the sign bit)
% with the less-significant words of the multiplier (which are unsigned).
\end{commentary}

MULW是一个RV64指令,它将源寄存器的低32位相乘,把符号扩展的低32位结果放入目的寄存器。
% MULW is an RV64 instruction that multiplies the lower 32 bits of the source
% registers, placing the sign-extension of the lower 32 bits of the result
% into the destination register.

\begin{commentary}
  在RV64中,MUL可以被用于获得64位乘积的高32位,但是有符号参数必须是合适的32位有符号值,
  反之无符号参数必须清除它们的高32位。如果参数不知道是符号扩展还是零扩展的,一个备选方案是把两个参数都向左移位32位,
  然后使用MULH[[S]U]。
% In RV64, MUL can be used to obtain the upper 32 bits of the 64-bit product,
% but signed arguments must be proper 32-bit signed values, whereas unsigned
% arguments must have their upper 32 bits clear.  If the
% arguments are not known to be sign- or zero-extended, an alternative is to
% shift both arguments left by 32 bits, then use MULH[[S]U].
\end{commentary}

\section{除法操作
  % Division Operations
  }

\vspace{-0.2in}
\begin{center}
\begin{tabular}{S@{}R@{}R@{}O@{}R@{}O}
\\
\instbitrange{31}{25} &
\instbitrange{24}{20} &
\instbitrange{19}{15} &
\instbitrange{14}{12} &
\instbitrange{11}{7} &
\instbitrange{6}{0} \\
\hline
\multicolumn{1}{|c|}{funct7} &
\multicolumn{1}{c|}{rs2} &
\multicolumn{1}{c|}{rs1} &
\multicolumn{1}{c|}{funct3} &
\multicolumn{1}{c|}{rd} &
\multicolumn{1}{c|}{opcode} \\
\hline
7 & 5 & 5 & 3 & 5 & 7 \\
MULDIV & 除数 & 被除数 & DIV[U]/REM[U]   & dest & OP    \\
MULDIV & 除数 & 被除数 & DIV[U]W/REM[U]W & dest & OP-32 \\
\end{tabular}
\end{center}

DIV和DIVU在{\em rs1}和{\em rs2}上执行XLEN位与XLEN位的有符号和无符号整数除法,结果向零取整。
REM和REMU提供了对应的除法操作的余数。对于REM,结果的符号等于被除数的符号。
% DIV and DIVU perform an XLEN bits by XLEN bits signed and unsigned integer
% division of {\em rs1} by {\em rs2}, rounding towards zero.
% REM and REMU provide the remainder of the corresponding division operation.
% For REM, the sign of the result equals the sign of the dividend.

\begin{commentary}
  对于有符号除法和无符号除法,都有
  \mbox{$\textrm{被除数} = \textrm{除数} \times \textrm{商} + \textrm{余数}$}
% For both signed and unsigned division, it holds that
% \mbox{$\textrm{dividend} = \textrm{divisor} \times \textrm{quotient} + \textrm{remainder}$}.
\end{commentary}

如果同一个除法的商和余数都需要,推荐的代码序列是:DIV[U] {\em rdq, rs1, rs2}; REM[U] {\em rdr, rs1, rs2} 
({\em rdq}不能与{\em rs1}或{\em rs2}相同)。然后微架构可以把这些代码融合为单独的一次除法操作,而不是执行两次分离的除法。
% If both the quotient and remainder
% are required from the same division, the recommended code sequence is:
% DIV[U] {\em rdq, rs1, rs2}; REM[U] {\em rdr, rs1, rs2} ({\em rdq}
% cannot be the same as {\em rs1} or {\em rs2}).  Microarchitectures can
% then fuse these into a single divide operation instead of performing
% two separate divides.

DIVW和DIVUW是RV64的指令,它们将{\em rs1}的低32位与{\em rs2}的低32位分别视为有符号整数和无符号整数并相除,
把32位商放入{\em rd},并符号扩展到64位。REMW和REMUW是RV64的指令,它们分别提供对应的有符号余数和无符号余数操作。
REMW和REMUW都总是把32位结果符号扩展到64位,包括除数为零时。
% DIVW and DIVUW are RV64 instructions that divide the
% lower 32 bits of {\em rs1} by the lower 32 bits of {\em rs2}, treating
% them as signed and unsigned integers respectively, placing the 32-bit
% quotient in {\em rd}, sign-extended to 64 bits.  REMW and REMUW
% are RV64 instructions that provide the corresponding
% signed and unsigned remainder operations respectively. Both REMW and
% REMUW always sign-extend the 32-bit result to 64 bits, including on a
% divide by zero.

表~\ref{tab:divby0}中总结了除数为零和除法溢出的语义。如果除数为零,商的所有位都被设置为1,余数等于被除数。
有符号的除法溢出只发生在最大负数被$-1$除的时候。溢出的有符号除法的商等于被除数,而余数为零。无符号除法不可能发生溢出。
% The semantics for division by zero and division overflow are summarized in
% Table~\ref{tab:divby0}.  The quotient of division by zero has all bits set, and
% the remainder of division by zero equals the dividend.  Signed division overflow
% occurs only when the most-negative integer is divided by $-1$.  The quotient of
% a signed division with overflow is equal to the dividend, and the remainder is
% zero. Unsigned division overflow cannot occur.

\begin{table}[h]
\center
\begin{tabular}{|l|c|c||c|c|c|c|}
\hline
Condition              & 被除数   & 除数 & DIVU[W]   & REMU[W] & DIV[W]     & REM[W] \\ \hline
Division by zero       & $x$        & 0       & $2^{L}-1$ & $x$     & $-1$       & $x$    \\
Overflow (signed only) & $-2^{L-1}$ & $-1$    & --        & --      & $-2^{L-1}$ & 0      \\
\hline
\end{tabular}
\caption{除数为零和除法溢出的语义。L是操作数的位宽度:或者是XLEN(对于DIV[U]和REM[U]),或者是32(对于DIV[U]W和REM[U]W)
% Semantics for division by zero and division overflow.
% L is the width of the operation in bits: XLEN for DIV[U] and REM[U], or
% 32 for DIV[U]W and REM[U]W.
}
\label{tab:divby0}
\end{table}

\begin{commentary}

  我们考虑过在整数被零除的时候产生异常,这些异常在大多数执行环境中会引发一个陷入。
  然而,这将是标准ISA中仅有的算数陷入(浮点异常会设置标志和写默认值,但是不会引起陷入),
  而对于这种情况,将需要语言解释器来与执行环境的陷入处理程序进行交互。此外,如果语言标准强制要求除数为零的异常必须引起控制流的立即改变,
  那么只需要为每个除法操作添加一条分支指令,而这个分支指令可以在除法之后被插入,并且通常应当非常大概率地被预测为不执行,
  几乎不增加运行时的负载。
% We considered raising exceptions on integer divide by zero, with these
% exceptions causing a trap in most execution environments.  However,
% this would be the only arithmetic trap in the standard ISA
% (floating-point exceptions set flags and write default values, but do
% not cause traps) and would require language implementors to interact
% with the execution environment's trap handlers for this case.
% Further, where language standards mandate that a divide-by-zero
% exception must cause an immediate control flow change, only a single
% branch instruction needs to be added to each divide operation, and
% this branch instruction can be inserted after the divide and should
% normally be very predictably not taken, adding little runtime
% overhead.

为了简化除法器电路,除数为零时,无论无符号还是有符号除法都返回所有位被设置为1的值。
全1值既是无符号除法返回的自然值,代表了最大的无符号数,也是简单无符号除法器实现的自然结果。
有符号除法经常使用无符号除法电路实现,并指定了相同的溢出结果来简化硬件。

% The value of all bits set is returned for both unsigned and signed
% divide by zero to simplify the divider circuitry.  The value of all 1s
% is both the natural value to return for unsigned divide, representing
% the largest unsigned number, and also the natural result for simple
% unsigned divider implementations.  Signed division is often
% implemented using an unsigned division circuit and specifying the same
% overflow result simplifies the hardware.
\end{commentary}

\section{Zmmul扩展(1.0版本)}

Zmmul扩展实现了M扩展的乘法子集。它添加了定义在第9.1节~\ref{multiplication-operations}中的所有指令,
即:MUL、MULH、MULHU、MULHSU、和(仅用于RV64的)MULW。这些编码与那些对应的M扩展的指令是相同的。
% The Zmmul extension implements the multiplication subset of the M extension.
% It adds all of the instructions defined in Section~\ref{multiplication-operations},
% namely: MUL, MULH, MULHU, MULHSU, and (for RV64 only) MULW.
% The encodings are identical to those of the corresponding M-extension instructions.

\begin{commentary}

  Zmmul扩展支持那些需要乘法操作但不需要除法操作的低成本实现。对于许多微控制器应用,除法操作太不频繁,
  导致除法器硬件的开销无法证明是正当的。相比之下,乘法操作更加频繁,使得乘法器硬件的开销更加正当。
  消除除法但保留乘法,令简单的FPGA软核特别获益,因为许多FPGA提供硬布线的乘法器,但是需要在软逻辑中实现除法器。
% The Zmmul extension enables low-cost implementations that require
% multiplication operations but not division.
% For many microcontroller applications, division operations are too
% infrequent to justify the cost of divider hardware.
% By contrast, multiplication operations are more frequent, making the cost of
% multiplier hardware more justifiable.
% Simple FPGA soft cores particularly benefit from eliminating division but
% retaining multiplication, since many FPGAs provide hardwired multipliers
% but require dividers be implemented in soft logic.
\end{commentary}
