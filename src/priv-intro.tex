\chapter{ 介绍
  % Introduction
}

% This document describes the RISC-V privileged architecture, which
% covers all aspects of RISC-V systems beyond the unprivileged ISA,
% including privileged instructions as well as additional functionality
% required for running operating systems and attaching external devices.

本文描述了RISC-V特权指令架构,其覆盖了在非特权ISA基础上的RISC-V系统的
各个方面,包括特权指令以及运行操作系统和连接外设需要的额外功能。

\begin{commentary}
% Commentary on our design decisions is formatted as in this paragraph,
% and can be skipped if the reader is only interested in the
% specification itself.

如果读者只关心规范本身,本段落的注释可以忽略。
\end{commentary}

\begin{commentary}
% We briefly note that the entire privileged-level design described in
% this document could be replaced with an entirely different
% privileged-level design without changing the unprivileged ISA, and
% possibly without even changing the ABI.  In particular, this
% privileged specification was designed to run existing popular
% operating systems, and so embodies the conventional level-based
% protection model.  Alternate privileged specifications could embody
% other more flexible protection-domain models.  For simplicity of
% expression, the text is written as if this was the only possible
% privileged architecture.

在不改变非特权指令集,甚至不改变ABI的情况下,本文描述的整个特权层级
设计可以完全用不同的特权层级设计取代。这种特权规范设计可以与当前
流行的操作系统相适应,体现了传统的基于层级的保护模型。另外,
特权规范可以体现其它更多的基于域保护的模型。为简洁起见,本文只以
一种可能的特权架构进行撰写。
\end{commentary}

\section{ RISC-V特权软件栈术语
  % RISC-V Privileged Software Stack Terminology
  }

% This section describes the terminology we use to describe components
% of the wide range of possible privileged software stacks for RISC-V.

本节对RISC-V的特权软件栈涉及的相关专业术语进行描述。

% Figure~\ref{fig:privimps} shows some of the possible software stacks
% that can be supported by the RISC-V architecture.  The left-hand side
% shows a simple system that supports only a single application running
% on an application execution environment (AEE).  The application is
% coded to run with a particular application binary interface (ABI).
% The ABI includes the supported user-level ISA plus a set of ABI calls to
% interact with the AEE.  The ABI hides details of the AEE from the
% application to allow greater flexibility in implementing the AEE.  The
% same ABI could be implemented natively on multiple different host OSs,
% or could be supported by a user-mode emulation environment running on
% a machine with a different native ISA.

图~\ref{fig:privimps}展示了RISC-V架构支持的可能软件栈。左图显示一个
简单的系统,这个系统只支持在应用执行环境(AEE)里面运行单一的程序。
应用程序使用特定的应用程序接口(ABI)运行。ABI包括支持
的用户级ISA,和一系列与AEE交互的ABI调用。从应用程序的角度看,ABI隐藏了
AEE的细节,这对于AEE的实现提供了更多的弹性。同样的ABI可能被不同的宿主
操作系统本地化实现,或者被运行在不同原生ISA的机器上的用户模式模拟环境所支持。

\begin{figure}[th]
\centering
\includegraphics[width=\textwidth]{figs/privimps.pdf}
\caption{支持各种特权执行形式的软件栈
  % Different implementation stacks supporting various forms of
  % privileged execution.
  }
\label{fig:privimps}
\end{figure}

\begin{commentary}
% Our graphical convention represents abstract interfaces using black
% boxes with white text, to separate them from concrete instances of
% components implementing the interfaces.

我们通常用黑框白字表示抽象接口,将其与实现接口的具体实例区别。
\end{commentary}

% The middle configuration shows a conventional operating system (OS)
% that can support multiprogrammed execution of multiple
% applications. Each application communicates over an ABI with the OS,
% which provides the AEE.  Just as applications interface with an AEE
% via an ABI, RISC-V operating systems interface with a supervisor
% execution environment (SEE) via a supervisor binary interface (SBI).
% An SBI comprises the user-level and supervisor-level ISA together with
% a set of SBI function calls.  Using a single SBI across all SEE
% implementations allows a single OS binary image to run on any SEE.
% The SEE can be a simple boot loader and BIOS-style IO system in a
% low-end hardware platform, or a hypervisor-provided virtual machine in
% a high-end server, or a thin translation layer over a host operating
% system in an architecture simulation environment.

中间的图显示了一个支持多程序执行的传统操作系统。每个应用通过ABI与OS进行交互,
OS为其提供AEE。RISC-V操作系统与特权执行环境(SEE)通过一个特权二进制
接口(SBI)进行交互。SBI由用户级和监管级ISA以及一套SBI函数调用组成。通过
一个所有SEE实现都支持的SBI,一个操作系统映像可以运行在任意的SEE上。
SEE可能是底层硬件平台的一个简单的启动加载器和BIOS风格的IO系统,也可能是一个
高端服务器里面的提供支持超级监管器的虚拟机,或者是一个体系架构模拟环境里的宿主操作系统上的
一个瘦中继层。
\begin{commentary}
% Most supervisor-level ISA definitions do not separate the SBI from the
% execution environment and/or the hardware platform, complicating
% virtualization and bring-up of new hardware platforms.

大部分监管级ISA定义不会将SBI与可执行环境或/和硬件平台分开(那样将会使虚拟化复杂
并产生新的硬件平台)
\end{commentary}

% The rightmost configuration shows a virtual machine monitor
% configuration where multiple multiprogrammed OSs are supported by a
% single hypervisor.  Each OS communicates via an SBI with the
% hypervisor, which provides the SEE.  The hypervisor communicates with
% the hypervisor execution environment (HEE) using a hypervisor binary
% interface (HBI), to isolate the hypervisor from details of the
% hardware platform.

最右边的配置展示了一个虚拟机监视器的配置,在其中,一个单一的超级监管器支持
多个多进程操作系统。每个操作系统通过SBI与提供SEE的超级监管器进行交互。
超级管理进程以HBI接口的形式与HEE进行交互,从而将超级监管器与具体的硬件平台解耦。

\begin{commentary}
% The ABI, SBI, and HBI are still a work-in-progress, but we are now
% prioritizing support for Type-2 hypervisors where the SBI is provided
% recursively by an S-mode OS.

ABI、SBI、和HBI规范依然在演进,我们优先支持第二种类型的超级进程,这种模式里面,
SBI由S模式(监管器模式)操作系统递归提供支持。
\end{commentary}

% Hardware implementations of the RISC-V ISA will generally require
% additional features beyond the privileged ISA to support the various
% execution environments (AEE, SEE, or HEE).

为了支持不同的执行环境(AEE,SEE,HEE),RISC-V ISA的硬件实现需要
在特权ISA的基础上增加额外的特征。

\section{特权级别
  % Privilege Levels
  }

% At any time, a RISC-V hardware thread ({\em hart}) is running at some
% privilege level encoded as a mode in one or more CSRs (control and
% status registers).  Three RISC-V privilege levels are currently defined
% as shown in Table~\ref{privlevels}.

在任意时刻,一个RISC-V硬件线程({\em hart})运行在某个特权层级(以一个或多个CSR进行编码)。
表~\ref{privlevels}定义了三种RISC-V特权层级。

\begin{table*}[h!]
\begin{center}
\begin{tabular}{|c|c|c|c|}
  \hline
   级别 & 编码        & 名称      & 缩写 \\ \hline  
   0     & \tt 00   & 用户(User)/应用(Application) & U     \\ 
   1     & \tt 01   & 监管器(Supervisor) & S           \\ 
   2     & \tt 10   & {\em 保留的} &            \\ 
   3     & \tt 11   & (机器)Machine    & M           \\ 
  \hline
 \end{tabular}
\end{center}
\caption{RISC-V特权级别。
  % RISC-V privilege levels.
  }
\label{privlevels}
\end{table*}

% Privilege levels are used to provide protection between different
% components of the software stack, and attempts to perform operations
% not permitted by the current privilege mode will cause an exception to
% be raised.  These exceptions will normally cause traps into an
% underlying execution environment.

特权级别为软件栈的不同组件提供保护,试图在当前特权模式执行不允许的操作将
导致抛出异常。这种异常一般会导致陷入下一层的执行环境。

\begin{commentary}
% In the description, we try to separate the privilege level for which
% code is written, from the privilege mode in which it runs, although
% the two are often tied.  For example, a supervisor-level operating
% system can run in supervisor-mode on a system with three privilege
% modes, but can also run in user-mode under a classic virtual machine
% monitor on systems with two or more privilege modes.  In both cases,
% the same supervisor-level operating system binary code can be used,
% coded to a supervisor-level SBI and hence expecting to be able to use
% supervisor-level privileged instructions and CSRs.  When running a
% guest OS in user mode, all supervisor-level actions will be trapped
% and emulated by the SEE running in the higher-privilege level.

在本文的描述中,我们试图将写代码的特权级别与其运行的特权模式相区分,
虽然二者通常紧密联系在一起。例如,一个监管级操作系统可以三种特权
模式运行在系统的监管器模式,它也可以在经典虚拟机下以用户模式运行
在具有两个或多个特权模式的系统上运行。在这两种情况下,同一个超级层级操作系统
二进制代码可以被编码为监管级SBI,从而能够使用监管级特权指令和相应的CSR。
当在用户模式运行客户操作系统时,所有的监管级行为被捕获并被更高特权级运行
的SEE模拟。
\end{commentary}

% The machine level has the highest privileges and is the only mandatory
% privilege level for a RISC-V hardware platform.  Code run in
% machine-mode (M-mode) is usually inherently trusted, as it has
% low-level access to the machine implementation.  M-mode can be used to
% manage secure execution environments on RISC-V.  User-mode (U-mode)
% and supervisor-mode (S-mode) are intended for conventional application
% and operating system usage respectively.

机器层级拥有最高的特权,它是RISC-V硬件平台的唯一强制性特权层级。在机器模式下运行
的代码被认为是天然可信的,它可以直接访问计算机底层硬件。机器模式用于管理RISC-V的
安全可执行环境。用户模式和监管器模式分别适用于传统的应用程序和操作系统。

% Each privilege level has a core set of privileged ISA extensions with optional
% extensions and variants.  For example, machine-mode supports an optional
% standard extension for memory protection.  Also, supervisor mode can be
% extended to support Type-2 hypervisor execution as described in
% Chapter~\ref{hypervisor}.

每一个特权层级拥有一套核心特权ISA及其可选的扩展和变体。例如,机器模式针对内存保护
支持可选的标准扩展。此外,监管器模式可以扩展以支持type-2类超级监管器执行,如
第~\ref{hypervisor}中所述。


% Implementations might provide anywhere from 1 to 3 privilege modes
% trading off reduced isolation for lower implementation cost, as shown
% in Table~\ref{privcombs}.

具体实现可以提供编号为从1到3的特权模式,以降低底层实现之间的解耦成本。如表~\ref{privcombs}所示。


\begin{table*}[h!]
\begin{center}
\begin{tabular}{|c|l|l|}
  \hline
   级别编号 &  支持的模式 & 预期的用途 \\ \hline  
   1     & M          & 简单嵌入式系统 \\ 
   2     & M, U       & 安全嵌入式系统 \\ 
   3     & M, S, U    & 运行类Unix操作系统的系统\\ 
  \hline
 \end{tabular}
\end{center}
\caption{支持的特权模式组合
  % Supported combinations of privilege modes.
  }
\label{privcombs}
\end{table*}

% All hardware implementations must provide M-mode, as this is the only
% mode that has unfettered access to the whole machine.  The simplest
% RISC-V implementations may provide only M-mode, though this will
% provide no protection against incorrect or malicious application code.

所有硬件实现必须提供机器模式,因为它是唯一的
可以不受限制地访问整个机器的模式。最简单的RISC-V实现可以只提供机器模式,
但是这样就会缺少对导致错误访问或恶意代码的保护。

\begin{commentary}
  % The lock feature of the optional PMP facility can provide some
  % limited protection even with only M-mode implemented.
  
  可选PMP的锁特性能为仅仅提供机器模式的实现提供有限的保护。
\end{commentary}

% Many RISC-V implementations will also support at least user mode
% (U-mode) to protect the rest of the system from application code.
% Supervisor mode (S-mode) can be added to provide isolation between a
% supervisor-level operating system and the SEE.

为了对除应用程序代码的其余部分提供保护,许多RISC-V系统的实现至少支持用户模式(U-模式)。
同时,为了保证监管级操作系统与SEE的隔离,监管器模式(S-模式)也可以添加到RISC-V的实现中。

% A hart normally runs application code in U-mode until some trap (e.g.,
% a supervisor call or a timer interrupt) forces a switch to a trap
% handler, which usually runs in a more privileged mode. The hart will
% then execute the trap handler, which will eventually resume execution
% at or after the original trapped instruction in U-mode.  Traps that
% increase privilege level are termed {\em vertical} traps, while traps
% that remain at the same privilege level are termed {\em horizontal}
% traps.  The RISC-V privileged architecture provides flexible routing
% of traps to different privilege layers.

在hart里,应用代码通常运行在用户模式下,当捕获到某个陷阱时(例如:一个监管器调用或一个时钟中断),hart会切换到特权级别更高的陷阱处理程序。
这个时候,hart会运行陷阱处理程序,在陷阱处理程序执行完之后,它将返回到用户模式中产生陷阱的那条指令,或下一条指令继续执行。
增加特权层级的陷阱称之为{\em 垂直} 陷阱,保持原有特权层级的陷阱称之为{\em 水平}陷阱。RISC-V
特权架构为不同的特权层级提供陷阱灵活的路由。

\begin{commentary}
% Horizontal traps can be implemented as vertical traps that
% return control to a horizontal trap handler in the less-privileged mode.

水平陷阱也可以通过在低特权即模式下返回一个水平陷阱的控制去实现垂直陷阱。
\end{commentary}

\section{调试模式
  % Debug Mode
  } 

% Implementations may also include a debug mode to support off-chip
% debugging and/or manufacturing test.  Debug mode (D-mode) can be
% considered an additional privilege mode, with even more access than
% M-mode. The separate debug specification proposal describes operation
% of a RISC-V hart in debug mode.  Debug mode reserves a few CSR
% addresses that are only accessible in D-mode, and may also reserve
% some portions of the physical address space on a platform.

RISC-V的某种实现还可能包括一个调试模式,去支持片外的调试模式调试和/或制造测试。 
调试模式(D 模式)可以认为是一种附加的特权模式,它甚至具有比机器模式更多的访问权限。
单独的调试规范提案描述了调试模式下的RISC-V hart的操作行为。调试模式只保留一些能在
D模式下访问的CSR地址,也保留了平台上物理地址空间上的一些部分。

