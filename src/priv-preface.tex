\chapter{Preface}

\iffalse
This document describes the RISC-V privileged architecture.
This release, version \privrev, contains the following versions of the RISC-V
ISA modules:
\fi
本文档描述了RISC-V特权体系结构。此版本(20211203)包含以下版本的RISC-V ISA模块

{
\begin{table}[hbt]
  \centering
  \begin{tabular}{|c|l|c|}
    \hline
%    Module                & Version  & Status\\
    模块                  & 版本      & 状态\\
    \hline
\iffalse
    \bf Machine ISA       & \bf 1.12 & \bf Ratified \\
    \bf Supervisor ISA    & \bf 1.12 & \bf Ratified \\
    \bf Svnapot Extension & \bf 1.0  & \bf Ratified \\
    \bf Svpbmt Extension  & \bf 1.0  & \bf Ratified \\
    \bf Svinval Extension & \bf 1.0  & \bf Ratified \\
    \bf Hypervisor ISA    & \bf 1.0  & \bf Ratified \\
\fi
    \bf Machine ISA       & \bf 1.12 & \bf 已批准 \\
    \bf Supervisor ISA    & \bf 1.12 & \bf 已批准 \\
    \bf Svnapot Extension & \bf 1.0  & \bf 已批准 \\
    \bf Svpbmt Extension  & \bf 1.0  & \bf 已批准 \\
    \bf Svinval Extension & \bf 1.0  & \bf 已批准 \\
    \bf Hypervisor ISA    & \bf 1.0  & \bf 已批准 \\
    \hline
  \end{tabular}
\end{table}
}

\iffalse
The following changes have been made since version 1.11, which, while not
strictly backwards compatible, are not anticipated to cause software
portability problems in practice:
\fi
自1.11版本以来,已经做出了以下更改,虽然不是严格的
向后兼容,但在实践中预计不会引起软件可移植性问题:

\vspace{-0.2in}
\begin{itemize}
  \parskip 0pt
  \itemsep 1pt
\iffalse
\item Changed MRET and SRET to clear {\tt mstatus}.MPRV when leaving M-mode.
\item Reserved additional {\tt satp} patterns for future use.
\item Stated that the {\tt scause} Exception Code field must implement
  bits 4--0 at minimum.
\item Relaxed I/O regions have been specified to follow RVWMO.  The previous
  specification implied that PPO rules other than fences and acquire/release
  annotations did not apply.
\item Constrained the LR/SC reservation set size and shape when using
  page-based virtual memory.
\item PMP changes require an SFENCE.VMA on any hart that implements
  page-based virtual memory, even if VM is not currently enabled.
\item Allowed for speculative updates of page table entry A bits.
\item Clarify that if the address-translation algorithm non-speculatively
  reaches a PTE in which a bit reserved for future standard use is set,
  a page-fault exception must be raised.
\fi
\item 当离开M模式时,更改MRET和SRET来清除{\tt mstatus}.MPRV。
\item 保留额外的{\tt satp}模式供将来使用。
\item 声明{\tt scause}异常代码字段必须至少实现bits 4--0。
\item 已根据RVWMO规定了宽松的I/O区域。之前的规范
      暗示除了fences 和 acquire/release annotations以外的PPO规则不适用。
\item 在使用基于页面的虚拟内存时,限制LR/SC预留集的大小和形状?。
\item 任何实现了基于页面的虚拟内存的hart上,PMP更改需要一个SFENCE.VMA,即使VM目前没有启用。
\item 允许对页表项A位进行推测性更新。
\item 说明如果地址转换算法非推测性地到达一个PTE,
      其中设置了一个为未来标准使用而保留的比特位,则必须引发缺页异常。
\end{itemize}

%Additionally, the following compatible changes have been made since version
此外,自版本以来还进行了以下兼容更改
1.11:
\vspace{-0.2in}
\begin{itemize}
  \parskip 0pt
  \itemsep 1pt
%\item Removed the N extension.
\item 删除N扩展
%\item Defined the mandatory RV32-only CSR {\tt mstatush}, which contains
%  most of the same fields as the upper 32 bits of RV64's {\tt mstatus}.
\item 定义了强制性RV32-only的CSR {\tt mstatush},
      它包含了与RV64的{\tt mstatus}高32位相同的大部分字段。
%\item Defined the mandatory CSR {\tt mconfigptr}, which if nonzero
%  contains the address of a configuration data structure.
\item 定义了强制性CSR{\tt mconfigptr},如果它不为零,则包含配置数据结构的地址。
%\item Defined optional {\tt mseccfg} and {\tt mseccfgh} CSRs, which control
% the machine's security configuration.
\item 定义了可选的{\tt mseccfg}和{\tt mseccfgh} CSRs,它们控制计算机的安全配置。
%\item Defined {\tt menvcfg}, {\tt henvcfg}, and {\tt senvcfg} CSRs
%  (and RV32-only {\tt menvcfgh} and {\tt henvcfgh} CSRs),
%  which control various characteristics of the execution environment.
\item 定义了{\tt menvcfg}、{\tt henvcfg}和{\tt senvcfg} CSRs(和RV32-only{\tt menvcfgh} 
      和 {\tt henvcfgh} CSRs),它们控制执行环境的各种特性。
%\item Designated part of SYSTEM major opcode for custom use.
\item 自定义使用SYSTEM主要操作码的指定部分。
%\item Permitted the unconditional delegation of less-privileged interrupts.
\item 允许对特权级较低的中断进行无条件授权。
%\item Added optional big-endian and bi-endian support.
\item 增加了可选的big-endian和bi-endian支持。
%\item Made priority of load/store/AMO address-misaligned exceptions
%  implementation-defined relative to load/store/AMO page-fault
%  and access-fault exceptions.
\item 相对于加载/存储/AMO页面错误和访问错误异常,优先考虑实现定义的加载/存储/AMO地址错误异常。
%\item PMP reset values are now platform-defined.
\item PMP重置值现在是平台定义的。
%\item An additional 48 optional PMP registers have been defined.
\item 另外还定义了48个可选的PMP寄存器。
%\item Slightly relaxed the atomicity requirement for A and D bit updates
  % performed by the implementation.
\item 稍微放宽了实现执行对A和D位更新的原子性要求。
% \item Clarify the architectural behavior of address-translation caches
\item 阐明地址转换高速缓存的体系结构行为
% \item Added Sv57 and Sv57x4 address translation modes.
\item 增加了Sv57和Sv57x4地址转换模式。
% \item Software breakpoint exceptions are permitted to write either 0
  % or the PC to {\em x}\/{\tt tval}.
\item 允许软件断点异常将0或PC写入 {\em x}\/{\tt tval}。
% \item Clarified that bare S-mode need not support the SFENCE.VMA instruction.
\item 阐明了裸S模式不需要支持SFENCE.VMA指令。
% \item Specified relaxed constraints for implicit reads of non-idempotent
  % regions.
\item 指定非幂等区域的隐式读取的宽松约束。
% \item Added the Svnapot Standard Extension, along with the N bit in
  % Sv39, Sv48, and Sv57 PTEs.
\item 增加了Svnapot标准扩展,以及Sv39、Sv48和Sv57 PTEs中的N位。
% \item Added the Svpbmt Standard Extension, along with the PBMT bits in
  % Sv39, Sv48, and Sv57 PTEs.
\item 增加了Svpbmt标准扩展,以及Sv39、Sv48和Sv57 PTEs中的PBMT位。
% \item Added the Svinval Standard Extension and associated instructions.
\item 增加了Svinval标准扩展和相关指令。
\end{itemize}

% Finally, the hypervisor architecture proposal has been extensively revised.
最后,对管理程序体系结构提议进行了广泛修改。

\newpage

\section*{Preface to Version 1.11}

% This is version 1.11 of the RISC-V privileged architecture.
% The document contains the following versions of the RISC-V ISA
% modules:
这是RISC-V特权架构的1.11版本。
文档中包含的RISC-V ISA模块版本如下:
{
\begin{table}[hbt]
  \centering
  \begin{tabular}{|c|l|c|}
    \hline
    % Module             & Version  & Status\\
    模块             & 版本  & 状态 \\
    \hline
    % \bf Machine ISA    & \bf 1.11 & \bf Ratified \\
    \bf Machine ISA    & \bf 1.11 & \bf 已批准 \\
    % \bf Supervisor ISA & \bf 1.11 & \bf Ratified \\
    \bf Supervisor ISA & \bf 1.11 & \bf 已批准 \\
    % \em Hypervisor ISA & \em 0.3  & \em Draft \\
    \em Hypervisor ISA & \em 0.3  & \em 草案 \\
    \hline
  \end{tabular}
\end{table}
}

% Changes from version 1.10 include:
与1.10版本相比的变化包括:
\vspace{-0.2in}
\begin{itemize}
  \parskip 0pt
  \itemsep 1pt
  % \item Moved Machine and Supervisor spec to {\bf Ratified} status.
  \item 将Machine和Supervisor规格转变到{\bf 已批准}状态。
% \item Improvements to the description and commentary.
\item 对描述和注释的改进。
% \item Added a draft proposal for a hypervisor extension.
\item 添加了hypervisor扩展的提案草案。
% \item Specified which interrupt sources are reserved for standard use.
\item 指定哪些中断源被保留用于标准使用。
% \item Allocated some synchronous exception causes for custom use.
\item 为自定义使用分配了一些同步异常原因。
% \item Specified the priority ordering of synchronous exceptions.
\item 指定同步异常的优先级顺序。
% \item Added specification that xRET instructions may, but are not
  % required to, clear LR reservations if A extension present.
\item 增加规范,如果扩展存在,xRET指令可以(但不是必需)清除LR保留。
% \item The virtual-memory system no longer permits supervisor mode to execute
  % instructions from user pages, regardless of the SUM setting.
\item 虚拟内存系统不再允许管理模式执行来自用户页面的指令,无论SUM如何设置。
% \item Clarified that ASIDs are private to a hart, and added commentary about
  % the possibility of a future global-ASID extension.
\item 阐明了ASIDs对一个hart是私有的,并添加了关于未来global-ASID扩展的可能性的评论。
% \item SFENCE.VMA semantics have been clarified.
\item SFENCE.VMA的语义已经被阐明。
% \item Made the {\tt mstatus}.MPP field \warl, rather than \wlrl.
\item 使 {\tt mstatus}.MPP 字段是\warl,而不是\wlrl。
% \item Made the unused {\em x}{\tt ip} fields \wpri, rather than \wiri.
\item 使未使用的{\em x}{\tt ip}字段是\wpri,而不是\wiri。
% \item Made the unused {\tt misa} fields \warl, rather than \wiri.
\item 使未使用的{\tt misa}字段\warl,而不是\wiri。
% \item Made the unused {\tt pmpaddr} and {\tt pmpcfg} fields \warl, rather than \wiri.
\item 使未使用的{\tt pmpaddr}和{\tt pmpcfg}字段是\warl,而不是\wiri。
% \item Required all harts in a system to employ the same PTE-update scheme as each other.
\item 要求系统中的所有harts都使用相同的PTE更新方案。
% \item Rectified an editing error that misdescribed the mechanism by which
  % {\tt mstatus}.{\em x}IE is written upon an exception.
\item 修正了一个编辑错误,它描述了在异常情况下编写 {\tt mstatus}.{\em x} 的机制。
% \item Described scheme for emulating misaligned AMOs.
\item 描述了模拟失调AMOs的方案。
% \item Specified the behavior of the {\tt misa} and {\em x}{\tt epc} registers in
  % systems with variable IALIGN.
\item 指定在可变IALIGN的系统中{\tt misa} 和 {\em x}{\tt epc} 寄存器的行为。
% \item Specified the behavior of writing self-contradictory values to the
  % {\tt misa} register.
\item 指定将自相矛盾的值写入{\tt misa}寄存器的行为。
% \item Defined the {\tt mcountinhibit} CSR, which stops performance
  % counters from incrementing to reduce energy consumption.
\item 定义了{\tt mcountinhibit} CSR,它阻止性能计数器增加以减少能耗。
% \item Specified semantics for PMP regions coarser than four bytes.
\item 为大于四个字节的PMP区域指定语义。
% \item Specified contents of CSRs across XLEN modification.
\item 指定跨XLEN修改CSRs的内容。
% \item Moved PLIC chapter into its own document.
\item 将PLIC章节移动到自己的文档中。
\end{itemize}

\newpage

\section*{Preface to Version 1.10}

% This is version 1.10 of the RISC-V privileged
% architecture proposal.  Changes from version 1.9.1 include:
这是RISC-V特权架构提案的1.10版本。从1.9.1版本开始的更改包括:

\begin{itemize}
  \parskip 0pt
  \itemsep 1pt
% \item The previous version of this document was released under a
%   Creative Commons Attribution 4.0 International License by the
%   original authors, and this and future versions of this document will
%   be released under the same license.
\item 本文档的前一个版本是由原作者在知识共享署名4.0国际许可下发布的,
本文档和未来的版本将在相同的许可下发布。
% \item The explicit convention on shadow CSR addresses has been removed
%   to reclaim CSR space.  Shadow CSRs can still be added as needed.
\item 为了回收CSR空间,对影子内阁的CSR地址的显式约定被移除。仍然可以根据需要添加影子内阁的CSRs。
% \item The {\tt mvendorid} register now contains the JEDEC code of the
%   core provider as opposed to a code supplied by the Foundation.  This
%   avoids redundancy and offloads work from the Foundation.
\item {\tt mvendorid}寄存器现在包含核心提供者的JEDEC代码,
而不是由基金会提供的代码。这避免了冗余并将减轻基金会的工作。
% \item The interrupt-enable stack   has been simplified.
\item 支持中断的堆栈规程已经简化。
% \item An optional mechanism to change the base ISA used by supervisor
%   and user modes has been added to the {\tt mstatus} CSR, and the
%   field previously called Base in {\tt misa} has been renamed to {\tt
%     MXL} for consistency.
\item {\tt mstatus} CSR中增加了一种可选的机制来改变管理员和用户模式使用的base ISA,并且为了
一致性,之前在{\tt misa}中称为base的字段被重命名为{\tt MXL}。
% \item Clarified expected use of XS to summarize additional extension
%   state status fields in {\tt mstatus}.
\item 阐明了期望使用XS来总结{\tt mstatus}中的其他扩展状态的状态字段。
% \item Optional vectored interrupt support has been added to the
%   {\tt mtvec} and {\tt stvec} CSRs.
\item {\tt mtvec}和{\tt stvec} CSRs中添加了可选的向量中断支持。
% \item The SEIP and UEIP bits in the {\tt mip} CSR have been redefined
%   to support software injection of external interrupts.
\item {\tt mip} CSR中的SEIP和UEIP位被重新定义,以支持外部中断的软件注入。
  % \item The {\tt mbadaddr} register has been subsumed by a more
  %   general {\tt mtval} register that can now capture bad
  %   instruction bits on an illegal instruction fault to speed
  %   instruction emulation.
\item {\tt mbadaddr}寄存器已被一个更通用的{\tt-mtval}寄存器所包含,
该寄存器现在可以捕获非法指令错误上的错误指令位,以加快指令仿真。
% \item The machine-mode base-and-bounds translation and protection
%   schemes have been removed from the specification as part of moving
%   the virtual memory configuration to {\tt sptbr} (now {\tt satp}).  Some of the
%   motivation for the base and bound schemes are now covered by the PMP
%   registers, but space remains available in {\tt mstatus} to add these
%   back at a later date if deemed useful.
\item 机器模式的基础边界转换和保护方案已经从规范中删除,
作为将虚拟内存配置移动到{\tt sptbr}(现在是{\tt satp})的一部分。
base和bound方案的一些诱因现在已经由PMP寄存器覆盖,
但{\tt mstatus}中仍然有空间,如果认为有用,可以在以后添加这些寄存器。
% \item In systems with only M-mode, or with both M-mode and U-mode but
%   without U-mode trap support, the {\tt medeleg} and {\tt mideleg}
%     registers now do not exist, whereas previously they returned zero.
\item 在只支持M模式,或者同时支持M模式和U模式但不支持U模式陷阱的系统中,
{\tt medeleg}和{\tt mideleg}寄存器现在不存在,而以前它们返回0。
% \item Virtual-memory page faults now have {\tt mcause} values distinct from
%   physical-memory access faults.  Page-fault exceptions can now be
%   delegated to S-mode without delegating exceptions generated by PMA
  % and PMP checks.
\item 虚拟内存页面错误现在具有不同于物理内存访问错误的{\tt mcause}值。
现在可以将页面错误异常授权到S模式,而无需授权由PMA和PMP检查生成的异常。
% \item An optional physical-memory protection (PMP) scheme has been proposed.
\item 提出了一种可选的物理内存保护(PMP)方案。
% \item The supervisor virtual memory configuration has been moved from the
%   {\tt mstatus} register to the {\tt sptbr} register.  Accordingly, the
%   {\tt sptbr} register has been renamed to {\tt satp} (Supervisor Address
%   Translation and Protection) to reflect its broadened role.
\item 管理员虚拟内存配置已经从{\tt mstatus}寄存器移动到{\tt sptbr}寄存器。
因此,{\tt sptbr}寄存器被重命名为{\tt satp}(Supervisor Address Translation and Protection),
以反映其扩大的作用。
% \item The SFENCE.VM instruction has been removed in favor of the improved
%   SFENCE.VMA instruction.
\item SFENCE.VM指令已被删除,取而代之的是改进后的SFENCE.VMA指令。
% \item The {\tt mstatus} bit MXR has been exposed to S-mode via {\tt sstatus}.
\item {\tt mstatus}位MXR已通过{\tt sstatus}暴露到s模式。
% \item The polarity of the PUM bit in {\tt sstatus} has been inverted to
%   shorten code sequences involving MXR.  The bit has been renamed to SUM.
\item {\tt sstatus}中的PUM位的极性被颠倒,以缩短涉及MXR的代码序列。位已重命名为SUM。
% \item Hardware management of page-table entry Accessed and Dirty bits has
%   been made optional; simpler implementations may trap to software to
%   set them.
\item 页表条目访问位和脏位的硬件管理是可选的;更简单的实现可能会让设置它们的软件陷入陷阱。
% \item The counter-enable scheme has changed, so that S-mode can
%   control availability of counters to U-mode.
\item 计数器使能方案已更改,因此S模式可以控制计数器对U模式的可用性。
% \item H-mode has been removed, as we are focusing on recursive
%   virtualization support in S-mode.  The encoding space has been
%   reserved and may be repurposed at a later date.
\item H模式已经被删除,因为我们专注于s模式下的递归虚拟化支持。
      编码空间已被保留,以后可以重新使用。
% \item A mechanism to improve virtualization performance by
%   trapping S-mode virtual-memory management operations has been added.
\item 添加了一种通过捕获S模式虚拟内存管理操作来提高虚拟化性能的机制。
% \item The Supervisor Binary Interface (SBI) chapter has been removed, so
%   that it can be maintained as a separate specification.
\item Supervisor Binary Interface (SBI)章节已经被删除,因此它可以作为一个单独的规范来维护。
\end{itemize}

\newpage

\section*{Preface to Version 1.9.1}

% This is version 1.9.1 of the RISC-V privileged architecture
% proposal.  Changes from version 1.9 include:
这是RISC-V特权架构提案的1.9.1版本。从1.9版本开始的更改包括:

\begin{itemize}
  \parskip 0pt
  \itemsep 1pt
% \item Numerous additions and improvements to the commentary sections.
\item 对说明部分进行了大量的补充和改进。
% \item Change configuration string proposal to be use a search process
%   that supports various formats including Device Tree String and
%   flattened Device Tree.
\item 将配置字符串提案更改为使用支持各种格式的搜索过程,包括设备树字符串和扁平设备树。
% \item Made {\tt misa} optionally writable to support modifying base
%   and supported ISA extensions.  CSR address of {\tt misa} changed.
\item 使{\tt misa}可选的写入,以支持修改基础和支持的ISA扩展。{\tt misa}的CSR地址已更改。
% \item Added description of debug mode and debug CSRs.
\item 增加调试模式和调试CSRs的说明。
% \item Added a hardware performance monitoring scheme.  Simplified the
%   handling of existing hardware counters, removing privileged versions
%   of the counters and the corresponding delta registers.
\item 增加硬件性能监控方案。简化了现有硬件计数器的处理,删除了计数器的特权版本和相应的增量寄存器。
% \item Fixed description of SPIE in presence of user-level interrupts.
\item 修正了用户级中断时SPIE的描述。
\end{itemize}

% \chapter{Preface}

% This document describes the RISC-V privileged architecture.
% This release, version \privrev, contains the following versions of the RISC-V
% ISA modules:

% {
% \begin{table}[hbt]
%   \centering
%   \begin{tabular}{|c|l|c|}
%     \hline
%     Module                & Version  & Status\\
%     \hline
%     \em Machine ISA       & \em 1.13 & \em Draft \\
%     \bf Supervisor ISA    & \bf 1.12 & \bf Ratified \\
%     \bf Svnapot Extension & \bf 1.0  & \bf Ratified \\
%     \bf Svpbmt Extension  & \bf 1.0  & \bf Ratified \\
%     \bf Svinval Extension & \bf 1.0  & \bf Ratified \\
%     \bf Hypervisor ISA    & \bf 1.0  & \bf Ratified \\
%     \hline
%   \end{tabular}
% \end{table}
% }

% The following compatible changes have been made to the Machine ISA since version 1.12:
% \vspace{-0.2in}
% \begin{itemize}
%   \parskip 0pt
%   \itemsep 1pt
% \item Defined the {\tt misa}.V field to reflect that the V extension has been implemented.
% \end{itemize}

% \section*{Preface to Version 20211203}

% This document describes the RISC-V privileged architecture.
% This release, version 20211203, contains the following versions of the RISC-V
% ISA modules:

% {
% \begin{table}[hbt]
%   \centering
%   \begin{tabular}{|c|l|c|}
%     \hline
%     Module                & Version  & Status\\
%     \hline
%     \bf Machine ISA       & \bf 1.12 & \bf Ratified \\
%     \bf Supervisor ISA    & \bf 1.12 & \bf Ratified \\
%     \bf Svnapot Extension & \bf 1.0  & \bf Ratified \\
%     \bf Svpbmt Extension  & \bf 1.0  & \bf Ratified \\
%     \bf Svinval Extension & \bf 1.0  & \bf Ratified \\
%     \bf Hypervisor ISA    & \bf 1.0  & \bf Ratified \\
%     \hline
%   \end{tabular}
% \end{table}
% }

% The following changes have been made since version 1.11, which, while not
% strictly backwards compatible, are not anticipated to cause software
% portability problems in practice:
% \vspace{-0.2in}
% \begin{itemize}
%   \parskip 0pt
%   \itemsep 1pt
% \item Changed MRET and SRET to clear {\tt mstatus}.MPRV when leaving M-mode.
% \item Reserved additional {\tt satp} patterns for future use.
% \item Stated that the {\tt scause} Exception Code field must implement
%   bits 4--0 at minimum.
% \item Relaxed I/O regions have been specified to follow RVWMO.  The previous
%   specification implied that PPO rules other than fences and acquire/release
%   annotations did not apply.
% \item Constrained the LR/SC reservation set size and shape when using
%   page-based virtual memory.
% \item PMP changes require an SFENCE.VMA on any hart that implements
%   page-based virtual memory, even if VM is not currently enabled.
% \item Allowed for speculative updates of page table entry A bits.
% \item Clarify that if the address-translation algorithm non-speculatively
%   reaches a PTE in which a bit reserved for future standard use is set,
%   a page-fault exception must be raised.
% \end{itemize}

% Additionally, the following compatible changes have been made since version
% 1.11:
% \vspace{-0.2in}
% \begin{itemize}
%   \parskip 0pt
%   \itemsep 1pt
% \item Removed the N extension.
% \item Defined the mandatory RV32-only CSR {\tt mstatush}, which contains
%   most of the same fields as the upper 32 bits of RV64's {\tt mstatus}.
% \item Defined the mandatory CSR {\tt mconfigptr}, which if nonzero
%   contains the address of a configuration data structure.
% \item Defined optional {\tt mseccfg} and {\tt mseccfgh} CSRs, which control
%  the machine's security configuration.
% \item Defined {\tt menvcfg}, {\tt henvcfg}, and {\tt senvcfg} CSRs
%   (and RV32-only {\tt menvcfgh} and {\tt henvcfgh} CSRs),
%   which control various characteristics of the execution environment.
% \item Designated part of SYSTEM major opcode for custom use.
% \item Permitted the unconditional delegation of less-privileged interrupts.
% \item Added optional big-endian and bi-endian support.
% \item Made priority of load/store/AMO address-misaligned exceptions
%   implementation-defined relative to load/store/AMO page-fault
%   and access-fault exceptions.
% \item PMP reset values are now platform-defined.
% \item An additional 48 optional PMP registers have been defined.
% \item Slightly relaxed the atomicity requirement for A and D bit updates
%   performed by the implementation.
% \item Clarify the architectural behavior of address-translation caches
% \item Added Sv57 and Sv57x4 address translation modes.
% \item Software breakpoint exceptions are permitted to write either 0
%   or the {\tt pc} to {\em x}\/{\tt tval}.
% \item Clarified that bare S-mode need not support the SFENCE.VMA instruction.
% \item Specified relaxed constraints for implicit reads of non-idempotent
%   regions.
% \item Added the Svnapot Standard Extension, along with the N bit in
%   Sv39, Sv48, and Sv57 PTEs.
% \item Added the Svpbmt Standard Extension, along with the PBMT bits in
%   Sv39, Sv48, and Sv57 PTEs.
% \item Added the Svinval Standard Extension and associated instructions.
% \end{itemize}

% Finally, the hypervisor architecture proposal has been extensively revised.

% \newpage

% \section*{Preface to Version 1.11}

% This is version 1.11 of the RISC-V privileged architecture.
% The document contains the following versions of the RISC-V ISA
% modules:

% {
% \begin{table}[hbt]
%   \centering
%   \begin{tabular}{|c|l|c|}
%     \hline
%     Module             & Version  & Status\\
%     \hline
%     \bf Machine ISA    & \bf 1.11 & \bf Ratified \\
%     \bf Supervisor ISA & \bf 1.11 & \bf Ratified \\
%     \em Hypervisor ISA & \em 0.3  & \em Draft \\
%     \hline
%   \end{tabular}
% \end{table}
% }

% Changes from version 1.10 include:
% \vspace{-0.2in}
% \begin{itemize}
%   \parskip 0pt
%   \itemsep 1pt
%   \item Moved Machine and Supervisor spec to {\bf Ratified} status.
% \item Improvements to the description and commentary.
% \item Added a draft proposal for a hypervisor extension.
% \item Specified which interrupt sources are reserved for standard use.
% \item Allocated some synchronous exception causes for custom use.
% \item Specified the priority ordering of synchronous exceptions.
% \item Added specification that xRET instructions may, but are not
%   required to, clear LR reservations if A extension present.
% \item The virtual-memory system no longer permits supervisor mode to execute
%   instructions from user pages, regardless of the SUM setting.
% \item Clarified that ASIDs are private to a hart, and added commentary about
%   the possibility of a future global-ASID extension.
% \item SFENCE.VMA semantics have been clarified.
% \item Made the {\tt mstatus}.MPP field \warl, rather than \wlrl.
% \item Made the unused {\em x}{\tt ip} fields \wpri, rather than \wiri.
% \item Made the unused {\tt misa} fields \warl, rather than \wiri.
% \item Made the unused {\tt pmpaddr} and {\tt pmpcfg} fields \warl, rather than \wiri.
% \item Required all harts in a system to employ the same PTE-update scheme as each other.
% \item Rectified an editing error that misdescribed the mechanism by which
%   {\tt mstatus}.{\em x}IE is written upon an exception.
% \item Described scheme for emulating misaligned AMOs.
% \item Specified the behavior of the {\tt misa} and {\em x}{\tt epc} registers in
%   systems with variable IALIGN.
% \item Specified the behavior of writing self-contradictory values to the
%   {\tt misa} register.
% \item Defined the {\tt mcountinhibit} CSR, which stops performance
%   counters from incrementing to reduce energy consumption.
% \item Specified semantics for PMP regions coarser than four bytes.
% \item Specified contents of CSRs across XLEN modification.
% \item Moved PLIC chapter into its own document.
% \end{itemize}

% \newpage

% \section*{Preface to Version 1.10}

% This is version 1.10 of the RISC-V privileged
% architecture proposal.  Changes from version 1.9.1 include:

% \begin{itemize}
%   \parskip 0pt
%   \itemsep 1pt
% \item The previous version of this document was released under a
%   Creative Commons Attribution 4.0 International License by the
%   original authors, and this and future versions of this document will
%   be released under the same license.
% \item The explicit convention on shadow CSR addresses has been removed
%   to reclaim CSR space.  Shadow CSRs can still be added as needed.
% \item The {\tt mvendorid} register now contains the JEDEC code of the
%   core provider as opposed to a code supplied by the Foundation.  This
%   avoids redundancy and offloads work from the Foundation.
% \item The interrupt-enable stack discipline has been simplified.
% \item An optional mechanism to change the base ISA used by supervisor
%   and user modes has been added to the {\tt mstatus} CSR, and the
%   field previously called Base in {\tt misa} has been renamed to {\tt
%     MXL} for consistency.
% \item Clarified expected use of XS to summarize additional extension
%   state status fields in {\tt mstatus}.
% \item Optional vectored interrupt support has been added to the
%   {\tt mtvec} and {\tt stvec} CSRs.
% \item The SEIP and UEIP bits in the {\tt mip} CSR have been redefined
%   to support software injection of external interrupts.
%   \item The {\tt mbadaddr} register has been subsumed by a more
%     general {\tt mtval} register that can now capture bad
%     instruction bits on an illegal instruction fault to speed
%     instruction emulation.
% \item The machine-mode base-and-bounds translation and protection
%   schemes have been removed from the specification as part of moving
%   the virtual memory configuration to {\tt sptbr} (now {\tt satp}).  Some of the
%   motivation for the base and bound schemes are now covered by the PMP
%   registers, but space remains available in {\tt mstatus} to add these
%   back at a later date if deemed useful.
% \item In systems with only M-mode, or with both M-mode and U-mode but
%   without U-mode trap support, the {\tt medeleg} and {\tt mideleg}
%     registers now do not exist, whereas previously they returned zero.
% \item Virtual-memory page faults now have {\tt mcause} values distinct from
%   physical-memory access faults.  Page-fault exceptions can now be
%   delegated to S-mode without delegating exceptions generated by PMA
%   and PMP checks.
% \item An optional physical-memory protection (PMP) scheme has been proposed.
% \item The supervisor virtual memory configuration has been moved from the
%   {\tt mstatus} register to the {\tt sptbr} register.  Accordingly, the
%   {\tt sptbr} register has been renamed to {\tt satp} (Supervisor Address
%   Translation and Protection) to reflect its broadened role.
% \item The SFENCE.VM instruction has been removed in favor of the improved
%   SFENCE.VMA instruction.
% \item The {\tt mstatus} bit MXR has been exposed to S-mode via {\tt sstatus}.
% \item The polarity of the PUM bit in {\tt sstatus} has been inverted to
%   shorten code sequences involving MXR.  The bit has been renamed to SUM.
% \item Hardware management of page-table entry Accessed and Dirty bits has
%   been made optional; simpler implementations may trap to software to
%   set them.
% \item The counter-enable scheme has changed, so that S-mode can
%   control availability of counters to U-mode.
% \item H-mode has been removed, as we are focusing on recursive
%   virtualization support in S-mode.  The encoding space has been
%   reserved and may be repurposed at a later date.
% \item A mechanism to improve virtualization performance by
%   trapping S-mode virtual-memory management operations has been added.
% \item The Supervisor Binary Interface (SBI) chapter has been removed, so
%   that it can be maintained as a separate specification.
% \end{itemize}

% \newpage

% \section*{Preface to Version 1.9.1}

% This is version 1.9.1 of the RISC-V privileged architecture
% proposal.  Changes from version 1.9 include:

% \begin{itemize}
%   \parskip 0pt
%   \itemsep 1pt
% \item Numerous additions and improvements to the commentary sections.
% \item Change configuration string proposal to be use a search process
%   that supports various formats including Device Tree String and
%   flattened Device Tree.
% \item Made {\tt misa} optionally writable to support modifying base
%   and supported ISA extensions.  CSR address of {\tt misa} changed.
% \item Added description of debug mode and debug CSRs.
% \item Added a hardware performance monitoring scheme.  Simplified the
%   handling of existing hardware counters, removing privileged versions
%   of the counters and the corresponding delta registers.
% \item Fixed description of SPIE in presence of user-level interrupts.
% \end{itemize}