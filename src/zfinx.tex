\chapter{用于在整数寄存器中使用浮点的“Zfinx”、“Zdinx”、“Zhinx”、“Zhinxmin”标准扩展(1.0版本)}
% \chapter{``Zfinx'', ``Zdinx'', ``Zhinx'', ``Zhinxmin'': Standard Extensions for Floating-Point in Integer Registers, Version 1.0}
\label{sec:zfinx}

本章定义了“Zfinx”扩展(发音为“z-f-in-x”),为单精度浮点指令提供了与标准浮点F扩展类似的指令,但是操作在{\tt x}寄存器而不是{\tt f}寄存器上。
本章还定义了“Zdinx”、“Zhinx”和“Zhinxmin”扩展,为其他浮点精度提供了类似的指令。
% This chapter defines the ``Zfinx'' extension (pronounced ``z-f-in-x'')
% that provides instructions similar to those in the standard
% floating-point F extension for single-precision floating-point
% instructions but which operate on the {\tt x} registers instead of the
% {\tt f} registers.  This chapter also defines the ``Zdinx'',
% ``Zhinx'', and ``Zhinxmin'' extensions that provide similar
% instructions for other floating-point precisions.

\begin{commentary}
F扩展使用单独的{\tt f}寄存器进行浮点计算,以减少寄存器压力并为宽超标量简化寄存器文件端口的提供。
然而,额外的\wunits{128}{B}架构状态增加了最小实现成本。
通过消除{\tt f}寄存器,Zfinx扩展极大地减少了支持带有浮点指令集的简单RISC-V实现的成本。Zfinx也降低了上下文切换成本。
% The F extension uses separate {\tt f} registers for floating-point
% computation, to reduce register pressure and simplify the provision of
% register-file ports for wide superscalars.
% However, the additional \wunits{128}{B} of architectural state increases the
% minimal implementation cost.
% By eliminating the {\tt f} registers, the Zfinx extension substantially
% reduces the cost of simple RISC-V implementations with floating-point
% instruction-set support.
% Zfinx also reduces context-switch cost.

一般来说,假定F扩展存在的软件与假定Zfinx扩展存在的软件是不兼容的,反之亦然。
% In general, software that assumes the presence of the F extension
% is incompatible with software that assumes the presence of the Zfinx
% extension, and vice versa.
\end{commentary}

{\em 除了}转移指令FLW、FSW、FMV.W.X、FMV.X.W、C.FLW[SP]和C.FSW[SP]外,Zfinx扩展添加了F扩展所增加的所有指令。
% The Zfinx extension adds all of the instructions that the F extension
% adds, {\em except} for the transfer instructions FLW, FSW, FMV.W.X,
% FMV.X.W, C.FLW[SP], and C.FSW[SP].

\begin{commentary}
Zfinx软件使用整数的加载与存储在内存之间传递浮点值。在寄存器之间的传输既可以使用整数算术指令,也可以使用浮点符号注入指令。
% Zfinx software uses integer loads and stores to transfer floating-point values
% from and to memory.
% Transfers between registers use either integer arithmetic or floating-point
% sign-injection instructions.
\end{commentary}

这些F扩展指令的Zfinx变体具有相同的语义,只是当这样一条指令将访问{\tt f}寄存器时,改为访问具有相同编号的{\tt x}寄存器。
% The Zfinx variants of these F-extension instructions have the same semantics,
% except that whenever such an instruction would have accessed an {\tt f}
% register, it instead accesses the {\tt x} register with the same number.

\section{处理更窄的值}
% \section{Processing of Narrower Values}

宽度\mbox{{\em w} $<$ XLEN bits}位的浮点操作数占据{\tt x}寄存器的位\mbox{{\em w}-1:0}。对于{\em w}位的浮点操作会忽略位\mbox{XLEN-1:{\em w}}。
% Floating-point operands of width \mbox{{\em w} $<$ XLEN bits} occupy bits
% \mbox{{\em w}-1:0} of an {\tt x} register.
% Floating-point operations on {\em w}-bit operands ignore operand bits
% \mbox{XLEN-1:{\em w}}.

产生\mbox{{\em w} $<$ XLEN-bit}位结果的浮点操作用位\mbox{{\em w}-1}(符号位)填充位\mbox{XLEN-1:{\em w}}。
% Floating-point operations that produce \mbox{{\em w} $<$ XLEN-bit} results
% fill bits \mbox{XLEN-1:{\em w}} with copies of bit \mbox{{\em w}-1} (the
% sign bit).

\begin{commentary}
{\tt f}寄存器中采用的NaN装箱策略的设计足够支持重新编码的浮点格式。
不过,由于浮点操作数和整数操作数占据相同的寄存器,重新编码对于Zfinx不太实用。因此,减少了对NaN装箱的需求。
% The NaN-boxing scheme employed in the {\tt f} registers was designed to
% efficiently support recoded floating-point formats.
% Recoding is less practical for Zfinx, though, since the same registers
% hold both floating-point and integer operands.
% Hence, the need for NaN boxing is diminished.

当符号扩展的32位浮点数保存在RV64的{\tt x}寄存器时,匹配现有的RV64调用约定,这需要所有的32位类型在传递到x寄存器、
或从{\tt x}寄存器返回时,都被符号扩展。为了使架构更加规范,我们在RV32和RV64中都把这个式样扩展到了16位浮点数。
% Sign-extending 32-bit floating-point numbers when held in RV64 {\tt x}
% registers matches the existing RV64 calling conventions, which require all
% 32-bit types to be sign-extended when passed or returned in {\tt x} registers.
% To keep the architecture more regular, we extend this pattern to 16-bit
% floating-point numbers in both RV32 and RV64.
\end{commentary}

\section{Zdinx}
% \section{Zdinx}

Zdinx扩展提供了类似于双精度浮点的指令。Zdinx扩展需要Zfinx扩展。
% The Zdinx extension provides analogous double-precision floating-point
% instructions.
% The Zdinx extension requires the Zfinx extension.

{\em 除了}转移指令FLD、FSD、FMV.D.X、C.FLD[SP]和C.FSD[SP],Zdinx扩展添加了D扩展增加的所有指令。
% The Zdinx extension adds all of the instructions that the D extension
% adds, {\em except} for the transfer instructions FLD, FSD, FMV.D.X,
% FMV.X.D, C.FLD[SP], and C.FSD[SP].

这些D扩展指令的Zdinx变体具有相同的语义,只是当这样一条指令将访问{\tt f}寄存器时,改为访问具有相同编号的{\tt x}寄存器。
% The Zdinx variants of these D-extension instructions have the same semantics,
% except that whenever such an instruction would have accessed an {\tt f}
% register, it instead accesses the {\tt x} register with the same number.

\section{处理更宽的值}
% \section{Processing of Wider Values}

RV32Zdinx中的双精度操作数保存在对齐的{\tt x}寄存器对中,也就是说,寄存器数量必须是偶数。为双宽度浮点操作数使用未对齐的(奇数数量的)寄存器是{\em 保留的}。
% Double-precision operands in RV32Zdinx
% are held in aligned {\tt x}-register pairs, i.e.,
% register numbers must be even.
% Use of misaligned (odd-numbered) registers for double-width floating-point
% operands is {\em reserved}.

不论字节序如何,较低编号的寄存器都保存低序位,而较高编号的寄存器保存高序位:例如,
RV32Zdinx中,双精度操作数的位31:0可能保存在寄存器{\tt x14}中,同时该操作数的位63:32保存在{\tt x15}中。
% Regardless of endianness, the lower-numbered register holds the low-order
% bits, and the higher-numbered register holds the high-order bits: e.g., bits
% 31:0 of a double-precision operand in RV32Zdinx might be held in register
% {\tt x14}, with bits 63:32 of that operand held in {\tt x15}.

当一个双宽度浮点结果写入{\tt x0}时,整个写入是无效的:例如,对于RV32Zdinx,向{\tt x0}写入一个双精度结果并不会导致{\tt x1}被写入。
% When a double-width floating-point result is written to {\tt x0}, the entire
% write takes no effect: e.g., for RV32Zdinx, writing a double-precision result
% to {\tt x0} does not cause {\tt x1} to be written.

当{\tt x0}被用作一个双宽度浮点操作数时,完整的操作数是零——即,{\tt x1}是未访问的。
% When {\tt x0} is used as a double-width floating-point operand, the entire
% operand is zero---i.e., {\tt x1} is not accessed.

\begin{commentary}
没有提供按对加载和按对存储的指令,所以在RV32Zdinx中从内存或向内存转移双精度操作数需要两次加载或存储。
然而,寄存器的移动只需要一个单独的FSGNJ.D指令。
% Load-pair and store-pair instructions are not provided, so transferring
% double-precision operands in RV32Zdinx from or to memory requires
% two loads or stores.
% Register moves need only a single FSGNJ.D instruction, however.
\end{commentary}

\section{Zhinx}

Zhinx扩展提供了类似于半精度浮点的指令。Zhinx扩展需要Zfinx扩展。
% The Zhinx extension provides analogous half-precision floating-point
% instructions.
% The Zhinx extension requires the Zfinx extension.

{\em 除了}转移指令FLH、FSH、FMV.H.X和FMV.X.H外,Zhinx扩展添加了所有Zfh扩展增加的指令。
% The Zhinx extension adds all of the instructions that the Zfh extension
% adds, {\em except} for the transfer instructions FLH, FSH, FMV.H.X,
% and FMV.X.H.

这些Zfh扩展指令的Zhinx变体具有相同的语义,只是在这样一条指令将访问{\tt f}寄存器时,改为访问具有相同编号的{\tt x}寄存器。
% The Zhinx variants of these Zfh-extension instructions have the same semantics,
% except that whenever such an instruction would have accessed an {\tt f}
% register, it instead accesses the {\tt x} register with the same number.

\section{Zhinxmin}

Zhinxmin扩展提供了对操作在{\tt x}寄存器上的16位半精度浮点指令的最小限度的支持。Zhinxmin扩展需要Zfinx扩展。
% The Zhinxmin extension provides minimal support for 16-bit half-precision
% floating-point instructions that operate on the {\tt x} registers.
% The Zhinxmin extension requires the Zfinx extension.

Zhinxmin扩展包含来自Zhinx扩展的下列指令:FCVT.S.H和FCVT.H.S。如果Zdinx扩展存在,也包含了FCVT.D.H和FCVT.H.D指令。
% The Zhinxmin extension includes the following instructions from the Zhinx
% extension: FCVT.S.H and FCVT.H.S.
% If the Zdinx extension is present, the FCVT.D.H and FCVT.H.D instructions are
% also included.

\begin{commentary}
在未来,可能定义与RV32Zdinx类似的RV64Zqinx四精度扩展。RV32Zqinx扩展也可能被定义,但是需要四寄存器组(译者注:指每组包含共同工作的四个寄存器,原文quad-register groups)。
% In the future, an RV64Zqinx quad-precision extension could be defined analogously
% to RV32Zdinx.
% An RV32Zqinx extension could also be defined but would require
% quad-register groups.
\end{commentary}

\section{特权架构的影响}
% \section{Privileged Architecture Implications}

在第二卷定义的标准特权架构中,如果实现了Zfinx扩展,{\tt mstatus}域FS被硬布线为零,
并且FS不再影响浮点指令或{\tt fcsr}访问的陷入行为。
% In the standard privileged architecture defined in Volume II, the
% {\tt mstatus} field FS is hardwired to 0 if the Zfinx extension is
% implemented, and FS no longer affects the trapping behavior of
% floating-point instructions or {\tt fcsr} accesses.

当实现了Zfinx扩展时,{\tt misa}位F、D和Q被硬布线为零。
% The {\tt misa} bits F, D, and Q are hardwired to 0 when the Zfinx
% extension is implemented.

\begin{commentary}
    未来的发现机制可能被用于探查Zfinx、Zhinx和Zdinx扩展的存在性。
% A future discoverability mechanism might be used to probe the existence
% of the Zfinx, Zhinx, and Zdinx extensions.
\end{commentary}
